%!TEX TS-program = xelatex
%!TEX encoding = UTF-8 Unicode

\def \papersize {a4paper}
\documentclass[12pt,\papersize]{extarticle}
% extarticle is like article but can handle 8pt, 9pt, 10pt, 11pt, 12pt, 14pt, 17pt, and 20pt text

\def \ititle {Origins of Mind: \\ Philosophical Issues in Cognitive Development}
\def \isubtitle {}
\def \iauthor {Università degli Studi di Milano, Autumn 2014-5}
\def \iemail{s.butterfill@warwick.ac.uk}
%\def \iauthor {}
%\def \iemail{}
\date{}


%\input{$HOME/Documents/submissions/preamble_steve_report}
\input{$HOME/Documents/submissions/preamble_steve_paper3}


%for putting citations into main text (for reading):
% use bibentry command
% nb this doesn’t work with mynewapa style; use apalike for \bibliographystyle
% nb2: use \nobibliography to introduce the readings 
\usepackage{bibentry}


%adjust title size & font
\pretitle{
	\begin{center}
	%\sffamily
	\LARGE
} 

%\author{}

%change fonts (not anonymous)
%\setromanfont[Mapping=tex-text]{Linux Libertine O} 
%\setsansfont[Mapping=tex-text]{Linux Biolinum O} 
%\setmonofont[Scale=MatchLowercase]{Andale Mono}


%use titlesec package to make subsections bold italics
%\titleformat*{\section}{\bfseries\itshape}  


%\setromanfont[Mapping=tex-text]{Sabon LT Std} 

\begin{document}

\setlength\footnotesep{1em}

\bibliographystyle{apalike} %newapa

\nobibliography{$HOME/endnote/phd_biblio}

\tolerance=5000


\maketitle
%\tableofcontents

% disables chapter, section and subsection numbering
\setcounter{secnumdepth}{-1} 

%\begin{abstract}
%\noindent
%***
%\end{abstract}

\section{Introduction}
How do humans first come to know about things like objects, causes, words, numbers, colours, actions and minds?  
This question goes back to Plato or earlier and remains unanswered.
%Philosophers have been pursuing this question since Plato or before. 
%More recently it has become the focus of a branch of psychology, developmental psychology.
%So far no one has convincingly answered the question, but two major breakthroughs have been made in the last three decades or so.
Two recent scientific breakthroughs appear to bring us closer to an answer, and to show that the question is even less straightforward than previously  assumed.
The first breakthrough concerns social interaction;
it is the discovery that preverbal infants enjoy surprisingly rich social abilities.
These abilities enable infants to engage in some forms of social interaction. 
This may well facilitate the subsequent acquisition of linguistic abilities and enable the emergence of knowledge \citep[e.g.][]{Csibra:2009xr,Meltzoff:2007pj,Tomasello:2005wx}. 
A second breakthrough involves the use of increasingly sensitive---and sometimes controversial---methods to detect  expectations without relying on subjects' abilities to talk or act.  
These methods have revealed sophisticated expectations about  causal interactions, numerosity, actions, mental states and more besides in preverbal infants \citep[e.g.][]{Spelke:1990jn,Baillargeon:gx}.
These expectations or the representations and processes underpinning them arguably also enable the emergence of knowledge. 
This course aims to introduce readers to  
%example of new---the question about simple joint action
new philosophical issues raised by these findings  
%example of existing question: modularity
and to explain their relevance to longstanding philosophical questions about the mind.  

\section{Outline of Lectures}
The course will be organised by domains of knowledge, so that one session concerns knowledge of objects, another knowledge of causes, and so on.  
See table \vref{table:schedule} for a provisional schedule.
The schedule may change depending on group discussion and research interests.

{
	%increase space between rows
	\renewcommand{\arraystretch}{1.5}
\begin{table}[htbp]
\begin{center}
\footnotesize	%shrink for better spacing
\begin{tabular*}{1\textwidth}{ l l m{0.80\textwidth} } 

\toprule

\newcounter{num}
\stepcounter{num}
\arabic{num}. & sept 23 
	&  \textit{Objects}
		\newline Reading: \citet{Spelke:1998im,moore:2008_factors}
\\  \stepcounter{num}
\arabic{num}. & sept 23
	& \textit{Causes}
		\newline Reading: \citet{Spelke:1993no,Hood:2000bf}
\\ \stepcounter{num}
\arabic{num}. & sept 23 
	& \textit{Colours}
		\newline  Reading: \citet{Franklin:2005hp,Kowalski:2006hk}
%
\\
%
% go from colours to language --- illustrates role for cooperation
% to understand how that might be so, we need to investigate language
%
\midrule
%
\stepcounter{num}
\arabic{num}. & sept 24 
	& 	\textit{Languages}
		\newline Reading: \citet{lidz:2003_what, lidz:2004_reaffirming}
\\ \stepcounter{num}
\arabic{num}. & sept 24 
	& 	\textit{Communication}
		\newline Reading: \citet{Tomasello:2007fi,Baldwin:2000qq}
\\ \stepcounter{num}
\arabic{num}. & sept 24 
	&  \textit{Minds}
		\newline Reading: \citet{Baillargeon:gx}
%
\\
%
\midrule

\stepcounter{num}
\arabic{num}. & sept 25  am
	& 	\textit{Actions}
		\newline Reading: \citet{Csibra:2003kp}

\\
%
\bottomrule
%
\end{tabular*}
\caption{Provisional schedule}
\label{table:schedule}
\end{center}	%careful -- position of this affects distance between table and caption(!)
\end{table}
}




\section{Preparatory Reading}
It would be useful to read these items before the course, ideally in addition to the papers in  table \vref{table:schedule}.

\begin{itemize}
%\item \bibentry{Fodor:1983dg}
\item \bibentry{Moll:2007gu}
\item \bibentry{robbins:2010_modularity}
\item \bibentry{Spelke:2007hb}
\end{itemize}



\bibliography{$HOME/endnote/phd_biblio}

\end{document}