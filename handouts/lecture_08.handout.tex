%!TEX TS-program = xelatex
%!TEX encoding = UTF-8 Unicode

\documentclass[12pt]{extarticle}
% extarticle is like article but can handle 8pt, 9pt, 10pt, 11pt, 12pt, 14pt, 17pt, and 20pt text

\def \ititle {Origins of Mind}

\def \isubtitle {Lecture 08}

\def \iauthor {Stephen A. Butterfill}
\def \iemail{s.butterfill@warwick.ac.uk}
\date{}

%for strikethrough
\usepackage[normalem]{ulem}

\input{$HOME/Documents/submissions/preamble_steve_handout}

%\bibpunct{}{}{,}{s}{}{,}  %use superscript TICS style bib
%remove hanging indent for TICS style bib
%TODO doesnt work
\setlength{\bibhang}{0em}
%\setlength{\bibsep}{0.5em}


%itemize bullet should be dash
\renewcommand{\labelitemi}{$-$}

\begin{document}

\begin{multicols}{3}

\setlength\footnotesep{1em}


\bibliographystyle{newapa} %apalike

%\maketitle
%\tableofcontents




%---------------
%--- start paste
\def \ititle {Lecture 08}

\begin{center}

{\Large

\textbf{\ititle}

}



\iemail %

\end{center}



\section{Joint Action: The Challenge}

‘participation in cooperative ... interactions … leads children to construct uniquely powerful forms of cognitive representation.’
\citep{Moll:2007gu}

‘perception, action, and cognition are grounded in social interaction’
\citep[p.\ 103]{Knoblich:2006bn}

‘human cognitive abilities … [are] built upon social interaction’
\citep{sinigaglia:2008_roots}



\section{What Is Joint Action? Bratman’s Account}

‘I take a collective action to involve a collective [shared] intention.’
\citep[p.~5]{Gilbert:2006wr}

‘The sine qua non of collaborative action is a joint goal [shared intention] and a joint commitment’
\citep[p.~181]{tomasello:2008origins}

‘the key property of joint action lies in its internal component [...] in the participants’ having
a “collective” or “shared” intention.’
\citep[pp.~444--5]{alonso_shared_2009}

‘Shared intentionality is the foundation upon which joint action is built.’
\citep[p.~381]{carpenter:2009_howjoint}

‘I will … adopt Bratman’s … influential formulation of joint action …
each partner needs to
intend to perform the joint action together ‘‘in accordance with and because of
meshing subplans’’ (p. 338) and this needs to be common knowledge between the participants.’
\citep[][p.\ 281]{carpenter:2009_howjoint}



\section{What Joint Action Could Not Be}

Objection: Meeting the sufficient conditions for joint action given by
Bratman’s account could not significantly \textit{explain} the development
of an understanding of minds because it already \textit{presupposes} too
much sophistication in the use of psychological concepts.

The objection arises because not all of the following claims are true:
%
\begin{quote}
(1) joint action fosters an understanding of minds;

(2) all joint action involves shared intention; and

(3) a function of shared intention is to coordinate two or more agents’ plans.
\end{quote}
%
These claims are inconsistent because if the second and third were both true, abilities to engage in joint action would presuppose, and so could not significantly foster, an understanding of minds.

What are our options?



\section{Development of Joint Action: Planning}

Objection: ‘Despite the common impression that joint action needs to be dumbed down
for infants due to their ‘‘lack of a robust theory of mind’’
... all the important social-cognitive building
blocks for joint action appear to be in place:
1-year-old infants
understand quite a bit about others’ goals and intentions and what
knowledge they share with others’
\citep[p.~383]{carpenter:2009_howjoint}.

‘I ... adopt Bratman’s (1992) influential formulation of joint
action or shared cooperative activity.
Bratman argued that in order for
an activity to be considered shared or joint each partner needs to intend
to perform the joint action together ‘‘in accordance with and because of
meshing subplans’’ (p. 338) and this needs to be common knowledge between
the participants’
\citep[p.~381]{carpenter:2009_howjoint}.

‘shared intentional agency [i.e. ‘joint action’] consists, at bottom, in
interconnected planning agency of the participants’ \citep{Bratman:2011fk}.

‘3- and 5-year-old children do not consider another person’s actions in
their own action planning (while showing action planning when acting
alone on the apparatus).
Seven-year-old children and adults
however, demonstrated evidence for joint action planning. ... While adult participants
demonstrated the presence of joint action planning from the very
first trials onward, this was not the case for the 7-year-old
children who improved their performance across trials.’
\citep[p.~1059]{paulus:2016_development}

‘by age 3 children are able to learn, under certain circumstances, to take
account of what a partner is doing in a collaborative problem-solving
context. By age 5 they are already quite skillful at attending to and even
anticipating a partner’s actions’ \citep[p.~57]{warneken:2014_young}.
‘proactive planning for two individuals, even
when they share a common goal, is more difficult than planning ahead solely
for oneself’ \citep[p.~128]{gerson:2016_social}.



\section{Development of Joint Action: Years 1-2}

‘By 12–18 months, infants are beginning to participate in a variety of
joint actions which show many of the characteristics of adult joint
action.’
\citep[p.~388]{carpenter:2009_howjoint}

‘infants learn about cooperation by participating in joint action
structured by skilled and knowledgeable interactive partners before they
can represent, understand, or generate it themselves. Cooperative joint
action develops in the context of dyadic interaction with adults in which
the adult initially takes responsibility for and actively structures the
joint activity and the infant progressively comes to master the
structure, timing, and communications involved in the joint action with
the support and guidance of the adult. ... Eager participants from the
beginning, it takes approximately 2 years for infants to become
autonomous contributors to sustained, goal-directed joint activity as
active, collaborative partners’
\citep[p.~200]{brownell:2011_early}.

‘While 4-year-olds coordinated the timing of drum hits, children between 2-
and 4 years of age showed indications of interpersonal coordination as
indicated by the beginnings and endings of drumming bouts. Children
showed more overlap in their bouts than would be expected by chance’
\citep[p.~720]{endedijk:2015_development}.

‘The 14-month-olds of this study displayed coordinated behaviors in the
elevator task Role A of positioning themselves in the right location and
retrieving the target object from the cylinder when the partner pushed it
up, but they had major problems performing Role B, pushing the cylinder up
and holding it in place until the partner could fetch the object. If they
pushed up the cylinder at all, they would repeatedly drop it when the other
person was just about to take the object out’ \citep{warneken:2007_helping}.

Infants’ ‘attempts to reactivate the
partner in interruption periods indicate that they were aware of the
interdependency of actions—that the execution of their own actions was
conditional on that of the partner ... these instances might also
exemplify a basic understanding of shared intentionality’
\citep[p.~290--1]{warneken:2007_helping}.

‘advances in infants’ ability to coordinate their behavior with one another
are associated with multiple measures of developing self-other
representations. One- and two-year olds’ symbolic representation of self
and other in pretend play (e.g., pretending that a doll is feeding itself)
was related to the amount of coordinated behavior they produced with a peer
on the structured cooperation tasks described above (Brownell and Carriger
1990)’
\citep[p.~206]{brownell:2011_early}.

‘children who better produced and comprehended language about their own and
others’ feelings and actions, and who could refer to themselves and others
using personal pronouns likewise monitored their peer’s behavior more often
and produced more joint activity with the peer (Brownell et al 2006)’
\citep[p.~206]{brownell:2011_early}.



\section{Collective Goals vs Shared Intentions}

‘all sorts of joint activity is possible without conscious goal
representations, complex reasoning, and advanced self-other understanding ...
both in other species and in our own
joint behavior as adults, some of which occurs outside of reflective
awareness ...
In studying
its development in children the problem is how to characterize and
differentiate primitive, lower levels of joint action operationally from more
complex and cognitively sophisticated forms’
\citep[p.~195]{brownell:2011_early}.

An outcome is a \emph{collective goal} of two or more actions involving multiple
agents if it is an outcome to which those actions are collectively directed \citep{butterfill:2016_minimal}.

For us to have a \emph{shared goal} $G$ is for $G$ to be a collective goal
of our present or future actions in virtue of the facts that:
\begin{enumerate}
\item We each expect the other(s) to perform an action directed to G.
\item We each expect that if G occurs, it will occur as a common effect of all of our actions.
\end{enumerate}
(Compare \citealp{Butterfill:2011fk,vesper_minimal_2010}.)

‘the basic skills and motivations for shared intentionality typically emerge
at around the first birthday from the interaction of two developmental
trajectories, each representing an evolutionary adaptation from some
different point in time.
The first trajectory is a general primate (or
perhaps great ape) line of development for understanding intentional action
and perception, which evolved in the context of primates’ crucially important
competitive interactions with one another over food, mates, and other
resources (Machiavellian intelligence; Byrne \& Whiten, 1988).
The second
trajectory is a uniquely human line of development for sharing psychological
states with others, which seems to be present in nascent form from very early
in human ontogeny as infants share emotional states with others in
turn-taking sequences (Trevarthen, 1979). The interaction of these two lines
of development creates, at around 1 year of age, skills and motivations for
sharing psychological states with others in fairly local social interactions,
and then later skills and motivations for reacting to and even internalizing
various kinds of social norms, collective beliefs, and cultural institutions’
\citep[p~124]{Tomasello:2007gl}.
 

%--- end paste
%---------------

\footnotesize
\bibliography{$HOME/endnote/phd_biblio}

\end{multicols}

\end{document}
