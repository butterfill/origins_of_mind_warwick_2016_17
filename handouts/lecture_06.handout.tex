%!TEX TS-program = xelatex
%!TEX encoding = UTF-8 Unicode

\documentclass[12pt]{extarticle}
% extarticle is like article but can handle 8pt, 9pt, 10pt, 11pt, 12pt, 14pt, 17pt, and 20pt text

\def \ititle {Origins of Mind}

\def \isubtitle {Lecture 06}

\def \iauthor {Stephen A. Butterfill}
\def \iemail{s.butterfill@warwick.ac.uk}
\date{}

%for strikethrough
\usepackage[normalem]{ulem}

\input{$HOME/Documents/submissions/preamble_steve_handout}

%\bibpunct{}{}{,}{s}{}{,}  %use superscript TICS style bib
%remove hanging indent for TICS style bib
%TODO doesnt work
\setlength{\bibhang}{0em}
%\setlength{\bibsep}{0.5em}


%itemize bullet should be dash
\renewcommand{\labelitemi}{$-$}

\begin{document}

\begin{multicols}{3}

\setlength\footnotesep{1em}


\bibliographystyle{newapa} %apalike

%\maketitle
%\tableofcontents




%---------------
%--- start paste





\def \ititle {Lecture 06}

\begin{center}

{\Large

\textbf{\ititle}

}



\iemail %

\end{center}



\section{Crossing the Gap}

\begin{enumerate}
\item Core knowledge exists.
\item There is a gap between core knowledge and knowledge knowledge.
\item Crossing the gap involves social interactions, perhaps involving words.
\end{enumerate}



\section{Action: The Basics}

When do human infants first track goal-directed actions
rather than mere movements only?

Background assumption:
‘intention attribution and action understanding are two separable processes’
\citep[p.~617]{uithol:2014_what}.

‘by the end of the first year infants are indeed capable of taking the intentional stance (Dennett, 1987) in interpreting the goal- directed behavior of rational agents.’
\citep[p.\ 184]{Gergely:1995sq}

‘12-month-old babies could identify the agent’s goal and analyze its actions causally in relation to it’
\citep[p.\ 190]{Gergely:1995sq}

'Six-month-olds and 9-month-olds showed a stronger novelty response (i.e., looked longer) on new-goal trials than on new-path trials (Woodward 1998). That is, like toddlers, young infants selectively attended to and remembered the features of the event that were relevant to the actor’s goal.'
\citep[p.\ 153]{woodward:2001_making}

‘Early in life, action expectations measured online seem to be organized
around goal locations whereas action expectations measured post-hoc around
goal identities. With increasing age, children then generally organize
their action expectations primarily around goal identities’
\citep[p.~10]{daum:2012_actions}



\section{The Teleological Stance}

‘in perceiving one object as having the intention of affecting another, the infant attributes to the object [...] intentions’
\citep[p.\ 14]{Premack:1990jl}
\citep[p.~53]{woodward:2009_infants}

‘to the extent that young infants are limited [...], their
understanding of intentions would be quite different from the mature
concept of intentions’
\citep[p.\ 168]{woodward:2001_making}.

‘by taking the intentional stance the infant can come to represent the agent’s action as intentional without actually attributing a mental representation of the future goal state’
\citep[p.\ 188]{Gergely:1995sq}

‘an action can be explained by a goal state if, and only if, it is seen as  the  most justifiable action towards that goal state that is available within the constraints of reality’
\citep[p.~255]{Csibra:1998cx}



\section{Marr’s Threefold Distinction}

If I apply the Teleological Stance successfully, do I thereby come to know a fact about the goal of an action?

\citet[p.~22ff]{Marr:1982kx} distinguishes:

\begin{itemize}

\item computational description---What is the thing for and how does it achieve this?

\item representations and algorithms---How are the inputs and outputs represented, and how is the transformation accomplished?

\item hardware implementation---How are the representations and algorithms physically realised?

\end{itemize}

‘when taking the teleological stance one-year-olds apply the same
inferential principle of rational action that drives everyday mentalistic
reasoning about intentional actions in adults’
(\citealp{Gergely:2003gb}; compare \citealp{Csibra:2003jv}, \citealp[p.~259]{Csibra:1998cx} )

`Such calculations require detailed knowledge of biomechanical factors that determine the motion capabilities and energy expenditure of agents.  However, in the absence of such knowledge, one can appeal to heuristics  that approximate the results of these calculations on the basis of knowledge in other domains that is certainly available to young infants.



\section{Action Observation by Adults}

\citet{Flanagan:2003lm} showed that
‘patterns of eye–hand coordination are similar when performing and observing a block stacking task’.

In human adults, motor representations and processes enable anticipatory looking
that is driven by goal ascription \citep[e.g.][]{Costantini:2012fk,ambrosini:2012_tie}.

See \citet{sinigaglia:2015_goal_ascription} for an outline of the Motor Theory of Goal Ascription.



\section{Performing vs Understanding Actions in Infancy}

From at least three months of age, some of infants’ abilities to identify
the goals of actions they observe are linked to their abilities to perform
actions \citep{woodward:2009_infants}.

In adults, tying the hands impairs proactive gaze \citep{ambrosini:2012_tie}; in
infants, boosting grasping with ‘sticky mittens’ facilitates proactive gaze
(\citealp{sommerville:2005_action}; see also \citealp{sommerville:2008_experience},
\citealp{ambrosini:2013_looking}).


    


%--- end paste
%---------------

\footnotesize
\bibliography{$HOME/endnote/phd_biblio}

\end{multicols}

\end{document}
