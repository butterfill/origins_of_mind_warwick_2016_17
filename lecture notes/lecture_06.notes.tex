 %!TEX TS-program = xelatex
%!TEX encoding = UTF-8 Unicode

%\def \papersize {a5paper}
\def \papersize {a4paper}
%\def \papersize {letterpaper}

%\documentclass[14pt,\papersize]{extarticle}
\documentclass[12pt,\papersize]{extarticle}
% extarticle is like article but can handle 8pt, 9pt, 10pt, 11pt, 12pt, 14pt, 17pt, and 20pt text

\def \ititle {Origins of Mind: Lecture Notes}
\def \isubtitle {Lecture 01}
%comment some of the following out depending on whether anonymous
\def \iauthor {Stephen A.\ Butterfill}
\def \iemail{s.butterfill@warwick.ac.uk% \& corrado.sinigaglia@unimi.it
}
%\def \iauthor {}
%\def \iemail{}
%\date{}

%\input{$HOME/Documents/submissions/preamble_steve_paper4}
\input{$HOME/Documents/submissions/preamble_steve_lecture_notes}

%no indent, space between paragraphs
\usepackage{parskip}

%comment these out if not anonymous:
%\author{}
%\date{}

%for e reader version: small margins
% (remove all for paper!)
%\geometry{headsep=2em} %keep running header away from text
%\geometry{footskip=1.5cm} %keep page numbers away from text
%\geometry{top=1cm} %increase to 3.5 if use header
%\geometry{bottom=2cm} %increase to 3.5 if use header
%\geometry{left=1cm} %increase to 3.5 if use header
%\geometry{right=1cm} %increase to 3.5 if use header

% disables chapter, section and subsection numbering
\setcounter{secnumdepth}{-1}

%avoid overhang
\tolerance=5000

%\setromanfont[Mapping=tex-text]{Sabon LT Std}


%for putting citations into main text (for reading):
% use bibentry command
% nb this doesn’t work with mynewapa style; use apalike for \bibliographystyle
% nb2: use \nobibliography to introduce the readings
\usepackage{bibentry}

%screws up word count for some reason:
%\bibliographystyle{$HOME/Documents/submissions/mynewapa}
\bibliographystyle{apalike}


\begin{document}



\setlength\footnotesep{1em}






%---------------
%--- start paste




\title {Origins of Mind \\ Lecture 06}



\maketitle

\subsection{title-slide}


\section{Crossing the Gap}

\subsection{slide-3}
How do humans first come to know simple facts about physical objects,
colours, minds and the rest?

\subsection{slide-4}
Core knowledge is real. Infants’ have unexpectedly sophisticated abilities concerning physical objects
and categorical colour properties (and much more) even from the first year of life.

\subsection{slide-5}
There is a gap between core knowledge and knowledge knowledge.  It takes months
if not years between clear manifestations of core knowledge and knowledge knowledge.
Importantly,

\subsection{slide-6}
Crossing the gap involves social interactions, perhaps involving words.

\subsection{slide-8}
Having core knowledge of something does not involve having any knowledge knowledge at all.
Here I'm going to use the term ‘concept of X’ for that which enables one to have knowledge
of Xs.
How do we get from core knowledge to concepts?

\subsection{slide-9}
Core knowledge enables one to distinguish things. For example it enables on to distinguish
those things which are blue from those which are not;
it enables one to distinguish those events which are causal interactions from those
which are not;
it enables one to distinguish those sets which have two members from others;
and it enables one to distinguish different beliefs about the location of an object (say).

(Here I'm using core knowledge in the broad, schematic sense to refer to representations
which are knowledge-like but not knowledge.)

\subsection{slide-10}
I conjecture that core knowledge faciliates acquisition of the correct use of a word,
perhaps very slowly.  The idea is that being able to discriminate things allows one to
apply a label to them.

Importantly we can discriminate without having concepts.
If one thought that all discrimination involved concepts, this picture would become circular.

\subsection{slide-11}
How does core knowledge enable one to correctly use words?
I think it modifies the overall phenomenal character of your experience,
typically by generating phenomenal expectations (which I called them perceptual expectations
earlier in this version of the course).
Tuning in to the perceptual expectations can take a long time, which is why there may be
a long interval between observing core knowledge and observing the correct use of words.

\subsection{slide-12}
I also conjecture that using the word facilitates concept acquisition.
Many people would probably agree.
But how does it do this?

My schematic suggestion is that using the word draws attention to all the things which are Xs.
The concept is acquired when you are struck by the question,
What do all these have in common?

(Clearly this is not an account of how thinking gets started at all; the appeal
to reflection should make this obvious.)

We have quite good evidence for this picture in the cases of colour and number,
and there is relevant evidence in the case of mindreading too.  (Also speech:
phonological awareness is linked to literacy and the particulars of the written
language learnt, so that alphabetic languages give a different profile --- alphabet is
roughly labelling phonemes.)

The question we've been looking at last week is how children come to correctly
use words.

\subsection{slide-13}
This is about the step from discrimination to learning the correct use of a verbal label.

So there's you and you're observing sequence of stimuli and thanks to core knowledge
you're able to discriminate them.

\subsection{slide-14}
And now along comes another person.  What are they doing?  Nothing yet.  But ...

\subsection{slide-15}
Oh look they're labelling stimuli.
So now the blue ones (say) are special.  You respond to them in one way and the
other responds to them in her way, which is by labelling.

\subsection{slide-16}
Now you can observe that your responses are correlated with her responses.
So when you discriminate in a certain way, she applies the label.
Observing this correspondence enables you to learn the label (say).
This is triangulation roughly as Davidson describes it.

And having got this far you can ask yourself what all the things labelled
have in common.

\subsection{slide-17}
Core knowledge is real. Infants’ have unexpectedly sophisticated abilities concerning physical objects
and categorical colour properties (and much more) even from the first year of life.

There is a gap between core knowledge and knowledge knowledge.  It takes months
if not years between clear manifestations of core knowledge and knowledge knowledge.
Importantly,

Crossing the gap involves social interactions, perhaps involving words.

\subsection{slide-19}
Gradually build up from understanding minds and actions to words.

\subsection{unit\_701}


\section{Action: The Basics}

\subsection{slide-25}
Our first question is, When do human infants first track goal-directed
actions and not just movements? In examining nonlinguistic communication,
we've assumed that infants from around 11 months of age can produce and
comprehend informative pointing. This commits us to saying that they have
understood action.

When do human infants first track goal-directed actions
rather than mere movements only?

\subsection{slide-26}
Background assumption:
‘intention attribution and action understanding are two separable processes’
\citep[p.~617]{uithol:2014_what}.

\subsection{slide-27}
\#source 'research/teleological stance -- csibra and gergely.doc'

\#source 'lectures/mindreading and joint action - philosophical tools (ceu budapest 2012-autumn fall)/lecture05 actions intentions goals'

\#source 'lectures/mindreading and joint action - philosophical tools (ceu budapest 2012-autumn fall)/lecture06 goal ascription teleological motor'

When do human infants first track goal-directed actions and not just movements?

Here's a classic experiment from way back in 1995.

The subjects were 12 month old infants.

They were habituated to this sequence of events.

\subsection{slide-28}
There was also a control group who were habituated to a display like this
one but with the central barrier moved to the right, so that the action
of the ball is 'non-rational'.

\subsection{slide-29}
For the test condition, infants were divided into two groups.  One saw a new action, ...

... the other saw an old action.

Now if infants were considering the movements only and ignoring information about the goal, the 'new action' (movement in a straight line) should be more interesting because it is most different.

But if infants are taking goal-related information into acction, the 'old action' might be unexpected and so might generate greater dishabituation.

\subsection{slide-31}
You might say, it's bizarre to have used balls in this study, that can't show us anything about infants' understanding of action.

But adult humans naturally interpret the movements of even very simple shapes in terms of goals.

So using even very simple stimuli doesn't undermine the interpretation of these results.

\subsection{slide-32}
Consider a further experiment by \citet{Csibra:2003jv}, also with
12-month-olds. This is just like the first ball-jumping experiment
except that here infants see the action but not the circumstances in
which it occurs. Do they expect there to be an object in the way behind
that barrier?

\subsection{slide-34}
Consider a related study by Woodward and colleagues.

(It's good that there is converging evidence from different labs, using quite different stimuli.)

\subsection{slide-35}
'Six-month-olds and 9-month-olds showed a stronger novelty response (i.e., looked longer) on new-goal trials than on new-path trials (Woodward 1998). That is, like toddlers, young infants selectively attended to and remembered the features of the event that were relevant to the actor’s goal.'
\citep[p.\ 153]{woodward:2001_making}

\subsection{slide-36}
Using a manipulation we’ll discuss later (‘sticky mittens’),
\citet{sommerville:2005_action} used this paradigm to show that even
three-month-olds can form expectations based on the goal of an action
(for another study with three-month-olds, see also \citealp{luo:2011_threemonthold}).

\subsection{slide-37}
\citet{daum:2012_actions} adapted Woodward et al’s paradigm so that they
could simultaneously measure looking time and anticipatory looking.

You can see that there’s a round occluder in the middle of the display.  The
agent, a red fish, moves behind this on it’s way to visit one of two objects.

In Experiment 1,
First a small fish moves towards one object six times (familiarization).
Then the locations of objects were swaped and infants’ responses were measured
during six test trials. In these trials, the agent (red fish) moved on the
same path towards a new goal three times; and it did the converse three
times.

Just as you'd expect given Woodward et al 2001, 9-month old infants looked
longer when the fish (agent) moved towards a new goal (same path) than when
it moved towards the same goal on a new path.

BUT where were the infants looking in anticpation of the agent’s
reappearance? It turns out that, on the whole, they were looking as if the
agent were moving on the same path to a new goal.
So there’s a dissocaition between anticipatory looking and violation-of-expectations
measures for 9 month olds.

‘The results of the analysis of the infants’ eye movements contrast the
looking time results and showed at the age of 12 months (and less reliably
at the age of 9 months) infants predicted the reappearance of the agent
based on the location of the goal during an observed action and that it
was not until the age of 3, that this dissociation disappeared and that
children predicted the reappearance of the agent after occlusion based on
goal identity’ \citep[p.~9]{daum:2012_actions}/

‘Early in life, action expectations measured online seem to be organized
around goal locations whereas action expectations measured post-hoc around
goal identities. With increasing age, children then generally organize
their action expectations primarily around goal identities’
\citep[p.~10]{daum:2012_actions}

\subsection{slide-38}
In a follow up experiment, \citet{daum:2012_actions} considered anticipatory looking
with these stimuli for a range of agents including adulthoot.

As you can see, adults’ anticipatory looking is generally to the same-goal but
different-path location, which is also what 3-year-olds do.

Why is there a dissocaition between anticipatory looking and violation-of-expectations
measures for 9 month olds?

One possibility noted by \citet{daum:2012_actions} is that anticipatory looking requires
rapid computation of the goal and its consequences for movement.  It may be that infants
simply cannot compute the goal in the few hundren milliseconds available for anticipatory
looking.

This doesn’t sound very interesting, but note that 9-month-old infants seem to be
making anticipatory looks.  This suggests that they may be guided by contingency
information in making their anticipatory looks.
This is potentially important as a clue that action anticipation is driven by a mixture
of goal ascription and statistical information.

Also these findings need cautious interpretation.
Cf \citep[p.~7]{ambrosini:2013_looking}: ‘some visual anticipation studies
show that 12-month-olds and even 10-month-olds, but not 6-month-olds, can
predictively gaze to the goal position when observing displacement actions
[9,12], while some others demonstrate that even 6-month-olds show
anticipatory fixations to the goal of observed actions [34,35].’
See also` \citet{cannon:2012_infants}.

\subsection{slide-39}
Chopsticks vs spoons, Sweeden vs China, 8-month-olds.

Would they look at the bowl when the chopsticks or spoon was held?
(No!)

Would they look at the mouth when the chopstick or spoon was loaded with
a cracker? Let’s have a look ...

\subsection{slide-42}
\citep[p.~743]{green:2016_culture}: ‘Consistent with prior findings, the
current study also demon- strates that infants need to have the ability to
per- form similar actions themselves. In other words, culture appears to
work together with the obser- vers’ own motor plans in order to facilitate
goal prediction. Infants predict the goal of actions that are similar to
their own motoric capability (like put- ting things in their mouth), but
only for actions per- formed with a tool frequently occurring in the
cultural context in which they live. The influence of one’s own motor
capability is expressed primarily as a lack of prediction during the
picking up food part of the action and the presence of prediction during
self-oriented actions (i.e., bringing objects to the mouth and eating).’

Figure caption: ‘Mean gaze latency (negative numbers = prediction) to arrive at
the goal when observing eating actions directed toward the mouth using a
spoon (squares) and chopsticks (circles) in China and Sweden. Error bars
represent standard errors.’

\subsection{unit\_702}


\section{The Teleological Stance}

\subsection{slide-45}
Let me first specify the problem to be solved.

\subsection{slide-47}
As this illustrates,
some actions involving are purposive in the sense that

\subsection{slide-48}
among all their actual and possible consequences,

\subsection{slide-49}
there are outcomes to which they are directed

\subsection{slide-50}
In such cases we can say that the actions are clearly purposive.

\subsection{slide-51}
It is important to see that the third item---representing the directedness---is necessary.

This is quite simple but very important, so let me slowly explain why goal ascription requires representing the directedness of an action to an outcome.

\subsection{slide-52}
Imagine two people, Ayesha and Beatrice, who each intend to break an egg.
Acting on her intention, Ayesha breaks her egg.
But Beatrice accidentally drops her egg while carrying it to the kitchen.
So Ayesha and Beatrice perform visually similar actions which result in the same type of outcome,
the breaking of an egg; but Beatrice's action is not directed to the outcome of her action whereas
Ayesha's is.

\subsection{slide-53}
Goal ascription requires the ability to distinguish between Ayesha's action and Beatrice's action.
This requires representing not only actions and outcomes but also the directedness of actions to outcomes.

This is why I say that goal ascription requires representing the directedness of an action to an outcome, and not just representing the action and the outcome.

\subsection{slide-54}
Next consider requirements on a solution to the problem.

\subsection{slide-56}
How about taking $R$ to be causation?
That is, how about defining $R(a,G)$ as $a$ causes $G$?
This proposal does not meet the first criterion, (1), above.
We can see this by mentioning two problems.

[*Might skip over-generate and discuss that as a problem for Rationality/Efficiency]
First problem: actions typically have side-effects which are not goals.
For example,
%---not a good example because can't be avoided by any account
%--- (would require attribution of desire)
%For example, walking to the corner results in me warming up, in me expending energy, and in me being at the corner.
%Sometimes I walk to the corner for exercise,
%so that being at the corner is an unwanted side-effect (I then have to walk back).
%And sometimes I walk to the corner to be at the corner (so that expending energy is an unwanted side-effect, I'd rather have been chauffeured there).
suppose that I walk over here with the goal of being next to you.
This action has lots of side-effects:
\begin{itemize}
\item 	I will be at this location.
\item	I will expend some energy.
\item	I will be further away from the front
\end{itemize}
These are all causal consequence of my action.
But they are not goals to which my action is directed.
So this version of $R$ will massively over-generate goals.

Second problem: actions can fail.  [...]
So this version of $R$ will under-generate goals.

\subsection{slide-60}
R(a,G) =df a is caused by an intention to G?

\subsection{slide-61}
\citet{Premack:1990jl} writes:

‘in perceiving one object as having the intention of affecting another, the infant attributes to the object [...] intentions’
\citep[p.\ 14]{Premack:1990jl}
\subsection{slide-62}
\citep[p.~53]{woodward:2009_infants}

Woodward et al qualify this view elsewhere

‘to the extent that young infants are limited [...], their
understanding of intentions would be quite different from the mature
concept of intentions’
\citep[p.\ 168]{woodward:2001_making}.

\subsection{slide-63}
By contrast, Geregely et al reject this possibility ...

‘by taking the intentional stance the infant can come to represent the agent’s action as intentional without actually attributing a mental representation of the future goal state’
\citep[p.\ 188]{Gergely:1995sq}

Btw, it isn't clear that this proposal can work (as introduced by Dennett, the intentional stance
involves ascribing mental states), as these authors probably realised later, but the point
about not representing mental states is good.

\subsection{slide-66}
The requirement that R(a,G) be detectable without any knowledge of mental
states is not met.  Why impose this requirement?
Imagine you are a three-month-old infant. Let’s assume that you know what
intentions are and can represent them. Still, on what basis can you
determine the intentions behind another’s actions? You can’t communicate
linguistically with them. In fact it seems that the only access you have
to another’s intentions is via the actions they perform. Now let’s
suppose that to identify the goals of the actions you have to identify
their intentions. Then you have to identify intentions on the basis of
mere joint displacements and bodily configurations. This is quite
challenging. How much easier it would be if you had a way of identifying
the goals of the actions independently of ascribing intentions. Then you
would be able to first identify the goals of actions and then use
information about goals to ascribe intentions to the agent.

\subsection{slide-68}
Why not define $R$ in terms of teleological function?
This would enable us to meet the first condition but not the second.
How could we tell whether an action happens because it brought about a particular outcome in the past?
This might be done with insects.
But it can's so easily be done with primates, who have a much broader repertoire of actions.

\subsection{slide-69}
What do we mean by teleological function?

\subsection{slide-70}
Here is an example:
%
\begin{quote}

Atta ants cut leaves in order to fertilize their fungus crops (not to thatch the entrances to their homes) \citep{Schultz:1999ps}

\end{quote}

\subsection{slide-71}
What does it mean to say that the ants’ grass cutting has this goal rather than some other? According to Wright:
\begin{quote}

‘S does B for the sake of G iff: (i) B tends to bring about G; (ii) B occurs because (i.e. is brought about by the fact that) it tends to bring about G.’ (Wright 1976: 39)

\end{quote}

\subsection{slide-72}
For instance:
%
\begin{quote}

The Atta ant cuts leaves in order to fertilize iff: (i) cutting leaves tends to bring about fertilizing; (ii) cutting leaves occurs because it tends to bring about fertilizing.

\end{quote}

\subsection{slide-78}
The Teleological Stance:
‘an action can be explained by a goal state if, and only if, it is seen as  the  most justifiable action towards that goal state that is available within the constraints of reality’
\citep[p.~255]{Csibra:1998cx}

\subsection{slide-82}
It will work if we can match observer and agent: both must be ‘equally optimal’.
But how can we ensure this?

\subsection{slide-83}
How good is the agent at optimising the rationality, or the efficiency, of her actions?
And how good is the observer at identifying the optimality of actions in relation to outcomes?
\textbf{
If there are too many discrepancies between
		how well the agent can optimise her actions
	and
		how well the observer can detect optimality,
then these principles will fail to be sufficiently reliable}.

\subsection{slide-84}
The Teleological Stance is a proposed solution.

\subsection{unit\_704}


\section{Marr’s Threefold Distinction}

\subsection{slide-86}
To answer this question, we need to get beyond the Teleological Stance
and consider the representations and algorithms that underpin it.
Let me explain.

\subsection{slide-87}
One possibility is to appeal to David Marr’s famous three-fold distinction
bweteen levels of description of a system: the computational theory, the
representations and algorithm, and the hardware implementation.

This is easy to understand in simple cases.
To illustrate, consider a GPS locator.
It receives information from four satellites and tells you where on Earth the device is.

There are three ways in which we can characterise this device.

\subsection{slide-88}
First, we can explain how in theory it is possible to infer the
device’s location from it receives from satellites.
This involves a bit of maths: given time signals from four different satellites,
you can work out what time it is and how far you are away from each
of the satellites.
Then, if you know where the satellites are and what shape the Earth is,
you can work out where on Earth you are.

\subsection{slide-89}
The computational description tells us what the GPS locator does and
what it is for.
It also establishes the theoretical possibility of a GPS locator.

But merely having the computational description does not enable you to build
a GPS locator, nor to understand how a particular GPS locator works.
For that you also need to identify representations and alogrithms ...

\subsection{slide-90}
At the level of representations and algorthms we specify
how the GPS receiver represents the information it receives from the satellites
(for example, it might in principle be a number, a vector or a time).
We also specify the algorithm the device uses to compute the time and its location.
The algorithm will be different from the computational theory: it is a procedure
for discovering time and location.
The algorithm may involve all kinds of shortcuts and approximations.
And, unlike the computational theory, constraints on time, memory and other
limited resources will be evident.

\subsection{slide-91}
So an account of the representations and algorithms tells us ...

-- How are the inputs and outputs represented, and how is the transformation accomplished?

\subsection{slide-92}
The final thing we need to understand the GPS locator is a description of the
hardware in which the algorithm is implemented.  It’s only here that
we discover whether the device is narrowly mechanical device, using cogs, say,
or an electronic device, or some new kind of biological entity.

\subsection{slide-93}
The hardware implementation tells us how the representations and algorithms are represented physically.

\subsection{slide-94}
How is this relevant to the teleological stance?
It provides a computational description of goal ascription.
(Whereas the Motor Theory provides an account of the representations and algorithms )

\subsection{slide-95}
The teleological stance  provides a computational description of goal ascription.

\subsection{slide-96}
For deeper insight into goal ascription, we need an account of representations and
algorithms.

Compare our research on infants’ abilities concerning physical objects.
Spelke’s principles of object perception provide a computational description
of infants’ abilities to segment, etc.
But to understand the nature of these abilities and their relation to knowledge,
and to explain the otherwise puzzling patterns of development, we needed to
identify representations and alogrithms.
(We did this by appeal to the operations of a system of object indexes.)

\subsection{slide-97}
The Teleological Stance:
‘an action can be explained by a goal state if, and only if, it is seen as  the  most justifiable action towards that goal state that is available within the constraints of reality’
\citep[p.~255]{Csibra:1998cx}

The teleological stance is a computational description.
What’s the algorithm?

\subsection{slide-98}
‘when taking the teleological stance one-year-olds apply the same
inferential principle of rational action that drives everyday mentalistic
reasoning about intentional actions in adults’
(\citealp{Gergely:2003gb}; compare \citealp{Csibra:2003jv}, \citealp[p.~259]{Csibra:1998cx} )

\subsection{slide-99}
Csibra and Gergely seem aware that this would make the Teleological Stance quite
complex to apply ...

`Such calculations require detailed knowledge of biomechanical factors that determine the motion capabilities and energy expenditure of agents.  However, in the absence of such knowledge, one can appeal to heuristics  that approximate the results of these calculations on the basis of knowledge in other domains that is certainly available to young infants.

\subsection{slide-102}
What heuristics.

\subsection{slide-105}
Csibra and Gergely’s newer proposal seems to assume the inferential integration
of core systems.
But principles governing object indexes are not typically available for
general reasoning.

\subsection{unit\_703}


\section{Action Observation by Adults}

\subsection{slide-110}
\citet{Flanagan:2003lm} showed that
‘patterns of eye–hand coordination are similar when performing and observing a block stacking task’.

\subsection{slide-111}
‘We recorded proactive eye movements while participants observed an actor
grasping small or large objects. The participants' right hand either freely
rested on the table or held with a suitable grip a large or a small object,
respectively. Proactivity of gaze behaviour significantly decreased when
participants observed the actor reaching her target with a grip that was
incompatible with respect to that used by them to hold the object in their
own hand.’

Follow ups: tie hands; TMS (impair)

Planning-like processes in action observation have also been demonstrated
by measuring observers' predictive gaze. If you were to observe just the
early phases of a grasping movement, your eyes might jump to its likely
target, ignoring nearby objects \citep{ambrosini:2011_grasping}. These
proactive eye movements resemble those you would typically make if you were
acting yourself \citep{Flanagan:2003lm}. Importantly, the occurrence of
such proactive eye movements in action observation depends on your
representing the outcome of an action motorically; even temporary
interference in the observer's motor abilities will interfere with the eye
movements \citep{Costantini:2012fk}.

In human adults, motor representations and processes enable anticipatory looking
that is driven by goal ascription \citep[e.g.][]{Costantini:2012fk,ambrosini:2012_tie}.

\subsection{slide-112}
Motor representations ocurring in action observation sometimes facilitate the identification of goals.

What are those motor representations doing here?

\subsection{slide-113}
The Motor Theory of Goal Ascription:

The idea is that we could solve the problem--the problem of matching optimisation in planning
actions
with optimisation in predicting them--by supposing that a single set of mechanisms is used
twice, once in planning action and once again in observing them.

What does this require?

\subsection{slide-114}
-- in action observation, possible outcomes of observed actions are represented

\subsection{slide-115}
-- these representations trigger planning as if performing actions directed to the outcomes

\subsection{slide-116}
-- such planning generates predictions

predictions about joint displacements and their sensory conseuqences

\subsection{slide-117}
-- a triggering representation is weakened if its predictions fail

The proposal is not specific to the idea of motor representations and processes,
although there is good evidence for it (which I won't cover here because we're in Milan!)

\subsection{slide-118}
There is evidence that a motor representation of an outcome can cause a determination of which movements are likely to be performed to achieve that outcome \citep[see, for instance,][]{kilner:2004_motor, urgesi:2010_simulating}. Further, the processes involved in determining how observed actions are likely to unfold given their outcomes are closely related, or identical, to processes involved in performing actions.
This is known in part thanks to studies of how observing actions can facilitate performing actions congruent with those observed, and can interfere with performing incongruent actions \citep{
	brass:2000_compatibility,
	craighero:2002_hand,
	kilner:2003_interference,
	costantini:2012_does}.
Planning-like processes in action observation have also been demonstrated by measuring observers' predictive gaze.  If you were to observe just the early phases of a grasping movement, your eyes might jump to its likely target, ignoring nearby objects \citep{ambrosini:2011_grasping}. These proactive eye movements resemble those you would typically make if you were acting yourself \citep{Flanagan:2003lm}.
Importantly, the occurrence of such proactive eye movements in action observation depends on your
representing the outcome of an action motorically; even temporary interference in the observer's
motor abilities will interfere with the eye movements \citep{Costantini:2012fk}.
These proactive eye movements also depend on planning-like processes; requiring the observer to
perform actions incongruent with those she is observing can eliminate proactive eye movements
\citep{Costantini:2012uq}. This, then, is further evidence for planning-like motor processes in
action observation.

So observers represent outcomes motorically and these representations trigger planning-like processes
which generate expectations about how the observed actions will unfold and their sensory consequences.
Now the mere occurrence of these processes is not sufficient to explain why, in action observation,
an outcome represented motorically is likely to be an outcome to which the observed action is
directed.

To take a tiny step further, we conjecture that, in action observation, \textbf{motor representations of
outcomes are weakened to the extent that the expectations they generate are unmet}
\citep[compare][]{Fogassi:2005nf}.
A motor representation of an outcome to which an observed action is not directed is likely to
generate incorrect expectations about how this action will unfold, and failures of these
expectations to be met will weaken the representation.
This is what ensures that there is a correspondence between outcomes represented motorically in
observing actions and the goals of those actions.

\subsection{slide-119}
Now we’ve solved this: the Motor Theory of Goal Ascription is the solution.

See \citet{sinigaglia:2015_goal_ascription} for an outline of the Motor Theory of Goal Ascription.

\subsection{slide-120}
Recall David Marr’s famous three-fold distinction
between levels of description of a system: the computational theory, the
representations and algorithm, and the hardware implementation.

-- How are the inputs and outputs represented, and how is the transformation accomplished?

\subsection{slide-121}
The teleological stance  provides a computational description of goal ascription.

\subsection{slide-122}
The motor theory of goal ascription provides an account of the
representations and algorithms, one that competes with Csibra and Gergely’s
account based on general reasoning.

\subsection{slide-123}
Proactive gaze indicates fast goal ascription.


          The Teleological Stance provides a computational description of the goal ascription
underpinning adults’ proactive gaze

          Proactive gaze depends on motor processes and representations: the
Motor Theory provides an account of the representations
and algorithms.


\subsection{slide-129}
Is there any evidence? ...

\subsection{unit\_709}


\section{Performing vs Understanding Actions in Infancy}

\subsection{slide-132}
In adults, we got at a parallel question by tieing their hands.
There’s a similar manipulation in infants involving sticky mittens ...

\subsection{slide-133}
Needham et al, 2002 showed that putting ‘sticky mittens’ on 3-month-old infants
(for 10-14 play sessions of 10 minutes each) resulted in their spending
more time visually and manually exporing novel objects.

\subsection{slide-134}
In this study, I think infants wore the mittens for just 200 seconds
(so the play sessions were much shorter than in Needhman et al, 2002).

\subsection{slide-135}
The observation was based on this study, which we saw earlier

\subsection{slide-137}
The results show that infants who played wearing the mittens first
were more attentive to the goal.

From at least three months of age, some of infants’ abilities to identify
the goals of actions they observe are linked to their abilities to perform
actions \citep{woodward:2009_infants}.

But one potential objection to this study concerns observation vs performance.
The infants who played wearing sticky mittens first had spent longer observing
actions by the time it came to the violation of expectations trial.
Could it be observation of action (including one’s own) rather than performance
that matters?

\subsection{slide-138}
To address this issue, \citet{sommerville:2008_experience} did a study in
which one group had observation while the other group had performance.
The participants were 10-month-old infants this time.

The materials were a bit different: so that training vs observation could
be as similar as possible with respect to the causal structure exposed,
there was a hook to get an object.

\subsection{slide-139}
The results show that infants with the training paid attention to the
distal goal (choice of toy) whereas those without paid attention to the
choice of cane.

\subsection{slide-140}
Further support for a link between action performance and goal ascription
comes from a developmental study by Ambrosini et al which studied whether proactive
gaze in infants is influenced by pre-shaping of the hand, and, in particular,
whether it is influenced by precision grips.

\subsection{slide-142}
By using no shaping (a fist), Ambrosini et al could treat sensitivity
to whole-hand grasp and precision grip separately.

\subsection{slide-143}
‘infants’ ability to perform specific grasping actions with fewer fingers directly predicted the degree with which they took advantage of the availability of corresponding pre-shape motor information in shifting their gaze towards the goal of others’ actions’ \citep[p.~6]{ambrosini:2013_looking}.

\subsection{slide-144}
In infants and adults,
abilities to perform actions enable identifying goals when observing them.

Why is this true?

\subsection{slide-145}
The Motor Theory of Goal Ascription:

The idea is that we could solve the problem--the problem of matching optimisation in planning
actions
with optimisation in predicting them--by supposing that a single set of mechanisms is used
twice, once in planning action and once again in observing them.

What does this require?

\subsection{slide-146}
-- in action observation, possible outcomes of observed actions are represented

\subsection{slide-147}
-- these representations trigger planning as if performing actions directed to the outcomes

\subsection{slide-148}
-- such planning generates predictions

predictions about joint displacements and their sensory conseuqences

\subsection{slide-149}
-- a triggering representation is weakened if its predictions fail

The proposal is not specific to the idea of motor representations and processes,
although there is good evidence for it (which I won't cover here because we're in Milan!)

\subsection{slide-150}

\subsection{slide-151}
In infants and adults,
abilities to perform actions enable identifying goals when observing them.

Why is this true?

In suggesting that it’s because of the Motor Theory,
I’m going beyond anything Sommerville et al would endorse
although moving in a direction they cautiously indicate.

I’m also contradicting how most people think of the relation
between the Teleological Stance and the Motor Theory.
Most theorists think of these as alternatives.
E.g. \citep{Csibra:2007hm} contrast what can be explained with
Teleological Stance and Motor Theory (they claim Teleological
Stance is required for novel situations in which Motor Theory
should fail; this is probably in part right, although success in the face of
novelty could be driven by associations.)
See also \citet[p.~204]{gredeback:2010_infantsa}:
‘We suggest that anticipation of action goals is mediated by a direct
matching process (Flanagan \& Johansson, 2003) whereas retrospective
evaluations of rationality are dependent on more abstract
well-formedness criteria as described by the teleological stance
(Gergely \& Csibra, 2003).’
(NB: their argument is not good: they claim that because infants who have
little or no experience of feeding themselves can show suprise (pupil dilation)
in response to nonrational self-feeding actions, they are using the Teleological
Stance here.  But it might be motoric; or it might be association (cf the ‘telephone
to ear’ study).)

\subsection{slide-153}
Proactive gaze indicates fast goal ascription.



The Teleological Stance provides a computational description of the goal ascription
underpinning adults’ proactive gaze



Proactive gaze depends on motor processes and representations: the
Motor Theory provides an account of the representations
and algorithms.


\subsection{slide-154}
The question was, Is there any evidence? ... Yes, we found evidence.

\subsection{slide-155}
There’s probably more than the Motor Theory to goal ascription
(in infants, and adults).

\subsection{slide-156}
\citep{melzer:2012_production}: ‘The infant was given a cube (occupation
object) in either his/her left or right hand. Subsequently,a second toy
(target object) was held either (a) in front of the empty hand to elicit an
ipsilateral reaction (ipsilateral presentation) or (b) in front of the
occupied hand to elicit a contralateral reaction (contralateral
presentation).’

\subsection{slide-157}
Infants become good at contralateral grasping between six and twelve months.

\subsection{slide-158}
And here is an anticipatory looking task ...

\subsection{slide-159}
How did the infants do?  The data in figure indicate that 12-month-olds
showed quite good, adult like anticipation of contralateral actions,
whereas 6-month-olds were arriving at the target object behind the hand.

This suggests a link between performance and goal ascription, once more.
(p. 577: ‘At 12 months, most infants were able to anticipate the goal of contralateral movements, whereas at 6 months, infants showed mainly reactive eye movements.’)

Further, p. 577: ‘Production and perception of contralateral reaching
movements were correlated at 12 months of age. The more sophisticated
12-month-olds’ reaching production was, the better they anticipated other
people’s contralateral movements.’

But the result that that \citet{melzer:2012_production} focus on is this:
p. 577: ‘perception and production were not yet correlated at 6 months. The
lack of a significant correlation was neither due to a larger variance in
the younger infant group nor to the influence of a bias in our sample.
Accordingly, our findings suggest that a link between production and
perception of contralateral arm movements, and possibly therefore a common
representation, develops in the second half of the first year of life.’

\subsection{slide-162}
Emphasize: (1) that they understand action is important because we can use it
to build an account of how you get from core knowledge to knowledge knowledge

\subsection{slide-163}
Emphasize: (2) that their understanding of action does not involve
knowledge is important because it allows us to invoke it without
assuming capacities for any knowledge at all on the part of the infant.
(And because it leaves us with a question about how infants get from
core knowledge of action to knowledge of action.)


    

%--- end paste
%---------------






\bibliography{$HOME/endnote/phd_biblio}



\end{document}
