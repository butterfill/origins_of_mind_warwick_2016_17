 %!TEX TS-program = xelatex
%!TEX encoding = UTF-8 Unicode

%\def \papersize {a5paper}
\def \papersize {a4paper}
%\def \papersize {letterpaper}

%\documentclass[14pt,\papersize]{extarticle}
\documentclass[12pt,\papersize]{extarticle}
% extarticle is like article but can handle 8pt, 9pt, 10pt, 11pt, 12pt, 14pt, 17pt, and 20pt text

\def \ititle {Origins of Mind: Lecture Notes}
\def \isubtitle {Lecture 01}
%comment some of the following out depending on whether anonymous
\def \iauthor {Stephen A.\ Butterfill}
\def \iemail{s.butterfill@warwick.ac.uk% \& corrado.sinigaglia@unimi.it
}
%\def \iauthor {}
%\def \iemail{}
%\date{}

%\input{$HOME/Documents/submissions/preamble_steve_paper4}
\input{$HOME/Documents/submissions/preamble_steve_lecture_notes}

%no indent, space between paragraphs
\usepackage{parskip}

%comment these out if not anonymous:
%\author{}
%\date{}

%for e reader version: small margins
% (remove all for paper!)
%\geometry{headsep=2em} %keep running header away from text
%\geometry{footskip=1.5cm} %keep page numbers away from text
%\geometry{top=1cm} %increase to 3.5 if use header
%\geometry{bottom=2cm} %increase to 3.5 if use header
%\geometry{left=1cm} %increase to 3.5 if use header
%\geometry{right=1cm} %increase to 3.5 if use header

% disables chapter, section and subsection numbering
\setcounter{secnumdepth}{-1}

%avoid overhang
\tolerance=5000

%\setromanfont[Mapping=tex-text]{Sabon LT Std}


%for putting citations into main text (for reading):
% use bibentry command
% nb this doesn’t work with mynewapa style; use apalike for \bibliographystyle
% nb2: use \nobibliography to introduce the readings
\usepackage{bibentry}

%screws up word count for some reason:
%\bibliographystyle{$HOME/Documents/submissions/mynewapa}
\bibliographystyle{apalike}


\begin{document}



\setlength\footnotesep{1em}






%---------------
%--- start paste
\title {Origins of Mind \\ Lecture 01}



\maketitle

\subsection{title-slide}


\section{The Question}

--------
\subsection{slide-3}
This course is based on a simple question. The question is,

How do humans first come to know  about---and to knowingly manipulate---objects,
causes, words, numbers, colours, actions and minds?

\subsection{slide-4}
At the outset we know nothing, or not very much.  (Like little Wy here.)
Sometime later we do know some things.
How does the transition occur?

We are going to approach this question by examining the evidence from developmental science,
exploring how it bears on philosophical positions like nativism and empiricsm,
and identifying philosophical problems created by the evidence.

Far from being a new question, it belongs to a family of questions about the origins of mind
that philosophers have been asking since Plato or before.

\subsection{slide-5}
Here is John Locke asking a version of my question ...

‘... ’tis past doubt, that Men have in their Minds several Ideas, such as are those expressed by the words, Whiteness, Hardness, ... and others: It is in the first place to be enquired, How he comes by them?’
\citep[p.\ 104]{Locke:1975qo}
\citep[p.\ 104]{Locke:1975qo}

‘How does it come about that the development of organic behavior into controlled inquiry brings about the differentiation and cooperation of observational and conceptual operations?’
\citep[p.\ 12]{Dewey:1938yp}
\citep[p.\ 12]{Dewey:1938yp}

‘the fundamental explicandum, is the organism and its propositional attitudes ... Cognitive psychologists accept ... the  ... necessity of explaining how organisms come to have the attitudes to propositions that they do.’
\citep[p.\ 198]{Fodor:1975pb}
\citep[p.\ 198]{Fodor:1975pb}

\subsection{slide-6}
And here is a nativist, Jerry Fodor, asking much the same question.

\subsection{slide-7}
Note that whereas Locke puts the question in terms of Ideas, Fodor, writing much later,
asks about propositional attitudes, that is, about states such as knowlegde and belief.
This difference need not concern us; we now have a much better understanding of the metaphysics
of mental states than Locke did.

\subsection{slide-8}
Finally note that much the same question has been asked by philosophers coming at things from a
completely different---pragmatist, in this case---point of view.

\subsection{slide-9}
Because he focusses on behaviour and mental operations rather than Ideas or states of mind,
he puts the question in terms of a transition from unthinking ('organic') behaviour
to capacities for observational and conceptual operations.

So my question about how we first come to know things in various domains is an ok question
for a philosopher to ask because it belongs to a family of questions that philosophers who
have been dead for a long time were asking.

\subsection{slide-10}
One last thing.
Fodor mentions cognitive psychologists rather than philosophers.  We will need to face up to
the question of why philosophers are asking this question about the origins of knowledge, why
is isn't just a scientific question.  I'll come to this shortly.

\subsection{unit\_021}


\section{From Myths to Mechanisms}

\subsection{slide-12}
How do humans first come to know  about---and to knowingly manipulate---objects, causes, words,
numbers, colours, actions and minds?
In a beautiful myth, Plato (who also asked this question) suggests that the answer is
recollection.
Before we are born, in another world, we become acquainted with the truth.
Then, in falling to earth, we forget everything.
But as we grow we are sometimes able to recall part of what we once knew.
So it is by recollection that humans come to know about objects, causes, numbers and everything
else.

\subsection{slide-13}
Leibniz explicitly endorses a version of Plato's view.

‘the soul inherently contains the sources of various notions and doctrines which external objects merely rouse up on suitable occasions’
\citep[p.\ 48]{Leibniz:1996bl}

The view is subtler than it seems: we'll return to the subtelties later.
[*Actually that isn't in these lectures, but it should be.]

\subsection{slide-14}
Locke, as you probably know, was an empiricist.  Here's his manifesto.

‘Men, barely by the Use of their natural Faculties, may attain to all the Knowledge
they have, without the help of any innate Impressions’
\citep[p.\ 48]{Locke:1975qo}

\subsection{slide-15}
Spelke is blunt.

‘Developmental science [...] has shown that both these views are false’
\citep[p.\ 89]{Spelke:2007hb}.

Spelke doesn't have exactly Locke vs Leibniz in mind here, but rather modern descendants
of their views.

[The quote continues ‘humans are endowed neither with a single, general-purpose learning system
nor with myriad special-purpose systems and predispositions. Instead, we believe that humans
are endowed with a small number of separable systems of core knowledge. New, flexible skills
and belief systems build on these core foundations.’]

Spelke's claim may be too bold.  As we will see, there is surprisingly little evidence about
the conflict between empiricists and nativists.
A more cautious claim would be this.  when we look at particular cases in detail---for
instance,
when we look at how humans come to know about colours---we will discover complexities that seem
to be incompatible with any one of the stories.

\subsection{slide-16}
To make progress we need to forget about the myths; we need to
identify various mechanisms and attempt to model their interactions.

The claim that we should shift from thinking about myths to mechanisms makes pressing
the worry that there is no task here for philosophers and that its really just a job for
scientists.

\subsection{unit\_031}


\section{Davidson’s Challenge}

\subsection{slide-18}
Why suppose there is any role for philosophers rather than scientists?
Part of the answer is provided by Donald Davidson.

The question is how humans come to know about objects, words, thoughts and other things.
In pursuing this question we have to consider minds where the knowledge is neither clearly
present nor obviously absent.
This is challenging because both commonsense and theoretical tools for describing minds are
generally designed for characterising fully developed adults.

\subsection{slide-19}
I love this: Davidson says we will fail.  So encouraging.
But why will we fail?

\subsection{slide-20}
Is he suggesting the issue is merely terminological?  Not quite ...

\subsection{slide-22}
So this is the challenge.  We are describing something which is neither mindless
nor involving full-blown thought and action.  And, as Davidson observes, we lack
a way of describing what is in between.

The question about the origins of mind is in part a philosophical question because
answering it will require new conceptual tools.

This challenge is one that we will face again and again.

\subsection{slide-23}
What does Davidson have in mind?  Let me introduce
what I’ll call the ‘Uncomplicated Account of Minds and Actions’.

\subsection{Uncomplicated Account of Minds and Actions}
For any given proposition [There’s a spider behind the book] and any given human [Wy]
...
\begin{enumerate}
\item Either Wy believes that there’s a spider behind the book, or she does not.
\item Either Wy can act for the reason that there is, or seems to be, a spider behind the book, or else she cannot.
\item The first alternatives of (1) and (2) are either both true or both false.
\end{enumerate}

\subsection{slide-26}
The Uncomplicated Account could be elaborated to take into account
the fact that how people act depends on attitudes other than belief.
For example, much may depend on whether Wy likes, or fears, spiders.

Philosophers have done much of this elaboration, building on the Uncomplicated
Account by identifying features of belief adding attitudes like desire
and intention as well as fear and pride.

\subsection{slide-27}
But there is a fundamental problem with the Uncomplicated Account,
one that probably can’t be fixed by elaborating it with further attitudes.
The problem is that, as we will see in a moment, there are cases in which
(2) turns out to be wrong and the connection between (1) and (2) breaks down.

If we want to understand what is going on in the head of an infant who is
in the process of developing capacities for knowledge, we will have to
fundamentally revise the Uncomplicated Account.

\subsection{slide-23}


\section{Unperceived Objects: An Illustration of Davidon's Challenge}

\subsection{slide-28}
If you've seen the outline of lectures, you'll know that my idea is to organise the lectures
by domains of knowledge.
As we will see, how we first come to know things about colours, say, isn't quite the same
as how we first come to know things about minds.
But there is one very general point we can make: in all these domains, we will face
Davidson's challenge, the challenge of explaining what is inbetween mindless behaviour
and thought.

\subsection{slide-29}
Let me preview how Davidson's challenge arises in the case of objects.

\subsection{slide-30}
When do humans first come to know facts about the locations of objects they are not
perceiving?

\subsection{slide-31}
A famous study by Renee Baillargeon and her collaborators provides evidence that humans can
represent unperceived objects from around four months of age or earlier.
This is called the 'drawbridge study'

What you are about to see are the test events from Experiment 1 of Baillargeon et al's
1987 study.  You're looking at them from the side whereas the subjects, four-month olds,
were looking at them from the front (which is to your right).

In showing you these test events,
I need to explain the method used in this experiment, \emph{habituation};
this is a method we will encounter repeatedly so it's good to understand how it is
supposed to work.

%glossary: habituation

What you see here is a barrier rotating through 180 degrees.
Infants were habituated to this; that is, they were shown it repeatedly until it no longer
held their interest.
The first time they're shown this, they might spend 60 seconds looking at it, which is a long
time for an infant; but after, say, five demonstrations, they'd only be looking at it for
around 10 seconds.  That is, they are habituated to this display.

\subsection{slide-32}
Now there is a very small change to the display.
The display is just as before, except for before the drawbridge moves an object is placed
behind it.
There are then two different things that could happen.  One is that the drawbridge moves
exactly as before, rotating a full 180 degrees.  This is called the 'impossible event'.
The other is that the drawbridge now rotates for 120 degrees, which is the 'possible event'.
In no case is the object visible after the drawbridge has started moving.
We want to know which events infants find more novel.
If they are unable to know facts about  the locations of unperceived objects, then they should
find the
'possible event' more novel than the 'impossible event' because it is more different
from the event they have been habituated to.
On the other hand, if infants are able know facts about the locations of unperceived objects,
they should find the impossible event more novel than the possible event because, well,
it's impossible.

To find out what infants find more interesting, they are divided into two groups.
One group sees the impossible event, the other the possible event.
The experimenters measure how long the infants look at these events, which is the measure of
their dishabituation.  The background assumptions are that looking longer indicates more
interest, and that interest is driven by novelty.

\subsection{slide-33}
In the control condition,
'The habituation event was exactly the same as the impossible event, except that the yellow
box was absent.' (Baillargeon et al 1985, 200)

\subsection{slide-34}
These are the results from Experiment 1 of Baillargeon et al's 1987 study.

This experiment provides evidence that infants know that the object is behind the barrier
even when they can't see it, for their having such knowledge would explain why they appear
surprised by the impossible event.

\subsection{slide-35}
Here you can see, reassuringly, that the effect is not present in the control condition
where the box is absent.

Some have been critical of the methods used in this experiment.
But not everything hangs on this experiment.
Fortunately there are at least a hundred further experiments which provide evidence pointing in the same direction.
Later we'll look at this is more detail.

\subsection{slide-36}
When do humans first come to know facts about the locations of objects they are not perceiving?

This result has been widely replicated, and it coheres with a large body of research we shall
explore later.

\subsection{slide-37}
By using more sensitive methods, \citet{Aguiar:1999jq} even demonstrated competence in a group
of 2.5 month old infants.

So far so good, but there is a problem ...
What happens if instead of measuring how infants look, we measure how they reach?

\subsection{slide-38}
\citet{Shinskey:2001fk} did just this.
Here you can see their appratus, which is quite similar to what \citet{baillargeon:1987_object}
used.
They had a screen that infants could pull forwards to get to an object that was sometimes
hidden behind it.
They made two comparisons.
First, were infants more likely to pull the screen forwards when an object was placed behind it?
Second, were how did infants' performance compare when the barrier was not opaque but transparent?

\subsection{slide-39}
Here are their results with 7-month old infants.

We are interested in whether infants were more likely to pull the screen forwards when the
object was present than when it was absent.
Since infants wanted the toy, if they knew it was behind the barrier they should have pulled
forward the barrier more often when the toy was behind it.
This is exactly what they did when the barrier was transparent.
But look what happens when the barrier is opaque, so that the toy is not visible to infants
when they have to prepare the pulling action: they no longer pull the barrier more often
when the toy was behind it.

This is good evidence that 7 month olds do not know facts about the locations of objects
they cannot perceive.
And this is not isolated evidence; for example, \citet{moore:2008_factors} use a different
methods also involving manual search to provide converging evidence for this conclusion.
But now we have a problem ...

\subsection{032-discrepant-findings}
When do humans first come to know facts about the locations of objects they are not perceiving?

The evidence appears to be contradictory.

\subsection{slide-41}
By measuring looking actions, we find infants can distinguish situations in ways that indicate
they do know facts about the locations of particular unperceived objects.

\subsection{slide-42}
But when measuring retrieval or searching actions, we find infants cannot distinguish these
situations; this indicates that they cannot know this.

\subsection{slide-43}
You might hope there would be a simple solution.  Perhaps, for example, infants have
difficulties reaching that mask their real knowledge of the facts about unperceived objects'
locations.
But As Jeanne Shinskey, one of the researchers most dedicated to this issue says,

‘action demands are not the only cause of failures on occlusion tasks’
\citep[p.\ 291]{shinskey:2012_disappearing}.

Many such explanations have been tried because many researchers have been puzzled by this;
\citet{Meltzoff:1998wp} go as far as to call it a paradox (the 'paradox of early permanence').
No explanation positing extraneous task requirements, such as difficulties performing
an the actions required, has yet succeeded.

\subsection{slide-44}
This is a discrepancy between two types of measure; one involves looking, other other
searching.
We find this pattern--discrepant findings pointing to opposite conclusions about what
infants and adults know--in many different domains.

\subsection{slide-45}
As \citet[p.\ 994]{charles:2009_object} put it, these findings are ‘the tip of an iceberg’.

‘violation-of-expectation experiments, using looking-time measures, suggested that infants
have object permanence in occlusion conditions; but simplified-search studies confirm that
infants fail to reach towards occluded objects, suggesting that infants do not have object
permanence in occlusion conditions. This discrepancy, however, is only the tip of the
iceberg. Results of studies attempting to measure infants’ cognitive abilities using reaching
measures often contradict results gained while using looking-time measures.’
\citep[p.\ 994]{charles:2009_object}

\subsection{slide-46}
You might be wondering whether there's a philosophical problem here.
Science is a messy business and you get conflicting results all the time.
But this particular pattern of conflicting results is extremely interesting philosophically.
It shows that we cannot say that, at, say, five months of age, infants know facts about the
locations
of particular unperceived objects.
We cannot say this because doing so generates predictions which are clearly false (predictions
about where they will search for an unperceived object).
But it also shows that we cannot say that they have no sense at all concerning facts about the
locations of particular unperceived objects.  We cannot say this because of the
competence they manifest in distinguishing possible from impossible events.

The problem, then, is that understanding the origins of knowledge requires us to identify
something inbetween knowledge and its absence, something that is like knowledge in some
respects but falls short of it in others.
This is an instance of Davidson's challenge ...

\subsection{slide-48}
Think of the spider as the object behind Baillargeon’s drawbridge
and behind Shinskey and Munakata’s screens.

\subsection{slide-49}
Can Wy (the infant) act for the reason that there is an object behind
the drawbridge?

Yes: she looks longer when the drawbridge rotates 180 degrees
for the reason that there is an object behind it.

No: if she could act for the reason that there is an object behind
behind Shinskey and Munakata’s screens, then she would search for that object.

So Wy is awkward. She can do some things for the reason that there is an
object behind a screen (e.g. look longer), but she cannot do other things
for this reason.
This is not allowed for by the Uncomplicated Account of Minds and Actions.

\subsection{slide-50}
This suggests that we can’t say that Wy belives that there is an object
behind the screen.
Yes, puzzlingly, she can perform some actions for the reason that
there is an object behind the screen.
So she is not entirely neutral on whether there is an object behind the screen.

\subsection{slide-51}
Just as Davidson says, Wy (and infants generally) are a problem because
their actions are not mindless, but we cannot think of them as having
beliefs or knowledge states as these are usually characterised
(for example, by the Uncomplicated Account of Minds and Actions).

The findings I’ve just reviewed indicate that infants’ behaviours in the face of
objects that disappear from view is not mindless,
but nor does it involve knowledge or belief about the objects.

The challenge we face is to characterise the nature of infants’ representations
and actions.

\subsection{slide-52}
So I agree with
Hood and colleagues about a central challenge involved in understanding
the developmental emergence of knowledge.

‘there are many separable systems of mental representations ... the task ... is to ... [find] the distinct systems of mental representation and to understand their development and integration’\citep[p.\ 1522]{Hood:2000bf}.

I don’t think this is the only challenge, by the way.
A further challenge concerns the role of social interaction in
explaining development.  We’ll come to that later.

\subsection{slide-53}
To sum up so far, the question for this course is, How do humans first come to know  about--and
to knowingly manipulate--objects, causes, words, numbers, colours, actions and minds?
I've been suggesting we can't answer it simply by appealing to nativism, empiricism or other
grand myths.
Instead we need to focus on the particular mechanisms that are involved in different cases.

But then you might wonder, What philosophical questions arise here?  Isn't this a narrowly
pscyhological--and therefore scientific--issue?
The answer is no because thinking about how humans come to know things requires us to meet
Davidson's challenge, to understand things that are neither mindless nor thought or knowledge
but somewhere in between.
As Hood suggests in the quote I just showed you, this might involve rethinking what knowledge
is.

\subsection{slide-54}
I hope I've given you a flavour of the approach we're going to take.
Good philosophy of mind has always been driven by scientific findings about the mind.
John Locke, David Hume as well as more recent philosophers like Jerry Fodor and Andy Clark all
start with a deep understanding of the science of the mind.
But there is a difference.

Fodor and many other contemporary philosophers are working on the big picture, trying to make
explicit general features of the conceptual framework which scientists have more or less
implicitly adopted.
They are also often intrested in questions about the foundations of psychology itself.

By contrast, what I want us to do in this course is to look at specific problems that arise
from the evidence,
and to provide philosophical tools for tackling this problem.
So you might say that whereas others are trying to be the architect, we're trying more
modestly to build bits of the picture.

This might sound too modest to be interesting.
You'd probably prefer to be the architect whose plans guide the scientists rather than the
underlabourer who puts the bricks in place; who wouldn't?
But, as we'll see, it turns out that attention to the details will give us new perspectives on
some key philosophical issues about the nature of knowledge, perception and action.

\subsection{slide-55}
We're going to try to understand how humans come to know about things by examining what
developmental psychology tells us about the acquisition of knowledge.
This turns out to be a partly philosophical project because understanding the apparently
conflicting evidence requires us to re-think notions like knowledge and representation.
In practice, this means looking carefully, and in detail, at the scientific evidence.
If you want to know how minds work, you have to start with the evidence.

\subsection{slide-57}
lectures are at this time every week

\subsection{slide-58}
there is a web page where you can find slides and handouts from lectures.

\subsection{slide-59}
submit a 1500 word unassessed essay by the standard deadline (which I think is 12 noon on
Thursday of week 7 but you should check).

\subsection{slide-60}
seminars start next week and run every week

\subsection{slide-61}
there are no lectures or seminars in reading week (week 6)

\subsection{slide-62}
sign up on tabula, as usual.

\subsection{slide-63}
I've assigned you a series of tasks to do in seminars; these are specified in the
document going around, which is also on the web page.

\subsection{slide-65}
I need to set you up for your first seminar, which starts next week.
Recall these findings ...

\subsection{slide-66}
When do humans first come to know facts about the locations of objects they are not perceiving?

I want you to look at this discrepancy more carefully in your first seminar.
This is partly because the discrepancy matters, and partly because part of doing this module
means becoming familiar with reading scientific papers.

\subsection{unit\_061b}


\section{Social Interaction: Acquiring Your First Words}

So far I have focussed on the nature of mental states and actions.
This will indeed be a big theme.
But a second big theme concerns the nature of social interaction.
I’m quite struck by the fact that in science, research on the
developmental origins of mind is neatly divided between researchers
who want to know what is going on the head of an infant
and researchers who want to know how infants interact with others and
how these interactions facilitate their development.
It seems obvious that we need to integrate both perspectives.
This turns out not to be trivial.

\subsection{slide-69}
Here is a bold conjecture about how humans come to know things.

\subsection{slide-70}
The challenge, of course, is to say *how* social interaction enables humans to come to
know things.

\subsection{slide-71}
Let me give you a hint about why social interaction will be important now.
As in the case of knowledge of objects, this is a preview of a topic
that we will later consider in more detail.

\subsection{slide-73}
‘we grasp the concept of truth only when we can communicate the
contents---the propositional contents---of the shared experience, and
this requires language’

\subsection{slide-72}


\section{Training}

\subsection{slide-77}
For now I'm assuming that Davidson is right that somoene who can think
communicate with language.
What account of language acquisition is consistent with this assumption?
A clue is given by Davidson ...

\subsection{slide-78}
So we might suppose that acquiring a language involves learning how to
act without learning that anything is the case.
This is the general idea.  How can we make it concrete?

\subsection{slide-79}
Our question is, How do humans first come to communicate using words?

Let's start with Bertrand Russell.

\subsection{slide-80}
But how does the environment determine habits and associations?

\subsection{slide-81}
Wittgenstein suggests that the habits are determined by training.

But how does this training work?

\subsection{slide-83}
But now what are these habits and associations?

\subsection{slide-84}
One answer is suggested by Quine.

So this is the picture.

For each word, there is a set of 'stimulations' in response to which an utterance of that word would be appropriate.

For instance, we might suppose there's a set of banana stimulations in response to which an utterance of the word 'banana' would be appropriate.

The child then comes to use the word 'banana' in response to the bananana-stimuluations by means of being trained.

She is rewarded for using 'banana' correctly or punished for using it incorrectly (or both) and so she gradually zeros in on the correct pattern of use.

\subsection{slide-86}
This seems to be approximately Davidson's own view.

\subsection{slide-95}
Children acquiring language create their own words before they learn to use those of the adults
around them.

‘Some children are so impatient that they coin their own demonstrative pronoun. For
instance, at the age of about 12 months, Max would point to different objects and say
“doh?,” some¬times with the intent that we do something with the objects, such as bring
them to him, and sometimes just wanting us to appreciate their existence’
(\citealp[p.\ 122]{Bloom:2000qz}; see further \citealp{Clark:1981bi,Clark:1982hj}).

Even where children have mastered a lexical convention, they will readily violate it in
their own utterances in order to get a point across.

‘From the time they first use words until they are about two or two-and-a-half,
children noticeably and systematically overextend words. For example, one child used
the word “apple” to refer to balls of soap, a rubber-ball, a ball-lamp, a tomato,
cherries, peaches, strawberries, an orange, a pear, an onion, and round biscuits’
\citep[p.\ 35]{Clark:1993bv}

\subsection{slide-100}
Children can create their own languages
with no experience of others' languages

\subsection{slide-101}
We know this from studies of profoundly deaf children brought up in purely oral
environments and therefore without experience of language (Goldin-Meadow 2003; Kegl,
Senghas and Coppola 1999; Senghas and Coppola 2001).
Individually or in groups these children invent their own signed languages.
These languages are not as rich as those of children with experience of other people's
languages but they have all of the essential features of language including lexicons and
syntax (Goldin-Meadow 2002, 2003).
The children invent gesture forms for words which they use with the same meanings in
different contexts, they adopt standard orderings for combining words into sentences,
and they use sentences in constructing narratives about past, present, future and
hypothetical events. Thus one profoundly deaf child, Qing, describes how swordfish can
poke a person so that she dies, and how they have long, straight noses and can swim
(Goldin-Meadow 2003: 170).

\subsection{slide-102}
So how is this related to the idea (mentioned a moment ago)
that social interaction plays a key role in the developmental emergence of knowledge?
My suggestion, to be developed more fully later in the couse,
is that children come to know their first words not through
being trained or taught, nor through observing others and mapping words to concepts.
Instead, some children come to know their first words through
creating words and making themselves understood to others.
That is, through social interaction.

One consequence of this is that it seems we must reject the
claim, made by Davidson and others, that
If someone can think, she can communicate with words ...

\subsection{slide-103}
We’ve just been considering how children do acquire their first words.

\subsection{slide-105}
So here's my challenge to Davidson and others who hold that anyone can communicate with language can think:

explain how someone could begin to create words without already being able to think.

As I've been explaining, the challenge arises because children who have no language and no significant experience of language can create languages of their own.

\subsection{slide-107}
So we have to reject this answer.

\subsection{slide-108}
For my part, I think it's probably time to drop the assumption.
Not because we've shown it's wrong, but because there's no good argument for it an a
significant obstactle to accepting it.
So let's return to our overall question without that assumption.
(Recall that the question was, How do humans first come to communicate with words?)

\subsection{slide-109}
In this first lecture, I’ve tried to give you a sense of what the module
will be about.
The question is, How do humans first come to know simple facts about objects,
words, colours, minds and the rest?
I’ve suggested that
reflecting on discoveries about how infants acquire knowledge of phyiscal objects
challenges us to rethink the nature of minds and actions.
And I’ve also suggested that reflecting on how humans learn---or create---their
first words motivates the idea that the emergence of knowledge in development
may hinge on  quite sophisticated forms of social interaction.


%--- end paste
%---------------






\bibliography{$HOME/endnote/phd_biblio}



\end{document}
