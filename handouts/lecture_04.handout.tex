%!TEX TS-program = xelatex
%!TEX encoding = UTF-8 Unicode

\documentclass[12pt]{extarticle}
% extarticle is like article but can handle 8pt, 9pt, 10pt, 11pt, 12pt, 14pt, 17pt, and 20pt text

\def \ititle {Origins of Mind}
 
\def \isubtitle {Lecture 02}
 
\def \iauthor {Stephen A. Butterfill}
\def \iemail{s.butterfill@warwick.ac.uk}
\date{}

%for strikethrough
\usepackage[normalem]{ulem}

\input{$HOME/Documents/submissions/preamble_steve_handout}

%\bibpunct{}{}{,}{s}{}{,}  %use superscript TICS style bib
%remove hanging indent for TICS style bib
%TODO doesnt work
\setlength{\bibhang}{0em}
%\setlength{\bibsep}{0.5em}


%itemize bullet should be dash
\renewcommand{\labelitemi}{$-$}

\begin{document}

\begin{multicols}{3}

\setlength\footnotesep{1em}


\bibliographystyle{newapa} %apalike

%\maketitle
%\tableofcontents




%--------------- 
%--- start paste

      
 
 
   
      
\def \ititle {Origins of Mind}
 
\def \isubtitle {Lecture 04}
 
 
 
\
 
 
 
\begin{center}
 
{\Large
 
\textbf{\ititle}: \isubtitle
 
}
 
 
 
\iemail %
 
\end{center}
 
 
 
\section{Categorical Perception of Colour}
 
What is categorical perception of colour commonly taken to explain?

    
The diagram below represents sequences of three colours.

    
The vertical sequence shows three greens and the uppermost horizontal sequence shows a blue, a purple and a pink.

    
\begin{center}

    
\includegraphics[scale=0.3]{daoutis_2006_fig_A1.png}

    
\end{center}

    
\begin{center} \citealp{Daoutis:2006ij} figure A1 \end{center}
 
Each colour differs from its neighbours by the same amount according to a standard measure based on the human eye's abilities to discriminate wavelengths.
 
Yet the greens are often judged to look quite similar and the blue-pink-purple to look very different \citep[p.\ 12--7]{Roberson:1999rk}.

    
When people are asked to name these colours, they often give the same name to the greens but different names to members of the blue-pink-purple sequence.

    
And people are generally faster and more accurate in discriminating between members of the blue-pink-purple sequence than members of the green sequence (faster: \citealp{Bornstein:1984cb}; more accurate: \citealp[p.\ 22--7]{Roberson:1999rk}).
 
\textbf{pop-out} ‘Such targets pop out of the display, so that the time it takes to find them is independent of the number of distractors’ \citep[p.\ 117]{Treisman:1986pm}.
 
When target and distractors differ in colour category there can be pop-out effects \citep{Daoutis:2006qk}.
 
A process is \emph{automatic} just if
whether it happens is independent of the subject’s task and motivation (to a significant degree)
 
vMMN (visual mismatch negativity): an event-related potential  thought to index pre-attentive change detection in the visual cortex

 
 
 
\section{Categorical Perception in Infancy}
 
Categorical perception of colour emerges early in infancy.  This has been demonstrated with four-month-olds using habituation \citep{Bornstein:1976of} and visual search \citep{Franklin:2005xk}.
 
Slightly older infants can make use of colour properties such as red and green to recognise objects.
 
For instance, nine-months-olds can determine whether an object they saw earlier is the same as a subsequently presented object on the basis of its colour \citep{Wilcox:2008jk}.
 
By the time they are two years old, toddlers who do not comprehend any colour words can use colour categories implicitly in learning and using proper names; for instance, they are able to learn and use proper names for toy dinosaurs that differ only in colour \citep[][Experiment 3]{Soja:1994np}.
 
So infants and toddlers enjoy categorical perception of colour and may benefit from it in recognising and learning about objects.
 
However children only acquire concepts of, and words for, colours some time later; and colour concepts, like colour words, are acquired gradually \citep{Pitchford:2005hm,Kowalski:2006hk,Sandhofer:1999if,Sandhofer:2006qo}.
 
\subsection{Other cases}
 
Infants enjoy categorical perception not only of colour but also of orientation \citep{franklin:2010_hemispheric}, speech \citep{Kuhl:1987la,Kuhl:2004nv,Jusczyk:1995it} and facial expressions of emotion \citep{Etcoff:1992zd,Kotsoni:2001ph,Campanella:2002aa}.
 
 
 
\section{Categorical Perception and Knowledge}
 
Categorical perception provides ‘the building blocks—the elementary units—for higher-order categories’ 
\citep[p.\ 3]{Harnad:1987ej}.
 
‘The building blocks of all our complex representations are the representations that are constructed from individual core knowledge systems.’
\citep[p.\ 307]{Spelke:2003fc}
 
‘The module  … automatically provides a conceptual identification of its input for central thought … in exactly the right format for inferential processes’ 
\citep[pp.\ 193--4]{Leslie:1988ct}
 
Acquiring colour concepts depends on acquiring colour words
\citep{Kowalski:2006hk}.
 
‘the course of acquisition for color is protracted and errorful’
\citep{Sandhofer:2006qo}
 
‘the earliest conceptual functioning consists of a redescription  of perceptual structure’ 
\citep{Mandler:1992vn}
 
Colour words shape adults’ categorical perception \citep{Roberson:2007wg,Winawer:2007im}.
 
Categorical perception provides ‘the building blocks—the elementary units—for higher-order categories’
\citep[p.\ 3]{Harnad:1987ej}.
 
\emph{A Conjecture}
‘humans acquire knowledge at a pace far outstripping that found in any other species. 
Recent evidence indicates that interpersonal understanding—in particular, skill at 
inferring others’ intentions—plays a pivotal role in this achievement.’
\citep[p.\ 40]{Baldwin:2000qq}
 
‘functions traditionally considered hallmarks of individual cognition originated through the need to interact with others ...\ perception, action, and cognition are grounded in social interaction.’
\citep[p.\ 103]{Knoblich:2006bn}
 
Vygotskian Intelligence Hypothesis:
‘the unique aspects of human cognition ... were driven by, or even constituted by, social co-operation.’
\citep[p.\ 1]{Moll:2007gu}
 
‘human cognitive abilities ... [are] built upon social interaction’
\citep{sinigaglia:2008_roots} %*page
 
 
 
\section{Core Knowledge}
 
For someone to have \textit{core knowledge of a particular principle or fact} is for her to have a core system where 
            either the core system includes a representation of that principle or else the principle plays a special role in describing the core system.
 
\subsection{Two-part definition}
 
‘Just as humans are endowed with multiple, specialized perceptual systems, so we are endowed with multiple systems for representing and reasoning about entities of different kinds.’
\citep[p.\ 517]{Carey:1996hl}
 
‘core systems are 
largely innate 

              
encapsulated 

              
unchanging 

              
arising from phylogenetically old systems 

              
built upon the output of innate perceptual analyzers’ 
\citep[p.\ 520]{Carey:1996hl}.
 
\textit{Note} There are other, slightly different statements \citep[e.g.][]{carey:2009_origin}.
 
\subsection{Compare modularity}
 
Modules are ‘the psychological systems whose operations present the world to thought’; 
    they ‘constitute a natural kind’; and 
    there is ‘a cluster of properties that they have in common’ \citep[p.\ 101]{Fodor:1983dg}.
 
These properties include:
 
\begin{itemize}
 
\item domain specificity (modules deal with ‘eccentric’ bodies of knowledge)
 
\item limited accessibility (representations in modules are not usually inferentially integrated with knowledge)
 
\item information encapsulation (modules are unaffected by general knowledge or representations in other modules)
 
\item innateness (roughly, the information and operations of a module not straightforwardly consequences of learning; but see \citet{Samuels:2004ho}).
 
\end{itemize}
 
\subsection{Objection}
 
‘there is a paucity of … data to suggest that they are the only or the best way of carving up the processing,
 
‘and it seems doubtful that the often long lists of correlated attributes should come as a package’
\citep[p.\ 759]{adolphs_conceptual_2010}
 
‘we wonder whether the dichotomous characteristics used to define the two-system models are … perfectly correlated …
[and] whether a hybrid system that combines characteristics from both systems could not be … viable’
\citep[p.\ 537]{keren_two_2009}
 
‘the process architecture of social cognition is still very much in need of a detailed theory’
\citep[p.\ 759]{adolphs_conceptual_2010}
\citep[p.\ 517]{Carey:1996hl}
 
 
 
\section{Appendix: Categorical Perception in Infants and Adults (Optional)}
 
In adults, categorical perception of colour disappears in the face of predictable verbal interference but not non-verbal interference
\citep{Roberson:2000ge,Pilling:2003bi,Wiggett:2008xt}.
 
‘surprising it would be indeed if I have a perceptual experience as of red because I call the perceived object ‘red’’
\citep[pp.\ 324--5]{Stokes:2006fd}
 
There is evidence that the infant mode of categorical perception of colour continues to operate in adults, although it is often inhibited or overshadowed by the adult mode \citep{Gilbert:2006yb}.
 

    
 
%--- end paste
%--------------- 
 
\footnotesize 
\bibliography{$HOME/endnote/phd_biblio}

\end{multicols}

\end{document}