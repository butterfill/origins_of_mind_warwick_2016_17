% !TEX encoding = UTF-8 Unicode

%\def \papersize {a5paper}
\def \papersize {a4paper}
%\def \papersize {letterpaper}

%\documentclass[14pt,\papersize]{extarticle}
\documentclass[12pt,\papersize]{extarticle}
% extarticle is like article but can handle 8pt, 9pt, 10pt, 11pt, 12pt, 14pt, 17pt, and 20pt text

\def \ititle {Origins of Mind}
\def \isubtitle {Pre-approved Essay Questions}
%comment some of the following out depending on whether anonymous
\def \iauthor {Stephen A.\ Butterfill}
\def \iemail{s.butterfill@warwick.ac.uk% \& corrado.sinigaglia@unimi.it
}
%\def \iauthor {}
%\def \iemail{}
%\date{}

% \usepackage{bibentry}
% \nobibliography{$HOME/endnote/phd_biblio}

%\input{$HOME/Documents/submissions/preamble_steve_paper3}
\input{$HOME/latex_imports/preamble_steve_report2_pdftex}

% disables chapter, section and subsection numbering
\setcounter{secnumdepth}{-1}

%avoid overhang
\tolerance=5000

% no indent, space between paragraphs:
\usepackage[parfill]{parskip}

\begin{document}



\setlength\footnotesep{1em}


\maketitle

%\renewcommand{\cftsecfont}{\normalsize}
\setcounter{tocdepth}{1}

\noindent
\textit{For any of these questions, your answer may focus on a particular domain, such as core knowledge of objects or of number.  You are not required to provide a comprehensive survey.}

\emph{The readings suggested here are for general guidance.  You’re welcome to see me to discuss readings in relation to your plans for the essay.}



\tableofcontents
\clearpage


\section{Mindreading}
What is the puzzle about when humans can first represent others’ beliefs?
How might the puzzle be resolved?

\subsubsection{---Reading}
\fullcite{Onishi:2005hm}

\fullcite{kovacs_social_2010}

\fullcite{Baillargeon:gx}

\fullcite{butterfill_minimal}

\fullcite{carruthers:2013_mindreading}



\section{Core knowledge}
What is core knowledge and what role, if any, could it play in explaining the transition from being unable to know things to being able to know things?


\subsubsection{---Reading}

\fullcite{Carey:1996hl}

\fullcite{spelke:1992_origins}

\fullcite{Spelke:2007hb}

\fullcite{carey:2009_origin}

\section{Innateness}
What if anything is innate in humans?

Hint: You should be careful to examine the notion of innateness  (see \citealp{Samuels:2004ho}).  Otherwise the reading is divided into topics; you should not try to cover all topics.  I also suggest \emph{not} structuring your essay by topic.

\subsubsection{---Reading}

\fullcite{Samuels:2004ho}

\subsubsection{---Reading: comparative (cross-species)}

\fullcite{chiandetti:2011_chicks_op}

\fullcite{haun:2010_origins}


\subsubsection{---Reading: syntax}

Note: this is one-sided.

\fullcite{lidz:2003_what}

\fullcite{lidz:2004_reaffirming}




\subsubsection{---Reading: replying to Fodor’s argument}

\fullcite{Fodor:1981ep}

\fullcite{carey:2009_origin} chapters 4, 8

(There is also an exchange between Carey and Rey forthcoming in the journal Mind and Language---their papers may be available by the time you read this.)




\section{Knowledge of colour}
At birth humans do not know this lime fruit is green whereas that tomato is red.
How could some humans come to be in a position to know this?

Hint: you should discuss categorical perception of colour and its relation to knowledge.  There was a lecture on this topic; the handout includes many references.

\subsubsection{---Reading}

\fullcite{Kowalski:2006hk}

\fullcite{Franklin:2005hp}

\fullcite{Franklin:2005xk}

%\fullcite{franklin:2010_hemispheric}

\fullcite{Wiggett:2008xt}




\section{Social interaction}
If humans were incapable of social interaction and could only observe each other from behind one-way mirrors (if such a thing existed), how if at all would this affect the processes by which they get to know things?

\subsubsection{---Reading}


\fullcite{tomasello:2008origins}

\fullcite{Tomasello:2005wx} [Read the commentaries.]

\fullcite{butterfill:2012_interacting}



\section{Joint action}

Could there be a role for joint action in explaining how humans come to know things about other minds?

\subsubsection{---Reading}

\fullcite{Moll:2007gu}

\fullcite{carpenter:2009_howjoint}

\fullcite{Tollefsen:2005vh}

\fullcite{Butterfill:2011fk}



\section{Language}
‘Children learn words through the exercise of reason’ (BLOOM).  Discuss.

Note: the reading for this is one-sided, which makes this question difficult.


\subsubsection{---Reading}

\fullcite{Bloom:2000qz}

\fullcite{Baldwin:2000qq}

\fullcite{Sabbagh:2001sp}

\fullcite{Matthews:2008yi}

\fullcite{Dummett:1993xn}




%nativism topic readings
%on animals:
% \fullcite{chiandetti:2011_chicks_op}
% \fullcite{haun:2010_origins}
%philosophy:
% \fullcite{Samuels:2004ho}



\section{What is ‘shared intentionality’ and what might it explain?}

Compare \citet[p. 688]{Tomasello:2005wx}: ‘Our ... hypothesis is that it is precisely these two developing capacities [to read intentions and to share psychological states]  that interact during the first year of life to create the normal human developmental pathway leading to participation in collaborative cultural practices.’

%How (if at all) could  shared intentionality be involved in creating ‘the normal human developmental pathway leading to participation in collaborative cultural practices’?

%(Tomasello, Carpenter, et al. 2005: 675): “it is only if a young child understands other persons as intentional agents that she can acquire and use linguistic symbols—because the learning and use of symbols requires an understanding that the partner can voluntarily direct actions and attention to outside entities.”


\subsubsection{---Reading: theoretical}
Pick one of these (first is probably the best; last is shortest; the middle one comes with critical commentaries).

*\fullcite{Moll:2007gu}

\fullcite{Tomasello:2005wx}

\fullcite{Tomasello:2007gl}


\subsubsection{---Reading: experimental (pick one or more)}

\fullcite{Moll:2008fh}

\fullcite{Warneken:2006qe}

\fullcite{Tomasello:2007fi}

\subsubsection{---Reading: background}

\fullcite{Bratman:1992mi}

\fullcite{Searle:1990em}

\fullcite{Butterfill:2011fk}


%\bibliography{$HOME/endnote/phd_biblio}



\end{document}
