%!TEX TS-program = xelatex
%!TEX encoding = UTF-8 Unicode

\documentclass[12pt]{extarticle}
% extarticle is like article but can handle 8pt, 9pt, 10pt, 11pt, 12pt, 14pt, 17pt, and 20pt text

\def \ititle {Origins of Mind}

\def \isubtitle {Lecture 02}

\def \iauthor {Stephen A. Butterfill}
\def \iemail{s.butterfill@warwick.ac.uk}
\date{}

%for strikethrough
\usepackage[normalem]{ulem}

\input{$HOME/Documents/submissions/preamble_steve_handout}

%\bibpunct{}{}{,}{s}{}{,}  %use superscript TICS style bib
%remove hanging indent for TICS style bib
%TODO doesnt work
\setlength{\bibhang}{0em}
%\setlength{\bibsep}{0.5em}


%itemize bullet should be dash
\renewcommand{\labelitemi}{$-$}

\begin{document}

\begin{multicols}{3}

\setlength\footnotesep{1em}


\bibliographystyle{newapa} %apalike

%\maketitle
%\tableofcontents




%---------------
%--- start paste



\def \ititle {Lecture 03}

\begin{center}

{\Large

\textbf{\ititle}

}



\iemail %

\end{center}



\section{A Problem}

‘action demands are not the only cause of failures on occlusion tasks’
\citep[p.\ 291]{shinskey:2012_disappearing}

‘A similar permanent dissociation in understanding object support relations
            might exist in chimpanzees. They identify impossible support relations in looking tasks,
            but fail to do so in active problem solving.’
\citep{gomez:2005_species}

‘to date, adult primates’ failures on search tasks appear to
            exactly mirror the cases in which human toddlers perform poorly.’
\citep[p.\ 17]{santos:2009_object}



\section{Like Knowledge and Like Not Knowledge}

‘no concept causes more problems in discussions of infant cognition than that of representation’
\citep{Haith:1998aq}.



\section{What Is Core Knowledge?}

For someone to have \textit{core knowledge of a particular principle or
fact} is for her to have a core system where
either the core system includes a representation of that principle or
else the principle plays a special role in describing the core system.

\subsection{Two-part definition}

‘Just as humans are endowed with multiple, specialized perceptual systems,
so we are endowed with multiple systems for representing and reasoning
about entities of different kinds.’
\citep[p.\ 517]{Carey:1996hl}

‘core systems are:
\begin{enumerate}
\item largely innate
\item encapsulated
\item unchanging
\item arising from phylogenetically old systems
\item built upon the output of innate perceptual analyzers’ \citep[p.\ 520]{Carey:1996hl}
\end{enumerate}

\textit{Note} There are other, slightly different statements
\citep[e.g.][]{carey:2009_origin}.

‘We hypothesize that uniquely human cognitive achievements build on systems
that humans share with other animals: core systems that evolved before the
emergence of our species.
The internal functioning of these systems depends on principles and processes
that are distinctly non-intuitive.
Nevertheless, human intuitions about space, number, morality and other
abstract concepts emerge from the use of symbols, especially language, to
combine productively the representations that core systems deliver’
\citep[pp.\ 2784-5]{spelke:2012_core}.

\subsection{The Core Knowledge View}

The \emph{Core Knowledge View}: the principles of object perception are not
knowledge, but they are core knowledge. And we generate expectations from
these principles by a process of inference.



\section{Objections to Core Knowledge}

‘there is a paucity of … data to suggest that they are the only or the best way of carving up the processing,

‘and it seems doubtful that the often long lists of correlated attributes should come as a package’
\citep[p.\ 759]{adolphs_conceptual_2010}

‘we wonder whether the dichotomous characteristics used to define the two-system
models are … perfectly correlated …
[and] whether a hybrid system that combines characteristics from both
systems could not be … viable’
\citep[p.\ 537]{keren_two_2009}

‘the process architecture of social cognition is still very much in need of a detailed theory’
\citep[p.\ 759]{adolphs_conceptual_2010}



\section{Core System vs Module}

‘In Fodor’s (1983) terms, visual tracking and preferential looking each may depend on modular mechanisms.’
\citep[p.\ 137]{spelke:1995_spatiotemporal}

\subsection{Modularity}

Fodor’s three claims about modules:

\begin{enumerate}

\item they are ‘the psychological systems whose operations present the world to thought’;

\item they ‘constitute a natural kind’; and

\item there is ‘a cluster of properties that they have in common’ \citep[p.\ 101]{Fodor:1983dg}.

\end{enumerate}

Properties of modules:

\begin{itemize}

\item domain specificity (modules deal with ‘eccentric’ bodies of knowledge)

\item limited accessibility (representations in modules are not usually inferentially integrated with knowledge)

\item information encapsulation (modules are unaffected by general knowledge or representations in other modules)

\item innateness (roughly, the information and operations of a module not straightforwardly consequences of learning; but see \citet{Samuels:2004ho}).

\end{itemize}



\section{A Hypothesis: Object Indexes Underpin Infants’ Abilities}

The \emph{object-specific preview benefit}: ‘observers can identify target
letters that matched the preview letter from the same object faster than
they can identify target letters that matched the preview letter from the
other object’ \citep[p.\ 2]{Krushke:1996ge}.

Hypothesis:



Tracking occluded objects depends on object indexes.



(And reaching for endarkened objects depends on motor representations of objects.)




\section{Phenomenal Expectations Connect Object Indexes to Looking Behaviours}

The principles of object perception



are not items of knowledge



instead



they characterise the operation of



object indexes (FINSTs, components of mid-level object files)

\citep{Leslie:1998zk,Scholl:1999mi,Carey:2001ue,scholl:2007_objecta}.


    



%--- end paste
%---------------

\footnotesize
\bibliography{$HOME/endnote/phd_biblio}

\end{multicols}

\end{document}
