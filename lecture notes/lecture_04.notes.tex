 %!TEX TS-program = xelatex
%!TEX encoding = UTF-8 Unicode

%\def \papersize {a5paper}
\def \papersize {a4paper}
%\def \papersize {letterpaper}

%\documentclass[14pt,\papersize]{extarticle}
\documentclass[12pt,\papersize]{extarticle}
% extarticle is like article but can handle 8pt, 9pt, 10pt, 11pt, 12pt, 14pt, 17pt, and 20pt text

\def \ititle {Origins of Mind: Lecture Notes}
\def \isubtitle {Lecture 01}
%comment some of the following out depending on whether anonymous
\def \iauthor {Stephen A.\ Butterfill}
\def \iemail{s.butterfill@warwick.ac.uk% \& corrado.sinigaglia@unimi.it
}
%\def \iauthor {}
%\def \iemail{}
%\date{}

%\input{$HOME/Documents/submissions/preamble_steve_paper4}
\input{$HOME/Documents/submissions/preamble_steve_lecture_notes}

%no indent, space between paragraphs
\usepackage{parskip}

%comment these out if not anonymous:
%\author{}
%\date{}

%for e reader version: small margins
% (remove all for paper!)
%\geometry{headsep=2em} %keep running header away from text
%\geometry{footskip=1.5cm} %keep page numbers away from text
%\geometry{top=1cm} %increase to 3.5 if use header
%\geometry{bottom=2cm} %increase to 3.5 if use header
%\geometry{left=1cm} %increase to 3.5 if use header
%\geometry{right=1cm} %increase to 3.5 if use header

% disables chapter, section and subsection numbering
\setcounter{secnumdepth}{-1} 

%avoid overhang
\tolerance=5000

%\setromanfont[Mapping=tex-text]{Sabon LT Std} 


%for putting citations into main text (for reading):
% use bibentry command
% nb this doesn’t work with mynewapa style; use apalike for \bibliographystyle
% nb2: use \nobibliography to introduce the readings 
\usepackage{bibentry}

%screws up word count for some reason:
%\bibliographystyle{$HOME/Documents/submissions/mynewapa} 
\bibliographystyle{apalike} 


\begin{document}



\setlength\footnotesep{1em}






%--------------- 
%--- start paste
\title {Origins of Mind: Lecture Notes \\ Lecture 04}
 
 
 
\maketitle
 
\subsection{slide-3}
Recall from the previous lecture that our Next Big Problem is this. 
We've said that infants' competence with causes and objects is not knowledge but something 
more primitive than knowledge, something which exists in adults too and can carry information 
discrepant with what they know. So, if at all, how does appealing to these early capacities 
enable us to explain the origins of knowledge?
 
Versions of this problem will arise in every domain of knowledge we shall consider.
In this lecture I want to consider our Next Big Problem by thinking about the domain of colour.
We are also, of course, simultaneously occupied with a question about the origins of knowledge
of colour.
Let me start by introducing the question ...
 
\subsection{unit\_110}
 
 
\section{Knowledge of Colour: a Question}
 
\subsection{slide-5}
Let me start with something quite basic.
 
Here are three patches of colour.
 
The patches are all different colours, but the two leftmost are both the same colour---they are both blue.
 
This sounds contradictory but isn't.
 
In one case we're talking about the particular colours of things; in the other case we're talking about colour category.
 
 
 
The question I want to ask about knowledge of colour concerns these cateories.
 
You know that these two are both blue whereas this is not (it's green).
 
How do you come to know this?
 
(Ask them to discuss.)
 
Does anyone thing the answer is that you can just see it?  As we'll see, that idea looks promising initiall but it's not quite that simple.
 
\subsection{unit\_111}
 
 
\section{Categorical Perception of Colour}
 
\subsection{slide-8}
Remember the idea that words like 'blue' pick out colour categories?
 
To claim that there is categorical perception is, very roughly and to a first approximation, to claim that perceptual processes are concerned not only with the particular colours of things but also with categories of colour like blue and red.
 
This is not a definition, only a very rough starting point.
 
\subsection{slide-10}
I can't tell you what categorical perception is.
 
The operational definitions used in psychology are unrevealling.
 
And the ways philosophers define it are wrong, or at least depend on unjustified assumptions.
 
\subsection{slide-12}
But how can we study it without knowing what it is?
 
We have to proceed by seeing what conclusions about it are supported by evidence.
 
Then we can work back to infer what it is by asking what it could be if these conclusions are true.
 
Put differently, we can start by asking, What is categorical perception commonly taken to explain?
 
\subsection{slide-13}
At this point we should take nothing for granted and make no assumptions about what sense, if any, categorical perception involves either categories or perception.
 
We  are only asking, What is categorical perception commonly taken to explain?
 
\subsection{slide-14}
What is categorical perception of colour commonly taken to explain?
 
The diagram below represents sequences of three colours.
 
The vertical sequence shows three greens and the uppermost horizontal sequence shows a blue, a purple and a pink.
 
\subsection{slide-15}
Each colour differs from its neighbours by the same amount according to a standard measure based on the human eye's abilities to discriminate wavelengths.
 
\subsection{slide-16}
Yet the greens are often judged to look quite similar and the blue-pink-purple to look very different \citep[p.\ 12--7]{Roberson:1999rk}.
 
When people are asked to name these colours, they often give the same name to the greens but different names to members of the blue-pink-purple sequence.
 
And people are generally faster and more accurate in discriminating between members of the blue-pink-purple sequence than members of the green sequence (faster: \citealp{Bornstein:1984cb}; more accurate: \citealp[p.\ 22--7]{Roberson:1999rk}).
 
\subsection{slide-17}
Let me show you how speed an accuracy of discimination are usually measured.
 
This is a 2-alternative forced-choice task (2AFC).
 
\subsection{slide-20}
So far we've talked about three ways in which the blue-pink-purple sequence differs from the greens.
 
The green sequence also differs from the blue-pink-purple sequence on other measures.
 
\subsection{slide-21}
But these results are not confined to just two sequences.
 
There is a way of dividing a colour space into categories that predicts all of these results and more.
 
When the members of a sequence of colours fall into different categories, the colours are said to look very different, they are given different verbal labels, and the subject can discriminate them faster and more accurately than when they don't fall into different categories (all other things being equal).
 
The fact that, for an individual subject, all these effects are predicted by a single way of dividing the colour space into categories is what categorical perception of colour is supposed to explain.
 
\subsection{categorical\_perception\_df}
Now we're in a position to say what categorical perception is.
 
A single way of dividing up the colour similarity space into a small finite number of equivalence classes enables us to predict a variety of effects including patterns in: reports of phenomenal similarity and difference, the use of linguistic labels, and speed and accuracy of discrimination. 
 
\subsection{slide-25}
This is a fact in need of explanation.
 
\subsection{slide-28}
So what is categorical perception?
 
Categorical perception is the process, whatever it is, that explains the unexpected effects.
 
Now this way of defining categorical perception is not very informative.
 
This is a virtue---categorical perception is something to be discovered, not defined.
 
We might ask, what sort of process is categorical perception?
 
In particular, we might wonder whether and in what sense it is a perceptual process.
 
How can we discover more about categorical perception?
 
One way to do this is to look for further unexpected patterns to be explained.
 
It's just here that pop-out is useful.
 
\subsection{slide-29}
Another measure on which the green sequence differs from the blue-pink-purple sequence is pop-out, which is defined operationally in terms of visual search tasks \citep[p.\ 117]{Treisman:1986pm}.
 
First I need to explain what a visual search task is.
 
In visual search tasks tasks, subjects are shown an array of objects and asked to identify the one with a certain property.
 
For example, one might be shown an array of twenty animal pictures and asked to identify the cat.
 
Normally the time it takes subjects to find a target increases linearly as the number of objects in the array increases.
 
But when the target of a visual search is defined in terms of colour and the colours of the distractors are categorically different from the colour of the target, subjects can find the target just as quickly when it is in among 36 distractors as when there are only four distractors \citep[Experiment 1]{Daoutis:2006qk}.
 
This is, the colours pop out.
 
\subsection{slide-39}
Recall that we defined categorical perception in terms of things it could explain.
 
Now we can add pop-out to that list.
 
This gives us more insight into what categorical perception is.
 
In particular, it indicates that categorical perception is a perceptual process.
 
And we can go on finding out more about categorical perception by finding out more about the things it explains.
 
This is why I refused to define it at the start, and why you have to start without knowing what categorical perception is.
 
\subsection{slide-42}
Our topic is still categorical perception, but I want to pause here to reflect on how we've defined categorical perception.
 
We should think of core knowledge using the same method.
 
At the outset we don't know what core knowledge is.
 
I just said it's like knowledge in some ways but not knowledge.
 
We'll get a handle on it by looking at what it is commonly supposed to explain.
 
More on this later; now back to categorical perception.
 
\subsection{slide-43}
Among the things categorical perception is commonly taken to explain there are also automatic processes.
 
(This will be important later.)
 
\subsection{slide-44}
For an example of a category effect which is automatic, I need to introduce you to the odd-ball paradigm.
 
In the odd-ball paradigm, you see a series of things that are all the same; and then, unpredictably, you see something different.
 
We can use sensitivity to odd-balls to measure the difference between the two blues and the blue-green pair.
 
This is useful because odd-ball effects can be detected very early in visual processessing, before attention kicks in.
 
They allow us to probe automatic processes.
 
\subsection{slide-54}
Now suppose we play you an odd ball sequence.
 
We want to know whether an automatic process picks up on the difference.
 
And we want to know whether the automatic process picks up on the difference always or only when a category change is involved.
 
To do this we need to measure an event-related potential called vMMN (which is short for visual mismatch negativity).
 
\subsection{slide-56}
Now I need to explain what an event-related potential is.
 
\subsection{slide-60}
An ERP like the vMMN stands to an overall EEG measurement a bit like the number 78 bus does to a collection of atoms.
 
\subsection{slide-61}
While we're on methods, let's quickly talk about fMRI.
 
\subsection{slide-65}
So now we know what an event-related potential is, let's get back to the vMMN (visual mismatch negativity).
 
\subsection{slide-66}
So let me show you a task that uses this odd-ball effect
 
Fixate on this cross.
 
And when you see the rectangle, press the space bar with both index fingers.
 
That's it.
 
Except that you also see a block of colour on every trial.
 
You're instructed to ignore this block of colour.
 
And generally that's easy because it's always the same colour.
 
But just occasionally it's a different colour.
 
In one condition, the usual and odd-ball colours are the blues.
 
And in the other condition, the usual and odd-ball colours are cross-category, from blue-green.
 
We're interested in whether there's a sign that your brain has detected the change when the odd ball colour occurs.
 
Put roughly, the results show that there is signal of pre-attentive change detection at around 100-200 milliseconds in the left visual field.
 
\subsection{slide-71}
This is another way of presenting the same finding.
 
The vMMN is a DRN, but not all DRNs are vMMNs (see the paper).
 
\subsection{slide-76}
Recall that we defined categorical perception in terms of things it could explain.
 
Now we can add a further item to that list: odd-ball detection.
 
Again, this gives us more insight into what categorical perception is.
 
In particular, it indicates that categorical perception is an automatic process.
 
It isn't only a matter of how you experience things to be.
 
\subsection{unit\_121}
 
 
\section{Categorical Perception in Infancy}
 
\subsection{slide-80}
Categorical perception of colour emerges early in infancy, from around four months or earlier, as has been demonstrated using habituation (Bornstein, Kessen and Weiskopf 1976).
 
\subsection{slide-86}
Categorical perception has also been demonstrated at four months of age using pop-out (Franklin, Pilling and Davies 2005).
 
\subsection{slide-92}
Slightly older infants can also make use of colour properties such as red and green to recognise objects.  For instance, nine-months-olds can determine whether an object they saw earlier is the same as a subsequently presented object on the basis of its colour (Wilcox, Woods and Chapa 2008).
 
\subsection{slide-93}
By the time they are two years old, toddlers who do not comprehend any colour terms can use colour categories implicitly in learning and using proper names; for instance, they are able to learn and use proper names for toy dinosaurs that differ only in colour (Soja 1994: Experiment 3).
 
So we see categorical perception of colour early in development and it is useful to children.
 
\subsection{slide-94}
Infants enjoy categorical perception not only of colour but also of orientation (Franklin et al. 2010), speech (Kuhl 1987, 2004; Jusczyk 1995) and facial expressions of emotion (Etcoff \& Magee 1992; Kotsoni et al. 2001; Campanella et al. 2002).
 
\subsection{slide-110}
To recap, infants enjoy categorical perception not only of colour but also of orientation (Franklin et al. 2010), speech (Kuhl 1987, 2004; Jusczyk 1995) and facial expressions of emotion (Etcoff \& Magee 1992; Kotsoni et al. 2001; Campanella et al. 2002).
 
\subsection{slide-113}
But our focus for now is colour.
 
\subsection{unit\_131}
 
 
\section{Categorical Perception and Knowledge}
 
\subsection{slide-115}
Recall that our Next Big Problem is this. 
We've said that infants' competence with causes and objects is not knowledge but something 
more primitive than knowledge, something which exists in adults too and can carry information 
discrepant with what they know. So, if at all, how does appealing to these early capacities 
enable us to explain the origins of knowledge?
 
As we've just seen, a version of this problem arises for the case of colour.
Perhaps we can partially resolve the Next Big Problem in the case of colour, and then use 
this as a model for other cases.
 
Let's see.
First I want to be clear about what the problem is ...
 
\subsection{slide-116}
Our overall question is about how humans come to know things about colours.
 
How is categorical perception of colour relevant to this question?
 
\subsection{slide-117}
Harnad offers a hypothesis which might seem natural to accept at this point.
 
\subsection{slide-118}
Incidentally, Spelke also uses the building blocks metaphor.
 
But of course Spelke is talking about core knowledge rather than categorical perception.  (I think the two might be related, but I suspect she would strongly disagree.)
 
\subsection{slide-121}
But is it true?
 
\subsection{slide-123}
The first problem is to understand the ‘building blocks’ metaphor.
 
\subsection{building\_blocks\_view\_1}
Here's a first attempt to understand the ‘building blocks’ metaphor.
 
Let's suppose that categorical perception of colour amounts to having colour concepts.
 
This way of thinking concords with Leslie's view about modular cognition.
 
\subsection{slide-127}
We'll say something about what modules are later---for now, just be aware that categorical perception counts as a modular process.
 
\subsection{slide-129}
It can't be right that having categorical perception of colour (or X) amounts to having colour (or X) concepts.
 
We know this can't be right because there is a gap of two or more years between categorical perception of colour and possession of colour concepts.
 
For consider what is involved in having a colour concept.
 
It is not just responding differently to red things than to green things (say).
 
It also requires
 
\subsection{slide-130}
The toy is put away so that the subjects couldn't see it.
 
\subsection{slide-135}
Kowalski and Zimiles' paradigm provides an excellent operationalisation of concept possession.
 
We should hold in mind the idea that having a concept of something involves being able to think abstractly about that thing.
 
\subsection{slide-136}
But for now Kowalski and Zimiles are important because they show that having categorical perception of colour is not the same thing as having colour concepts.
 
\subsection{slide-139}
So the first view can't be right.
 
It can't be right that having categorical perception of colour (or X) amounts to having colour (or X) concepts.
 
And the same applies in other cases of categorical perception as well.
 
(Of course a defender of the view that categorical perception is or entails concept possession might insist that subjects have concepts but cannot use them, at least not in the sort of task I have just described.  This is possible, of course, but it seems to be a desperate defence.)
 
How else might we understand the ‘building blocks’ metaphor?
 
\subsection{building\_blocks\_view\_2}
Consider Mandler's claim that ‘the earliest conceptual functioning consists of a redescription of perceptual structure’ 
Mandler (1992)

 
\subsection{slide-144}
One worry here is that we've replaced one metaphor with another, ‘redescription’ for ‘building block’.
 
But let's put this aside for a moment in order to reflect on why Mandler is wrong.
 
\subsection{slide-146}
The boundaries of adults' perceptual categories do not match the boundaries of infant and toddler perceptual categories.
 
How do we know?
 
First note that the extensions of colour terms vary between languages.
 
\subsection{slide-147}
Indeed, the extensions of colour terms vary quite radically between languages.
 
Ongoing research concerns whether there is any kind of universal prinicple behind the pattern.
 
\subsection{slide-149}
So the extensions of colour terms varies between langauges.
 
Why is this relevant to the relation between infant and adult categorical perception?
 
\subsection{slide-150}
Because the boundaries of adults' (but not infants') perceptual categories are influenced by the extensions of their culturally variable colour terms.
 
For evidence that adults' perceptual colour categories are influenced by their culturally variable knowledge of colour words, see \citet{Kay:2006ly}, \citet{Roberson:2007wg}, and \citet{Winawer:2007im};
 
\subsection{slide-152}
This is an amazing finding about the power of words.
 
Learning to use words for colours changes how we perceive them.
 
Colour words shape adults’ categorical perception \citep{Roberson:2007wg,Winawer:2007im}.
 
We'll see more on this later.
 
\subsection{slide-153}
But, as you'd expect, infants’ and toddlers’ perceptual categories are not influenced by the extensions of colour terms.
 
For evidence that toddlers who know some colour words show no influence of language on categorical perception of colours, see \citet{Franklin:2005hp}.
 
\subsection{slide-157}
So the second view probably isn't right either.
 
For whatever redescription amounts to, it should surely involve there being some kind of match between the perceptual categories and the concepts.
 
And, on the face of, it there is no match at all.
 
\subsection{slide-160}
So what are we to make of the ‘building blocks’ idea?
 
It may be that we have to reject the building blocks idea altogether.
 
For infant categorical perception of colour seems only very distantly related to adult uses of colour concepts.
 
Does this mean that categorical perception plays no role at all in explaining the emergence of knowledge about colours?
 
No, we haven't shown that.  Only that categorical perception alone is not enough; and also that  it is hard to understand exactly how categorical perception plays a role in the emergence of knowledge.
 
\subsection{slide-162}
The role of some early-developing abilities in explaining the later acquisition of conceptual 
understanding does not involve direct representational connections; rather the early developing 
abilities influence attention and inform behaviour, and these influences facilitate 
development.
 
\subsection{slide-163}
What is missing from the building blocks idea is adequate recognition that social interaction is 
important for the emergence of knowledge.
 
Here is a bold conjecture about how humans come to know things.
 
\subsection{slide-164}
The challenge, of course, is to say *how* social interaction enables humans to come to know 
things.
 
\subsection{slide-165}
In emphasising rediscovery I want to identify a major role for social interaction while
also recognising the importance of the early developing representations of colours, objects 
and causes.
 
The idea is, roughly, that coming to have knowledge involves rediscovering information that is
already represented.
 
\subsection{slide-166}
How might the idea of development as rediscovery apply in the case of knowledge of colour?
 
In acquiring a colour concept or word you need to:
 
\subsection{slide-168}
On locating the boundaries, I think having colour words might be helpful as a way of 
marking a category before you know where its boundaries are.
That is, the word allows you to re-identify a category without knowing where it begins 
or ends.
 
Given differences between the boundaries of pre-verbal infants' perceptual categories
and the extensions of colour words, it seems unlikely that categorical perception could 
help here.  (Although it is possible in principle that, due to abstract features common to 
all languages' systems of colour words, categorical perception might help; see Kay and ?*.)
 
\subsection{slide-171}
It's with this first problem that I think categorical perception helps with.
I suggest that categorical perception enables you to lock onto the relevant dimension.
 
How?
Categorical perception alters your experience so that there's a correspondence between 
(i) changes on the relevant dimension, and (ii) changes in one phenomenal aspect of your 
experience.
This correspondence helps you, eventually, to lock onto the relevant dimension and so to 
acquire concepts and words.
Let me illustrate.
 
\subsection{slide-172}
One effect of categorical perception is phenomenological.
 
Consider the (a) difference between an experience of the first blue and the second blue;
 
and (b) the difference between an experience of the second blue and the green.
 
Phenomenally, the second pair of experiences is more different than the first.
 
There is some kind of phenomenological marker of categorical colour properties.
 
To put it colourfully, there is something that it is like for you to experience blue, and something else that it is like for you to experience green.
 
So, to return to the two problems, this is why I think categorical perception might be important for identifying the dimension.
 
\subsection{slide-173}
Let me return to the picture I finished with at the end of the last lecture.
 
This picture is about core knowledge, but I think we should broaden that to include categorical perception.
 
In fact I think core knowledge and categorical perception are likely to be closely related, as I'll explain later.
 
 
 
The question for this lecture was, how do we come to know things about colour.
 
The answer, I think, is this.
 
We have categorical perception of colour from early in infancy, certainly from around four months of age.
 
Having categorical perception is not by itself sufficient for having colour concepts.
 
But it does play an important role, because it means that there is a correspondence between (i) changes on the relevant dimension, and (ii) changes in one phenomenal aspect of our experience.
 
I conjecture that this helps us to identify the relevant dimension, colour not shape or size, in acquiring colour words and concepts.
 
But by itself this is not enough.
 
As we've seen, acquiring colour concepts also depends on having some colour words.
 
I speculate that this is because words function as a tool for marking categories prior to knowing where they start and end.
 
The word allows you to re-identify a category without knowing where it begins or ends.
 
So this is my picture.
 
Acquiring knowledge of colour depends on an interaction between categorical perception, something developmentally quite primitive, and social interaction, in particular the negotiation of words for colour categories.
 
\subsection{slide-174}
To conclude, we've seen that although infants enjoy categorical perception of colour from four months or earlier,
 
this doesn't straightforwardly explain how they come to know about the colours of things.
 
In particular, having categorical perception does not entail having a concept.
 
Nor does having a concept seem to be a matter of somehow redescribing or latching on to what is catgeorically perceived.
 
On the other hand, having colour words and being able to use them accurately does seeem to matter for the acquisition of knowledge of colour.
 
So this is one example where having verbal labels for things may help with acquiring concepts of them.
 
\subsection{unit\_601\_milan}
 
 
\section{Core Knowledge}
 
\subsection{slide-178}
I talked about the notion of core knowledge in the very first lecture, but since then I 
have not appealed to the notion.
This is deliberate because the notion is tricky; so I thought it would be good to postpone
our discussion of it for as long as possible.
Now I can put it off no longer.
 
\subsection{slide-180}
The first, very minor thing is to realise that there are two closely related notions, core 
knowledge and core system.
These are related this: roughly, core knowledge states are the states of core systems.  More 
carefully:
 
For someone to have \textit{core knowledge of a particular principle or fact} is for her to have a core system where 
either the core system includes a representation of that principle or else the principle plays a special role in describing the core system.
 
So we can define core knowlegde in terms of core system.
 
\subsection{slide-181}
What do people say core knowledge is?
 
There are two parts to a good definition.  The first is an analogy that helps us get a fix on what we is meant by 'system' generally.  (The second part tells us which systems are core systems by listing their characteristic features.)
 
\subsection{slide-183}
So talk of core knowledge is somehow supposed to latch onto the idea of a system.
 
What do these authors mean by talking about 'specialized perceptual systems'?
 
They talk about things like perceiving colour, depth or melodies.
 
Now, as we saw when talking about categorical perception of colour, we can think of the 'system' underlying categorical perception as largely separate from other cognitive systems--- we saw that they could be knocked out by verbal interference, for example.
 
So the idea is that core knowledge somehow involves a system that is separable from other cognitive mechanisms.
 
As Carey rather grandly puts it, understanding core knowledge will involve understanding something about 'the architecture of the mind'.
 
\subsection{slide-184}
Illustration: edge detection.
 
\subsection{slide-187}
 
This, them is the two part definition.  An analogy and a list of features.
 
\subsection{slide-188}
There is one more feature that I want to mention; this is important although I won't disucss it here.
 
To say that a represenation is iconic means, roughly, that parts of the representation represent parts of the thing represented.
 
Pictures are paradigm examples of representations with iconic formats.
 
For example, you might have a picture of a flower where some parts of the picture represent the petals and others the stem.
 
\subsection{slide-189}
The first problem we encountered was that the Simple View is false.
But maybe we can appeal to the Core Knowledge View.
 
According to the Core Knowledge View, the principles of object perception, and maybe also the 
expectations they give rise to, are not knowledge.
But they are core knowledge.
 
This raises some issues.  Is the Core Knowledge View consistent with the claims that 
we have ended up with, e.g. about categorical perception and the Principles of Object 
Perception characterising the way that object indexes work?
I think the answer is, basically, yes.  Categorical perception involves a system that has
many of the features associated with core knowledge.
 
But before going further, I think we critically examine the notion and see whether we really
understand what core knowledge is.
 
\subsection{slide-190}
As a sort of aside, I might mention for anyone familiar with Fodor on modularity 
that core systems basically coincide modules.
 
\subsection{slide-191}
Aside: compare the notion of a core system with the notion of a module
 
The two definitions are different, but the differences are subtle enough that we don't want both.
 
My recommendation: if you want a better definition of core system, adopt core system = module as a working assumption and then look to research on modularity because there's more of it.
 
An example contrasting Grice and Davidson on the wave.
 
\subsection{slide-199}
But is the notion of core system (or module) explanatory?
 
\subsection{slide-200}
One reason for doubting that the notion of a core system is explanator arises from the 
way we have introduced it.
We have introduced it by providing a list of features.
But why suppose that this particular list of features constitutes a natural kind?
This worry has been brought into sharp focus by criticisms of 'two systems' approaches.
(These criticisms are not directed specifically at claims about core knowledge, but the criticisms apply.)
 
‘there is a paucity of … data to suggest that they are the only or the best way of carving up the processing,
 
‘and it seems doubtful that the often long lists of correlated attributes should come as a package’
\citep[p.\ 759]{adolphs_conceptual_2010}
 
\subsection{slide-201}
‘we wonder whether the dichotomous characteristics used to define the two-system models are … perfectly correlated …
[and] whether a hybrid system that combines characteristics from both systems could not be … viable’
\citep[p.\ 537]{keren_two_2009}
 
\subsection{slide-203}
This is weak.
 
Remember that criticism is easy, especially if you don't have to prove someone is wrong.
 
Construction is hard, and worth more.
 
\subsection{slide-206}
Even so, there is a problem here.
 
‘the process architecture of social cognition is still very much in need of a detailed theory’
\citep[p.\ 759]{adolphs_conceptual_2010}
 
\subsection{slide-207}
Given the problems, maybe we should just abandon the notions of core system and core knowledge.
After all, we have come this far without them.
 
Why do we need a notion like core system?
 
\subsection{slide-208}
So why do we need a notion like core knowledge?
 
Think about these domains.
 
In each case, we're pushed towards postulating that infants know things, but also pushed against this.
 
Resolving the apparent contradiction is what core knowledge is for.
 
Key question: What features do we have to assign to core knowledge if it's to describe these discrepancies?
 
I think the fundamental feature is inaccessibility.
 
\subsection{slide-214}
If this is what core knowledge is for (if it exists to explain these discrepancies), what features must core knowledge have?
 
\subsection{slide-215}
Look at those features characterising core knowledge as defined by Carey and Spelke again
-- innate, encapsulated, unchanging and the rest.
None of these straightforwardly enable us to predict that core knowledge of objects will 
guide looking but not reaching.
(I'll say in a moment: 
The iconic format is important, but because it explains a missing feature, namely limited
accessibility.)
 
\subsection{slide-216}
The feature we most need is actually missing from their list.
 
To say that a system or module exhibits limited accessibility is to say that the representations in the system are not usually inferentially integrated with knowledge.
 
I think this is the key feature we need to assign to core knowledge in order to explain the apparent discrepancies in the findings about when knowledge emerges in development.
 
\subsection{slide-217}
Limited accessbility is a familar feature of many cognitive systems.
 
When you grasp an object with a precision grip, it turns out that there is a very reliable pattern.
 
At a certain point in moving towards it your fingers will reach a maximum grip aperture which is normally a certain amount wider than the object to be grasped, and then start to close.
 
Now there's no physiological reason why grasping should work like this, rather than grip hand closing only once you contact the object.
 
Maximum grip aperture shows anticipation of the object: the mechanism responsible for guiding your action does so by representing various things including some features of the object.
 
But we ordinarily have no idea about this.
 
The discovery of how grasping is controlled depended on high speed photography.
 
This is an illustration of limited accessibility.
 
(This can also illustrate information encapsulation and domain specificity.)
 
\subsection{slide-218}
To sum up, there are two problems with the notion of core knowledge ...
 
core systems -- questions:

Is appealing to core systems (modules) explanatory?
Can the Core Knowledge View explain the discrepancy?
Is the view I've been developing consisten with the Core Knowledge View?
Is core knowledge all one thing?
How do you get from core knowledge to knowledge knowledge?
 
The Core Knowledge View is the view that infants' competence with objects, causes, colours
and the rest depends not on knowledge (as the Simple View has it) but on core knowledge.
 
On the first question, I could tell you a long story about computational processes and 
representational formats.  But I won't do that here.
 
On the second question, I think the answer is probably yes because the difference in
representational formats between knowledge proper and core knowledge explains limited
accessibility, and this in turn partially explains the discrepancies.
(But note that it doesn't explain is why the discrepacies fall exactly where they do.)
 
On the third question, I think that thinking about categorical perception, or the perceptual 
systems that underpin objects indexes is a way of getting at what core knowledge really 
involves.  Core systems are not always something over and above perceptual and motor systems
(although it seems probably that they sometimes are if our competence with syntax rests on 
core knowledge).
 
On the fourth question,
Most people are committed to there being different stories for (a) colour (categorical 
perception); (b) syntax (tacit knowledge); and (c) physical objects.
I'm not sure. I'm tempted to think that perceptual and motor cognition are at the root of these 
in every case, but this is a radical idea for which there is currenly very little evidence.
 
The fifth question is just a variant of what I called The Problem (see lecture 2; it arises 
from the failure of the Simple View).
 
\subsection{slide-219}
This picture is significantly different from some competitors (but not Carey on number):
 
(1) because it shows we aren't done when we've explained the acquisition of core knowledge (contra e.g. Leslie, Baillargeon), and
 
(2) because it shows we can't hope to explain the acquisition of knowledge if we ignore core knowledge (contra e.g. Tomasello)
 
\subsection{unit\_141}
 
 
\section{Appendix: Categorical Perception in Infants and Adults (Optional)}
 
\subsection{slide-221}
Earlier I mentioned that infant and adult categorical perception of colour involve different colour categories.
 
Adults’ perceptual colour categories are shaped by their knowledge of words, whereas infants’ are not.
 
This means we should ask, What is the relation between infant and adult categorical perception?
 
A natural hypothesis is that they are one and the same, so that the infant categories transform into the adult ones.
 
This natural hypothesis turns out to be wrong.
 
Before explaining why, let me introduce one more puzzling finding.
 
\subsection{slide-223}
Let me show you how this works (roughly following Wiggett and Davies 2008)
 
Recall from earlier how speed and accuracy were tested using a two-alternative forced-choice task.
 
\subsection{slide-226}
Now Wiggett and Davies 2008 adapted this very slightly.
 
They put a word on the target (or, in Experiment 2, on the test stimuli --- that had no effect, showing that the word is important for priming and categorical perception is not a matter of matching label to label).
 
(They are using the Stroop effect, which I won't explain here but is worth looking up.)
 
\subsection{slide-228}
And here are the results from Experiment 1B.  The vertical axis is mean accuracy.
 
\subsection{slide-230}
Why say that impact of verbal interference plus shaping of perceptual categories by extensions of words gives an interesting sense in which we have expeirences as of red because we label perceived objects 'red'?  Two points: long-term, the extentions of perceptual category is influenced by the extension of the word; short-term, covert labelling primes the perceptual category and without this priming you do not have CP (Wiggett and Davies 2008).
 
\subsection{slide-231}
These findings leave us with a puzzle.
 
In adults, categorical perception of colour depends on the availability of linguistic labels for colour.
 
But infants lack such labels yet still manifest categorical perception.
 
How is this possible?
 
I think we have to resolve the puzzle by recognising that infant and adult categorical perception are different processes.
 
\subsection{slide-232}
The adult mode of categorical perception of colour differs from the infant-and-toddler mode in at least four respects: it disappears in the face of predictable verbal interference but not non-verbal interference (Roberson, Davies and Davidoff 2000: 985; Pilling, Wiggett, et al. 2003: 549-50; Wiggett and Davies 2008), it can be affected by short-term perceptual learning (Ozgen and Davies 2002), it depends on parts of the brain other than those on which infants' categorical perception of colour depends (Franklin, Drivonikou, et al. 2008), and the boundaries of adults' perceptual categories do not match the boundaries of infant and toddler perceptual categories.
 
\subsection{slide-233}
There is evidence that the infant mode of categorical perception of colour continues to operate in adults, although it is often inhibited or overshadowed by the adult mode \citep{Gilbert:2006yb}.
 
\subsection{slide-234}
Our question was how humans make the transition from not being in a position to know that these are blue but this is not to being in a position to know this.
 
The appearance of categorical perception of colour at around four months of age made attractive the idea that the knowledge might come from, or be based on, perception.
 
The tempting thought was that categorical perception of colour somehow provides a building block for colour concepts.
 
We saw that there were some reasons, not conclusive but quite powerful, to reject this idea.
 
In its place I offered a conjecture about development as rediscovery.
 
That conjecture was that in thought we rediscover something that our brains are already able to process, namely colour categories.
 
This suggests a certain amount of duplication.  There's both a low-level cognitive process and a conceptual processes that deal with colour, that are not directly linked by representations but rather only indirectly, via experience, the body and behaviour.
 
What we've just seen is that there is even more duplication than this suggests.
 
There are not just one but two perceptual processes which are concerned with categorising colour.
 
These appear to operate largely independently of each other, to have different neural bases and to be quite different kinds of process.
 
Minds are more complex than you think.
 
 %--- end paste
%--------------- 
 





\bibliography{$HOME/endnote/phd_biblio}



\end{document}