 %!TEX TS-program = xelatex
%!TEX encoding = UTF-8 Unicode

%\def \papersize {a5paper}
\def \papersize {a4paper}
%\def \papersize {letterpaper}

%\documentclass[14pt,\papersize]{extarticle}
\documentclass[12pt,\papersize]{extarticle}
% extarticle is like article but can handle 8pt, 9pt, 10pt, 11pt, 12pt, 14pt, 17pt, and 20pt text

\def \ititle {Origins of Mind: Lecture Notes}
\def \isubtitle {Lecture 01}
%comment some of the following out depending on whether anonymous
\def \iauthor {Stephen A.\ Butterfill}
\def \iemail{s.butterfill@warwick.ac.uk% \& corrado.sinigaglia@unimi.it
}
%\def \iauthor {}
%\def \iemail{}
%\date{}

%\input{$HOME/Documents/submissions/preamble_steve_paper4}
\input{$HOME/Documents/submissions/preamble_steve_lecture_notes}

%no indent, space between paragraphs
\usepackage{parskip}

%comment these out if not anonymous:
%\author{}
%\date{}

%for e reader version: small margins
% (remove all for paper!)
%\geometry{headsep=2em} %keep running header away from text
%\geometry{footskip=1.5cm} %keep page numbers away from text
%\geometry{top=1cm} %increase to 3.5 if use header
%\geometry{bottom=2cm} %increase to 3.5 if use header
%\geometry{left=1cm} %increase to 3.5 if use header
%\geometry{right=1cm} %increase to 3.5 if use header

% disables chapter, section and subsection numbering
\setcounter{secnumdepth}{-1}

%avoid overhang
\tolerance=5000

%\setromanfont[Mapping=tex-text]{Sabon LT Std}


%for putting citations into main text (for reading):
% use bibentry command
% nb this doesn’t work with mynewapa style; use apalike for \bibliographystyle
% nb2: use \nobibliography to introduce the readings
\usepackage{bibentry}

%screws up word count for some reason:
%\bibliographystyle{$HOME/Documents/submissions/mynewapa}
\bibliographystyle{apalike}


\begin{document}



\setlength\footnotesep{1em}






%---------------
%--- start paste


\title {Origins of Mind \\ Lecture 04}



\maketitle

\subsection{slide-3}
The question for this course is ...
Our current question is about physical objects.
How do humans first come to know simple facts about particular physical objects?

In attempting to answer this question, we are focussing on the abilities of
infants in the first six months of life.

What have we found so far? ...

\subsection{slide-5}
Recall the puzzle ....

\subsection{slide-6}
Berthier et al, Where’s the ball reaching study

\subsection{slide-8}
More than two decades of research strongly supports the view that
infants fail to search for objects hidden behind barriers or screens
until around eight months of age \citep[p.\ 202]{Meltzoff:1998wp} or
maybe even later \citep{moore:2008_factors}.
Researchers have carefully controlled for the possibility that infants’
failures to search are due to extraneous demands on memory or the
control of action.
We must therefore conclude, I think, that four- and five-month-old
infants do not have beliefs about the locations of briefly occluded
objects.
It is the absence of belief that explains their failures to search.

\subsection{slide-10}
Here’s what we’ve found so far.

We examined how three requirements on having knowledge of physical objects are met.
Knowledge of objects depends on abilities to (i) segment objects, (ii) represent them as
persisting and (iii) track their interactions.
To know simple facts about particular physical objects you need, minimally,
to meet these three requirements.

The second discovery concerned how infants meet these three requirements this.

\subsection{slide-11}
The second was that a single set of principles is formally adequate to
explain how someone could meet these requirements, and to describe
infants' abilities with segmentation, representing objects as persisting
and tracking objects' interactions.

This is exciting in several ways.
\begin{enumerate}
\item That infants have all of these abilities.
\item That their abilities are relatively sophisticated: it doesn’t seem
that we can characterise them as involving simple heuristics or relying
merely on featural information.
\item That a single set of principles underlies all three capacities.
\end{enumerate}

\subsection{slide-12}
three requirements, one set of principles: this suggests us that infants’
capacities are characterised by a model of the physical.

\subsection{slide-13}
[slide: model]
three requirements, one set of principles: this suggests us that infants’
capacities are characterised by a model of the physical (as opposed to
being a collection of unrelated capacities that only appear, but don’t
really, have anything to do with physical objects).

\subsection{slide-14}
1. How do four-month-old infants  model  physical objects?

In asking how infants
model physical objects, we are seeking to understand not how physical objects
in fact are but how they appear from the point of view of
an individual or system.

The model need not be thought of as something used by the system: it is
a tool the theorist uses in describing what the system is for and
broadly how it works.
This therefore leads us to a second question ...

\subsection{slide-15}
2. What is the relation between the model and the infants?

\subsection{slide-16}
3. What is the relation between the model and the things modelled (physical objects)?

\subsection{slide-21}
Let’s build on the simple view, extending it so that we do generate
relevant predictions.

(There were also theoretical objections: maybe we can overcome these
by extending it too.)

\subsection{unit\_665}


\section{The CLSTX Hypothesis: Object Indexes Underpin Infants’ Abilities}

\subsection{slide-25}
In adult humans,
there is a system of object indexes which enables them to track
potentially moving objects in ongoing actions such as visually tracking or
reaching for objects, and which influences how their attention is allocated
\citep{flombaum:2008_attentional}.

The leading, best defended hypothesis is that their abilities to do so
depend on a system of
object indexes like that which underpins multiple object tracking or
object-specific preview benefits
\citep{Leslie:1998zk,Scholl:1999mi,Carey:2001ue,scholl:2007_objecta}.

\subsection{slide-26}
But what is an object index?
Formally, an object index is ‘a mental token that functions as a
pointer to an object’ \citep[p.\ 11]{Leslie:1998zk}.
If you imagine using your fingers to track moving objects,
an object index is the mental counterpart of a finger \citep[p.~68]{pylyshyn:1989_role}.

Leslie et al say an object index is ‘a mental token that functions as a pointer to an
object’ \citep[p.\ 11]{Leslie:1998zk}

‘Pylyshyn’s FINST model: you have four or five indexes which can be attached to objects;
it’s a bit like having your fingers on an object: you might not know anything about the
object, but you can say where it is relative to the other objects you’re fingering.
(ms. 19-20)’ \citep{Scholl:1999mi}

The interesting thing about object indexes is that a system of object
indexes (at least one, maybe more)
appears to underpin cognitive processes which are not
strictly perceptual but also do not involve beliefs or knowledge states.
While I can’t fully explain the evidence for this claim here,
I do want to mention the two basic experimental tools that are used to
investigate the existence of, and the principles underpinning,
a system of object indexes which operates
between perception and thought ...

\subsection{slide-27}
Suppose you are shown a display involving eight stationary circles, like
this one.

\subsection{slide-28}
Four of these circles flash, indicating that you should track these circles.

\subsection{slide-29}
All eight circles now begin to move around rapidly, and keep moving unpredictably for some time.

\subsection{slide-30}
Then they stop and one of the circles flashes.
Your task is to say whether the flashing circle is one you were supposed to track.
Adults are good at this task \citep{pylyshyn:1988_tracking}, indicating that they can use at least four object indexes simultaneously.

(\emph{Aside.} That this experiment provides evidence for the existence of
a system of object indexes has been challenged.
See \citet[p.\ 59]{scholl:2009_what}:
\begin{quote}
`I suggest that what Pylyshyn’s (2004) experiments show is exactly what they intuitively
seem to show: We can keep track of the targets in MOT, but not which one is which.
[...]
all of this seems easily explained [...] by the view
that MOT is simply realized by split object-based attention to the MOT targets as a set.'
\end{quote}
It is surely right that the existence of MOT does not, all by itself,
provide support for the existence of a system of object indexes.
However, contra what Scholl seems to be suggesting here, the MOT paradigm
can be adapated to provide such evidence.
Thus, for instance, \citet{horowitz:2010_direction} show that, in a MOT paradigm, observers
can report the direction of one or two targets without advance knowledge of which
targets' directions they will be asked to report.)

\subsection{slide-31}
There is a behavioural marker of object-indexes called the object-specific preview benefit.
Suppose that you are shown an array of two objects, as depicted here.
At the start a letter appears briefly on each object.
(It is not important that letters are used; in theory, any readily
distinguishable features should work.)

\subsection{slide-32}
The objects now start moving.

\subsection{slide-33}
At the end of the task, a letter appears on one of the objects.
Your task is to say whether this letter is one of the letters that appeared at the start or whether it is a new letter.
Consider just those cases in which the answer is yes: the letter at the end is one of those which you saw at the start.
Of interest is how long this takes you to respond in two cases: when the letter appears on the same object at the start and end, and, in contrast, when the letter appears on one object at the start and a different object at the end.
It turns out that most people can answer the question more quickly in the first case.
That is, they are faster when a letter appears on the same object twice than when it appears on two different objects
\citep{Kahneman:1992xt}.
This difference in response times is the
% $glossary: object-specific preview benefit
\emph{object-specific preview benefit}.
Its existence shows that, in this task, you are keeping track of which object is which as they move.
This is why the existence of an object-specific preview benefit is taken to be evidence that object indexes exist.

The \emph{object-specific preview benefit} is the reduction in time needed to identify that a
letter (or other feature) matches a target presented earlier when the letter and target both
appear on the same object rather than on different objects.

\subsection{slide-34}
To see the need for principles,
return to the old-fashioned logistician who is keeping track of supply trucks.
In doing this she has only quite limited information to go on.
She receives sporadic reports that a supply truck has been sighted at one or another location.
But these reports do not specify which supply truck is at that location.
She must therefore work out which pin to move to the newly reported location.
In doing this she might rely on assumptions about the trucks’ movements being constrained to trace continuous paths, and about the direction and speed of the trucks typically remaining constant.
These assumptions allow her to use the sporadic reports that some truck or other is there in forming views about the routes a particular truck has taken.
A system of object indexes faces the same problem when the indexed objects are not continuously perceptible.
What assumptions or principles are used to determine whether this object at time $t_1$ and that object at time $t_2$ have the same object index pinned to them?

\subsection{slide-35}
[object indexes and segmentation: ducks picture]
Is one object index assigned or two?
Assigning object indexes requires segmentation.

\subsection{slide-36}
[object indexes and segmentation: partially occluded stick]

\subsection{slide-37}
Consider a stick moving behind a screen, so that the middle part of it is occluded.
Assigning one index even though there is no information about continuity of surfaces
may depend on analysis of motion.

\subsection{slide-38}
[object indexes and representing occluded objects]

[Here we’re interested in the issue rather than the details: the point is just that
continuity of motion is important for assigning and maintaining object indexes.]

Suppose object indexes are being used in tracking four or more objects simultaneously and one of these objects—call it the \emph{first object}—disappears behind a barrier.
Later two objects appear from behind the barrier, one on the far side of the barrier (call this the \emph{far object}) and one close to the point where the object disappeared (call this the \emph{near object}).
If the system of object indexes relies on assumptions about speed and direction of movement, then the first object and the far object should be assigned the same object index.
But this is not what typically happens.
Instead it is likely that the first object and the near object are assigned the same object index.%
\footnote{
See \citet{franconeri:2012_simple}.  Note that this corrects an earlier argument for a contrary view \citep{scholl:1999_tracking}.
}
If this were what always happened, then we could not fully explain how infants represent objects as persisting by appeal to object indexes because, at least in some cases, infants do use assumptions about speed and direction in interpolating the locations of briefly unperceived objects.
There would be a discrepancy between the Principles of Object Perception which characterise how infants represent objects as persisting and the principles that describe how object indexes work.

But this is not the whole story about object indexes.
It turns out that object indexes behave differently when just one object is being tracked and the object-specific preview benefit is used to detect them.
In this case it seems that assumptions about continuity and constancy in speed and direction do play a role in determining whether an object at $t_1$ and an object at $t_2$ are assigned the same object indexes \citep{flombaum:2006_temporal,mitroff:2007_space}.
In the terms introduced in the previous paragraph, in this case where just one object is being tracked, the first object and the far object are assigned the same object index.
This suggests that the principles which govern object indexes may match the principles which characterise how infants represent objects as persisting.

\subsection{slide-39}
[object indexes and representing causal interactions]

\subsection{slide-40}
The Principles of Object Perception are the key to specifying one way
of meeting these three requirements.

This suggests that, maybe,
The Principles of Object Perception which characterise infants’ abilities to track
physical objects also characterise the operations of a system of object indexes.
\subsection{slide-41}
\emph{The CLSTX conjecture}
Infants’ abilities concerning physical objects are
characterised by the Principles of Object Perception because infants’ abilities
are a consequence of the operations of a system of object indexes
\citep{Leslie:1998zk,Scholl:1999mi,Carey:2001ue,scholl:2007_objecta}.

(‘CLSTX’ stands for Carey-Leslie-Scholl-Tremoulet-Xu \citep[see][]{Leslie:1998zk,Scholl:1999mi,Carey:2001ue,scholl:2007_objecta})
Their upshot is not knowledge about particular objects and their movements but rather a
perceptual representation involving an object index.

One reason the hypothesis seems like a good bet is that object
indexes are the kind of thing which could in principle explain
infants’ abilities to track unperceived objects because object indexes
can, within limits, survive occlusion.

Note that the CLSTX conjecture assumes that
the Principles of Object Perception which characterise infants’ abilities to track
physical objects also characterise the operations of a system of object indexes.

:t
            reflecting on \citet{mccurry:2009_beyond} in one of the seminars ... distinguished
            initiating action and continuing to perform an action ... object indexes support
            guidance of action but not its initiation.

This amazing discovery is going to take us a while to fully digest.  As a first step, note its
significance for Davidson's challenge about characterising what is going on in the head of the
child who has a few words, or even no words.

\footnote{\label{fn:mot_proximity}
The findings cited in this paragraph all involve measuring object-specific preview benefits.
Some researchers have argued that in multiple object tracking with at least four objects,
motion information is not used to update indexes during the occlusion of the corresponding objects \citep{keane:2006_motion,horowitz:2006_how}; rather, `MOT through occlusion seems to rely on a simple heuristic based only on the proximity of reappearance locations to the objects’ last known preocclusion locations' (\citealp{franconeri:2012_simple}, p.\ 700).
However information about motion is sometimes available \citep{horowitz:2010_direction} and used in tracking multiple objects simultaneously \citep{howe:2012_motion, clair:2012_phd}.
One possibility is that, in tracking four objects simultaneously, motion information can be used to distinguish targets from distractors but not to predict the future positions of objects \citep[p.\ 8]{howe:2012_motion}.
}

\subsection{slide-42}
2. What is the relation between the model and the infants?

\subsection{slide-46}
behavioural: OSPB-like-effect (Richardson \& Kirkham; note their caveats); neural Kaufmann, Csibra
et al

If we consider six-month-olds, we can also find behavioural markers
of object indexes in infants \citep{richardson:2004_multimodal} ...

\subsection{slide-47}
... and there are is also a report of neural markers too \citep{kaufman:2005_oscillatory}.

\subsection{slide-48}
(\citet{kaufman:2005_oscillatory} measured brain activity in
six-month-olds infants as they observed a display typical of an object
disappearing behind a barrier.
(EEG gama oscillation over right temporal cortex)
They found the pattern of brain activity characteristic of maintaining
an object index.
This suggests that in infants, as in adults, object indexes can attach
to objects that are briefly unperceived.)

\subsection{slide-49}
The evidence we have so far gets us as far as saying, in effect, that someone capable of
committing a murder was in the right place at the right time.
Can we go beyond such circumstantial evidence?

\subsection{slide-50}
The key to doing this is to exploit signature limits.

A \emph{{signature limit} of a system} is a pattern of behaviour the system exhibits which is
both defective given what the system is for and peculiar to that system.

\citet{carey:2009_origin} argues that what I am calling the signature
limits of object indexes in adults are related to signature limits on
infants’ abilities to track briefly occluded objects.

\subsection{slide-51}
To illustrate, a moment ago I mentioned that one signature limit of
object indexes is that featural information sometimes fails to influence how objects are assigned in ways that seem quite dramatic.

\subsection{slide-52}
There is evidence that, similarly, even 10-month-olds will sometimes
ignore featural information in tracking occluded objects
\citep{xu:1996_infants}.%
\footnote{
This argument is complicated by evidence that infants around 10 months of age do not always fail
to use featural information appropriately in representing objects as persisting
\citep{wilcox:2002_infants}.
In fact \citet{mccurry:2009_beyond} report evidence that even five-month-olds can make use of
featural information in representing objects as persisting \citep[see also][]{wilcox:1999_object}.
%they use a fringe and a reaching paradigm.  NB the reaching is a problem for the simple interpretation of looking vs reaching!
% NB: I think they are tapping into motor representations of affordances.
Likewise, object indexes are not always updated in ways that amount to ignoring featural
information \citep{hollingworth:2009_object,moore:2010_features}.
It remains to be seen whether there is really an exact match between the signature limit on
object indexes and the signature limit on four-month-olds’ abilities to represent objects as
persisting.
The hypothesis under consideration---that infants’ abilities
to track briefly occluded objects depend on a system of
object indexes like that which underpins multiple object tracking or
object-specific preview benefits---is a bet on the match being exact.
}
\subsection{slide-54}
\emph{The CLSTX conjecture}
Infants’ abilities concerning physical objects are
characterised by the Principles of Object Perception because infants’ abilities
are a consequence of the operations of a system of object indexes
\citep{Leslie:1998zk,Scholl:1999mi,Carey:2001ue,scholl:2007_objecta}.

(‘CLSTX’ stands for Carey-Leslie-Scholl-Tremoulet-Xu \citep[see][]{Leslie:1998zk,Scholl:1999mi,Carey:2001ue,scholl:2007_objecta})

While I wouldn’t want to suggest that the evidence on siganture limits is decisive, I think it
does motivate considering the hypothesis and its consequences. In what follows I will assume the
hypothesis is true: infants’ abilities to track briefly occluded objects depend on a system of
object indexes.

\subsection{slide-55}
How does help us with the puzzles?

\subsection{slide-56}
Object indexes can survive occlusion ...

\subsection{slide-57}
... but not the endarkening of a scence

\subsection{slide-58}
But why do we get the opposite pattern with search measures?

\subsection{slide-60}
The hypothesis has an advantage which I don’t think is widely
recognised.
This is that object indexes are independent of beliefs and knowledge
states.
Having an object index pointing to a location is not the same thing
as believing that an object is there.
And nor is having an object index pointing to a series of locations over time
is the same thing as believing or knowing that these locations
are points on the path of a single object.
Further, the assignments of object indexes do not invariably give rise
to beliefs and need not match your beliefs.

To emphasise this point, consider once more this scenario
in which a patterned square disappears behind the barrier; later a
plain black ring emerges.  You probably don't believe that they are
the same object, but they probably do get assigned the same object index.
Your beliefs and assignments of object indexes are inconsistent in this
sense: the world cannot be such that both are correct.

\subsection{slide-62}
So this is a virtue of the hypothesis that four- and five-month-old
infants’ abilities
to track briefly occluded objects depend on a system of
object indexes.
Since assignments of object indexes do not entail the existence of
corresponding beliefs,
the fact that infants of this age systematically
fail to search for briefly occluded objects is not an objection to the
hypothesis.

\subsection{slide-63}
So why do 5 month olds fail to manifest their ability to track briefly
occluded objects by initiating searches for them after they have been
fully occluded?

\subsection{slide-64}
Because object indexes are independent of beliefs
and do not by themselves support the initiation of action.
Further, I guess that occlusion interferes
with motor representations  of objects in infants because occlusion
involves two objects, one in front of the other.

\subsection{slide-65}
But we still have to explain this ...

Why do infants succeed in searching for momentarily endarkend objects?
Because they can represent objects motorically, and endarkening does
not immediately interfere with such representations.
What does this mean?

\subsection{slide-68}
Object indexes survive occlusion but not endarkening;
motor representations survive endarkening but not occlusion.

\subsection{slide-69}
Why do 5 month old infants reach towards the cloth screen more often when a cube
goes in and a circle comes out than when a cube goes in and a cube comes out?

Here's the authors' description of their procedure.
'Once the ball came to rest at the right edge of the platform, the platform was pushed forward
until the edge of the platform was directly in front of, and within easy reach of, the infant. In
the second phase, the infant was allowed to search for 20 s. ' \citep{mccurry:2009_beyond}

What should we predict?
Cloth screen does not prevent action,\footnote{
‘In the first familiarization trial, infants were shown the fringed-screen and were encouraged to
reach through the fringe. If necessary, the experimenter gently guided the infant’s hand through
the fringed-screen. Once the infant placed his or her hand through the fringed-screen twice, the
trial ended.’
}
so reaching should be possible.
Further, if motor representations are responsible for the effect, the fact that
the experiment requires sensitivity to featural information should not be an issue.
(Further, a version of this task using violation of expectations may fail
because featural information is critical.)
And although these are very far from the terms in which they interpret their findings,
this is exactly what McCurry et al 2009 found.

\subsection{slide-71}
Object indexes survive occlusion but not endarkening;
motor representations survive endarkening but not occlusion.
\subsection{slide-73}
\emph{The CLSTX conjecture}
Infants’ abilities concerning physical objects are
characterised by the Principles of Object Perception because infants’ abilities
are a consequence of the operations of a system of object indexes
\citep{Leslie:1998zk,Scholl:1999mi,Carey:2001ue,scholl:2007_objecta}.

(‘CLSTX’ stands for Carey-Leslie-Scholl-Tremoulet-Xu \citep[see][]{Leslie:1998zk,Scholl:1999mi,Carey:2001ue,scholl:2007_objecta})

\subsection{slide-74}
Return to this amazing discovery.

2. What is the relation between the model and the infants?

\subsection{unit\_669}


\section{Core Knowledge vs Object Indexes}

Consider the conjecture that infants’ abilities concerning physical objects are
characterised by the Principles of Object Perception because infants’ abilities
are a consequence of the operations of a system of object indexes.
If this conjecture is true, should we reject the claim that infants have a core
system for physical objects?
Or does having a system of object indexes whose operations are characterised by the Principles of
Object Perception amount to having core knowledge of those principles?

\emph{Outstanding problem}
Since having core knowledge of objects does not imply having knowledge knowledge of objects, how
can the emergence in development of knowledge of simple facts about particular physical objects be
explained?
What is the role of core knowledge of objects, and what other factors might be involved?

\subsection{slide-76}
Let’s consider some consequences of the CLSTX conjecture.
\emph{The CLSTX conjecture}
Infants’ abilities concerning physical objects are
characterised by the Principles of Object Perception because infants’ abilities
are a consequence of the operations of a system of object indexes
\citep{Leslie:1998zk,Scholl:1999mi,Carey:2001ue,scholl:2007_objecta}.

(‘CLSTX’ stands for Carey-Leslie-Scholl-Tremoulet-Xu \citep[see][]{Leslie:1998zk,Scholl:1999mi,Carey:2001ue,scholl:2007_objecta})

\subsection{slide-77}
We saw this quote in the first lecture ...

\subsection{slide-78}
Actually we don’t lack a way of describing what is in between.
We already have it.
We were simply not aware of it because we hadn’t thought carefully enough
about the representations and processes involved in perception and action.

The discovery that the principles of object perception characterise the operation of
object-indexes doesn't mean we have met the challenge exactly.
We haven't found a way of describing the processes and representations that underpin infants'
abilities to deal with objects and causes.
However, we have reduced the problem of doing this to the problem of characterising how
some perceptual mechanisms work.
And this shows, importantly, that understanding infants' minds is not something different from
understanding adults' minds, contrary to what Davidson assumes.
The problem is not that their cognition is half-formed or in an intermediate state.
The problem is just that understanding perception requires science and not just intuition.

\subsection{slide-79}
What is the relation between infants' competencies with objects and adults'?
Is it that infants' competencies grow into more sophisticated adult competencies?
Or is it that they remain constant throught development, and are supplemented by quite
separate abilities?

\subsection{slide-83}
The identification of the Principles of Object Perception with object-indexes suggests that
infants' abilities are constant throughout development.
They do not become adult conceptual abilities; rather they remain as perceptual systems
that somehow underlie later-developing abilities to acquire knowledge.

Confirmation for this view comes from considering that there are discrepancies in adults'
performances which resemble the discrepancies in infants between looking and action-based
measures of competence ...
[This links to unit 271 on perceptual expectations ...]

\subsection{slide-85}
Which of these features are features of a system of object indexes?

\subsection{slide-92}
Left half:
The Core Knowledge View



Infants, like most adults, do not know the principles of object perception; but they have core knowledege of them.


The CLSTX conjecture



The principles of object perception characterise how a system of object indexes should work.



Infants’ (and adults’) object indexes track objects through occlusion.



Five-month-olds do not know the location of an occluded object.



Five-month-olds do have perceptual expectations concerning its location.


‘CLSTX’ stands for Carey-Leslie-Scholl-Tremoulet-Xu \citep[see][]{Leslie:1998zk,Scholl:1999mi,Carey:2001ue,scholl:2007_objecta}

Are these two views compatible?
I think we had better characterise core knowledge in such a way that they
turn out to be true!

\subsection{slide-93}
Our Next Big Problem is this.
We've said that infants' competence with causes and objects is not knowledge but something
more primitive than knowledge, something which exists in adults too and can carry information
discrepant with what they know.
So, if at all, how does appealing to these early capacities enable us to explain the origins
of knowledge?

\subsection{slide-94}
Berthier et al, Where’s the ball reaching study

Further issue ... we now have a more complicated story about the emergence of knowledge

\subsection{slide-95}
Broadly, my suggestion will be that the competence which appears in the first months of
development leads to knowledge of objects and causes only in conjunction with various
additional things, like social interaction, perhaps language and abilities to use tools.

The picture I want to offer differs from those of researchers like Vygotsky and Tomasello in
that there is an essential role for early-developing forms of representation that are more
primitive that concepts or thoughts and do not appear to have any kind of social origin.

But the picture also differs from those of researchers like Spelke and Carey in that these
early developing forms of representation are only one of several components that are needed
to understand the origins of knowledge.

To explore this idea I want to switch to a completely different domain, colour.

[*Aside on tool use:]
Basic forms of tool use may not require understanding how objects interact
(Barrett, Davis, \& Needham; Lockman, 2000), and may depend on core cognition of
contact-mechanics (Goldenberg \& Hagmann, 1998; Johnson-Frey, 2004).
Experience of tool use may in turn assist children in understanding notions of manipulation,
a key causal notion (Menzies \& Price, 1993; Woodward, 2003). Perhaps non-core capacities for
causal representation are not innate but originate with experiences of tool use.

\subsection{unit\_667}


\section{Phenomenal Expectations Connect Object Indexes to Looking Behaviours}

\subsection{slide-97}
So those who, like me, are impressed by the evidence for the hypothesis
that four- and five-month-olds’ abilities to track occluded objects
are underpinned by the operations of a system of object indexes
are left with a question.
The question is, What links the operations of object indexes to
patterns in looking duration?

I’ve just argued that it can’t be beliefs or knowledge states.
So far we’ve been assuming it is object indexes.  But this
assumption is not quite right ...

\subsection{slide-98}
This is an important question for me so I want to pause to emphasise it.
This question is, What can the operations of a system of
object indexes explain?

\subsection{slide-99}
Functions of object indexes:



✔ influence how attention is allocated



✔ guide ongoing actions (e.g. visual tracking, reaching)



✘ initiate purposive actions


If these are the functions of object indexes, how can we explain
looking times in habituation or violation-of-expectation experiments?

The primary functions of object indexes include influencing the allocation
of attention and perhaps guiding ongoing action.
If this is right, it may be possible to explain anticipatory looking
directly by appeal to the operations of  object indexes.
But the operations of object indexes cannot directly explain differences
in how novel things are to an infant.
And nor can the operations of object indexes directly explain why infants
look longer at stimuli involving discrepancies in the physical behaviour
of objects.

\subsection{slide-100}
Consider this case where a ball falls and lands on a bench.
Suppose that there was a barrier in front of the bench, like the dotted line.
Because the bench protrudes from the barrier, you could easily see where the ball will land.
But of course you can only see this if you know that barriers stop solid balls.
Spelke used this observation to provide evidence that 4-month-old infants can track objects' causal interactions.

Infants were habituated to a display in which a ball fell behind a screen,
the screen came forwards and the ball was revealed to be on the ground,
just where you'd expect it to be.
After habituation infants were shown one of two displays.
Infants in the 'consistent group' were shown this.

Whereas infants in the 'inconsistent group' were shown this.

What should we predict?
If infants were only paying attention to the shapes and ignoring properties
like solid, they should have dishabituated more to the consistent than to
the inconsistent stimlus.
After all, that stimlus is more different from the habituation stimulus in
terms of the surfaces.
But if infants were are to track some simple causal interactions, then they
might dishabituate to the 'inconsistent' stimulus more than to the
'consistent' stimulus because that one involves an apparent violation of a
physical laws.

\subsection{slide-101}
Here are the results.

(Recall that the subjects are 4-month-old infants.)

Why do infants manifest an ability to track briefly occluded objects on
habituation and violoation-of-expectations tasks?
Ah ... just here we face a significant challenge ...

As I said earlier, infants’ abilities to track briefly
occluded objects are manifested in several different ways.
They are manifested in (iii) anticipatory looking, (ii) reactions
indicating the violation of an expectation, and (i) dishabituation
indicating interest in certain stimuli.

Can all of these behaviours be explained merely by invoking object
indexes?

We know that infants are likely to maintain object indexes for the object
while it is they are occluded.
Accordingly, when the screen drops in the condition labelled
‘impossible outcome’, there is an interruption to the normal
operation of object indexes: infants have assigned two object indexes
but there is only one object.
But why does this cause infants to look longer at in the
‘impossible outcome’ condition than in the ‘possible outcome’ condition?
\textbf{How does a difference in operations involving object indexes result
in a difference in looking times?}

\subsection{slide-102}
Recall this situation.
Suppose you have seen it a hundred times before, so you know just what
to expect.
Still, the tendancy to expect two objects is on some level barely
diminished, and event in which a single object is revealled
is liable to feel magical in some small way.
This feeling of magic is a phenomenal expectation.

Let me give you two more illustrations [the wire and the face]. ...

--------
\subsection{slide-104}
Now I want to suggest that it is something called a phenomenal expectation.

Before I explain what phenomenal expectations are, let me illustrate the
idea informally.

\subsection{slide-105}
What is a phenomenal expectation?  Consider a second illustration.

Here is a wire.
Contrast two sensory encounters with this wire. In the first you visually
experience the wire as having a certain shape. In the second you receive an
electric shock from the wire without seeing or touching it.%
\footnote{This illustration is borrowed from Campbell (2002: 133–4); I use it to support a claim weaker than his.}
The first sensory encounter involves perceptual experience as of a property of
the wire whereas, intuitively, the second does not.
I take this intuition to be correct.%
\footnote{
Notice that the intuition is not that the shock involves no
perceptual experience at all, only that the shock does not involve
perceptual experience as of any property of the wire. Notice also that the
intuition concerns what a perceptual experience is as of, and not directly
what is represented in perception. The relation between these two is
arguably not straightforward (compare, e.g., \citet[p.~28]{Shoemaker:1994el} or
\citet[pp.~50--2]{Chalmers:2006xq} on distinguishing representational from
phenomenal content).
}

The intuition is potentially revealing because the electric shock involves
rich phenomenology, and its particular phenomenal character depends in part
on properties of its cause (changes in the strength of the electric current
would have resulted in an encounter with different phenomenal character).
So there are sensory encounters which, despite having phenomenal characters
that depend in part on which properties are encountered, are not perceptual
experiences as of those properties.

\subsection{slide-106}
What is a phenomenal expectation?  Consider a third (and final) illustration.

Here is a face that I hope will seems familiar to most people.
When you see this face, you have a feeling of familiarity.
This feeling of familiarity is not just a matter of belief:
even if you know for sure that you have never encountered the person
depicted here (and trust me, you haven’t), the feeling of familiarity
will persist.
Nor is the feeling a matter of perceptual experience: you can’t
perceptually experience familiarity
any more than you can perceptually experience electricity.

(The face is a composite of Bush and Obama.  It is chosen to illustrate that
the feeling of familiarity is not a consequence of how familiar things
actually are; instead it may be a consequnece of
the degree of fluency with which unconscious processes can identify
perceived items \citep{Whittlesea:1993xk,Whittlesea:1998qj}.
Learning a grammar can also generate feelings of familiarity.
Subjects who have implicitly learned an artificial grammar report feelings
of familiarity when they encounter novel stimuli that are part of
the learnt grammar \citep{scott:2008_familiarity}.
They are also not doomed to treat feelings of familiarity as being
about actual familiarity:
instead subjects can use feeling of
familiarity in deciding whether a stimulus is from that grammar
\citep{Wan:2008_familiarity}.)

I could go on to mention the feeling you have when someone’s eyes are
boring into your back, or the feeling that a name is on the tip of your
tongue.
But let me focus just on the feelings associated with electricity and
with familiarity.
These feelings are paradigm cases of phenomenal expectation.

\subsection{slide-107}
All three examples (the feelings of magic, of electricity and of
familiarity) show that:

There are aspects of the overall phenomenal character of experiences which their subjects take to be informative about things that are only distantly related (if at all) to the things that those experiences intentionally relate the subject to.

To illustrate, having a feeling of familiarity is not a matter of standing
in any
intentional relation to the property of familiarity, but it is something
that we can interpret as informative about famility.

\subsection{slide-108}
Phenomenal expectations are these aspects of experience.

\subsection{slide-109}
Phenomenal expectations can be thought of as sensations in approximately
Reid’s sense.%
\footnote{
\citet{Reid:1785cj,Reid:1785nz}.
Even if you don’t believe that there are sensations in Reid’s sense,
thinking of phenomenal expectations as if they were sensations will
serve to illustrate their characteristic features.
The main points that follow are consistent with several different ways of
thinking about phenomenal expectations.
For instance, you might take the view that
what I am calling phenomenal expectations are
perceptual experiences of the body or of bodily reactions,
or that they involve some kind of cognitive phenomenology.
The essential claim is just that the phenomenal expectations associated with
the operations of object indexes are not constituted by states which involve
intentional relations to any of the things which are assigned an object index.
}

\subsection{slide-110}
Sensations are:
\begin{enumerate}
\item monadic properties of events, specifically perceptual experiences,
\item individuated by their normal causes% %{Tye, 1984 #1744@204}
---in the case of feelings of familiarity, its normal cause is ease of processing
\item which alter the overall phenomenal character of those experiences
\item in ways not determined by the experiences’ contents
(so two perceptual experiences can have the same content while one has a sensational property which the other lacks).
\end{enumerate}

\subsection{slide-111}
An important consequence is that phenomenal expectations can lead to beliefs
only via associations or further beliefs.
They are signs which need to be interpreted by their subjects
(\citealp[Essay~II, Chap.~16, p.~228]{Reid:1785cj}
\citealp[Chap.~VI sect.~III, pp.~164–5]{Reid:1785nz}).
Let me explain.

As a scientist, you can pick out the feeling of familiarity as that
phenomenal expectation which is normally caused by the degree to which
certain processes are fluent.
But as the subject of who has that phenomenal expectation, you do not
necessarily know what its typical causes are.
This is something you have to work out in whatever ways you work out
the causes of any other type of event.

(Contrast phenomenal expectations with perceptual experiences.
Having
a perceptual experience of, say, a wire’s shape, involves standing
in an intentional relation to the wire’s shape; and the phenomenal
character of this perceptual experience is specified by this
intentional relation.%
\footnote{
Compare \citet[p.~380]{Martin:2002yx}:
‘I attend to what it is like for me to inspect the lavender bush through
perceptually attending to the bush itself.’
And \citet[p.~211]{byrne:2001_intentionalism}
‘subject can only discover the phenomenal character of her experience by
attending to the world ... as her experience represents it.’
}
Such perceptual experiences are often held to reveal the wire’s shape to the
subject and so lead directly to beliefs.%
\footnote{
Compare \citet[p.~222]{Johnston:1992zb}:
‘[j]ustified belief … is available simply on the basis of visual perception’;
\citet[p.~143–4]{Tye:1995oa}:
‘Phenomenal character “stands ready … to make a direct impact on beliefs’;
and
\citet[p.~291]{Smith:2001iz}:
‘[p]erceptual experiences are … intrinsically … belief-inducing.’
})

(By contrast, having a phenomenal expectation concerning familiarity or an
physical object’s path does not involve standing in any intentional relation
to these things.
The phenomenal expectation is individuated by its normal causes, rather
than by any intentional relation.
And a phenomenal expectation leads to belief, if at all, only indirectly.
For learning is required in order for the subject to come to a view on
what tends to cause the phenomenal expectation.)

Phenomenal expectations have been quite widely neglected in philosophy and
developmental psychology.
They are a means by which cognitive processes enable perceivers to
acquire dispositions to form beliefs about objects’ properties which are
reliably true.
Phenomenal expectations provide a low-cost but efficient bridge between
non-conscious cognitive processes and conscious reasoning.

\subsection{slide-112}
So my question was how the operations of object indexes might
explain patterns of looking duration in habituation and
violation-of-expectation experiments.
My guess is that some operations of object indexes give rise to
phenomenal expectations, which in turn influence looking durations.

\subsection{slide-113}
\section{Development Is Rediscovery}

This guess gives rise to a further question (which I want to articulate
but won’t attempt to answer).
In asking how the operations of object indexes might give rise to
patterns in looking duration, we have been concerned with what happens a
short interval of time.
But the guess about phenomenal expectations raises a question
about the course of development in the first months or years of life.
Let me explain.

\subsection{slide-114}
In the beginning Spelke and others conjectured that infants’
abilities to track briefly occluded objects were a consequence of their
having core knowledge for objects.
This conjecture is related to the later hypothesis about object indexes.
The idea is that we can further specify the mechanisms that realise
infants’ core knowledge of physical objects by identifying it with
two things: a system of object indexes and a system capable of representing
physical objects motorically.

\subsection{slide-115}
There was always a question about how infants’ core knowledge about
objects might explain the emergence of knowledge knowledge
(that is, knowledge proper) about objects.
Now this question becomes, What is the role of a system of object
indexes in the emergence in development of knowledge of physical
objects?
In short, How do you get from object indexes to knowledge?

Answers to these questions typically assume that core knowledge
provides a conceptual identification of objects and some of their
properties such as location or size,
or else that it involves standing in some kind of intentional
relation to these things.
This is true of
Spelke’s suggestion that mature understanding of objects, number,
and mind derives from core knowledge by virtue of core knowledge
representations being assembled \citep{Spelke:2000nf};
claims by Leslie and others
that modules provide conceptual identifications of their inputs
\citep{Leslie:1988ct};
Karmiloff-Smith’s representational re-description
\citep{Karmiloff-Smith:1992lv};
and Mandler’s claim
that ‘the earliest conceptual functioning consists of a redescription
of perceptual structure’ \citep{Mandler:1992vn}.

\subsection{slide-116}
But recall the guess about phenomenal expectations linking object
indexes to patterns of looking duration.
If this guess is right, then it is not true that
core knowledge provides a conceptual identification of objects.
And it is not true that having core knowledge involves standing in any
kind of intentional relation to objects and their properties.
This makes the question about development particularly
difficult to answer.

It means that rather than assembing or redescribing representations,
development must be a process of rediscovery.

The step from phenomenal expectations to knowledge is like the step
from feeling electric shocks to understanding electricity.
So coming to know simple facts about particular physical
objects may begin with object indexes and the
phenomenal expectations these give rise to, but it does not end there.
Interpreting the phenomenal expectations may involve interacting with
objects, learning to use tools, and perhaps interacting with others
and objects simultaneously.

Coming to know facts about physical objects is a matter of
rediscovering things already implicit in a system of object indexes.
Some might object that development can’t require such rediscovery
because it would be hopelessly inefficient to require things already
encoded to be learnt anew.
But rediscovery is an elegant solution to a practical problem.
If you are building a survival system you want quick and dirty
heuristics that are good enough to keep it alive: you don’t
necessarily care about the truth.
If, by contrast, you are building a thinker, you want her to be
able to think things that are true irrespective of their survival value.
This cuts two ways.
On the one hand, you want the thinker’s thoughts not to be
constrained by heuristics that ensure her survival.
On the other hand, in allowing the thinker freedom to pursue the
truth there is an excellent chance she will end up profoundly
mistaken %(Malebranche?)
or deeply confused %(Hegel?)
about the nature of physical objects.
So you don’t want thought contaminated by survival heuristics and you
don’t want survival heuristics contaminated by thought.  Or, even if
some contamination is inevitable, you want to limit it.
%So you want inferential isolation.
This combination is beautifully achieved by giving your thinker a
system or some systems for tracking objects and their interactions
which appear early in development, and also a mind which allows her
to acquire knowledge of physical objects gradually over months or
years, taking advantage of interactions with objects as well as
social interactions about objects—providing, of course, that the
two are not directly connected but rather linked only very loosely,
via phenomenal expectations.

\subsection{slide-118}
\section{Conclusion}

To conclude,
I started by mentioning the wide variety of evidence that
four- and five-month-olds can track briefly occluded objects.
This evidence raises the question, How do infants do that?
On the leading, best supported hypothesis,
four- and five-month-olds’ abilities to track briefly occluded objects
depend on a system of object indexes like that which underpins multiple
object tracking or object-specific preview benefits.
This hypothesis
also has the virtue of being consistent with the most straightforward
explanation of why infants of this age (four- to five-months) and even older
systematically fail to manually search for occluded objects.
(The explanation is that they lack beliefs about the locations of objects.)

Accepting this hypothesis forces us to confront a question.
How could the operations of object indexes explain patterns in looking
duration?
This question arises because
facts about the operations of object indexes do not themselves
straightforwardly imply anything about how things seem to infants, nor about
what they believe.

The answer, I suggested, is phenomenal expectations.
Much as there are phenomenal expectations associated with the ease or
difficulty of processing a complex stimulus like a face or letter sequence,
so also phenomenal expectations are associated with operations involving
object indexes.
These phenomenal expectations are not intentional relations to
the phyiscal objects whose behaviours normally cause them.
Instead they can be thought of as sensations in roughly Reid’s sense.
So they are monadic properties of perceptual experiences which carry
information about physical objects.

Importantly, phenomenal expectations (like sensations) require
interpretation.
In order to get from a phenomenal expectation to a belief you need to
form a view about what the phenomenal expectation is a sign of.
This requires learning, and your view can change as you learn more.

This has consequences for understanding the emergence in
development of knowledge of physical objects.
Such knowledge is probably a consequence of the (core) system of
object indexes, but on the view I have been defending the two can be only
indirectly related.
Having core knowledge of objects is a matter of having a system of object
indexes.
The system can affect what you believe or know about objects only by way
of phenomenal expectations.
Gaining knowledge proper requires interpreting the phenomenal expectations,
and so is in part a matter of rediscovering information already processed
by your core systems.

\subsection{slide-119}
The question is ...
How do humans first come to know simple facts about particular physical objects?

\subsection{slide-120}
Here’s what we’ve found so far.

We examined how three requirements on having knowledge of physical objects are met.
Knowledge of objects depends on abilities to (i) segment objects, (ii) represent them as
persisting and (iii) track their interactions.
To know simple facts about particular physical objects you need, minimally,
to meet these three requirements.

The second discovery concerned how infants meet these three requirements this.

\subsection{slide-121}
The second was that a single set of principles is formally adequate to
explain how someone could meet these requirements, and to describe
infants' abilities with segmentation, representing objects as persisting
and tracking objects' interactions.

This is exciting in several ways.
\begin{enumerate}
\item That infants have all of these abilities.
\item That their abilities are relatively sophisticated: it doesn’t seem
that we can characterise them as involving simple heuristics or relying
merely on featural information.
\item That a single set of principles underlies all three capacities.
\end{enumerate}

\subsection{slide-122}
Return to this amazing discovery.

2. What is the relation between the model and the infants?

Let me make some more points about it.

First, it works in concert with a claim about motor representation.
We neeed this claim otherwise we haven't fully explained the discrepancy between looking and
action-based measures for representing objects as persisting and tracking their causal
interactions.
After all, why do these perceptual representations of objects--the object indexes--not guide
purposive actions like reaching and pulling?
This is an issue we shall return to.

Second, it leaves us with a question we didn't have before.
What is the relation between these abilities to segment objects, represent them as persisting
and track their causal interactions and knowledge about objects?
Clearly having an object-index stuck to an object is not the same thing as having knowledge
about the object's location and movements.  (If it were, we'd face just the problems that are
fatal for the Simple View.)
What then is the relation between these things?

\subsection{slide-123}
Only phenomenal expectations



connect ‘core knowledge’ of objects



to thought.


Phenomenal expectations have been quite widely neglected in philosophy and
developmental psychology.
They are a means by which cognitive processes enable perceivers to
acquire dispositions to form beliefs about objects’ properties which are
reliably true.
Phenomenal expectations provide a low-cost but efficient bridge between
non-conscious cognitive processes and conscious reasoning.

Development is rediscovery.



If you accept my story about phenomenal expectations then
you also face a problem understanding the how the existence of core knowledge
systems can explain the emergence of knowledge in development.


If you accept my story about phenomenal expectations then
you also face a problem understanding the how the existence of core knowledge
systems can explain the emergence of knowledge in development.

\subsection{unit\_641}


\section{Syntax / Innateness}

\subsection{slide-125}
So far we have considered examples of core knowledge.  But we have ignored a paradigm case,
            one which has inspired much work on this topic (although it is not a case Spelke or Carey would recognize! *todo: stress throughout) ...

\subsection{slide-126}
Human adults have extensive knowledge of the syntax of their languages,
            as illustrated by, for example, their abilities to detect grammatical and ungramatical sentences which they have never heard before, independently of their meanings.
            To adapt a famous example from Chomsky, ...

\subsection{slide-127}
We need to task two questions.

\subsection{slide-128}
First, what is this thing, syntax, which is known?

This thing they know, the syntax, isn't plausibly just a list of which sentences are grammatical.

Because people can make judgements about arbitrarily long, entirely novel sentences.

Rather, the thing known must be something that enables people to make judgements about sentences.

We might think of it roughly as a theory of syntax.

It's like a theory in this sense: knowledge of it enables you to make judgements about the grammaticality of arbitrary sentences.

\subsection{slide-129}
The second question is, Is it *knowledge* we have syntax or something else?

There's something interesting.

The knowledge can be revealed indirectly, by asking people about whether particular sentences are grammatical.

But people can't say anything about how they know the sentence is grammatical.

It's like perceiving the shape of something: there isn't much to say about how you know.

So the theory of syntax isn't something we can discover by introspection:

we have to *rediscover* it from scratch by investigating people's linguistic abilities.

\subsection{slide-130}
Knowledge of syntax therefore seems to have some of the features associated with core knowledge.

First, it is domain-specific.

Second, it is inaccessible.  That is, it can't guide arbitrary actions.

In what follows I want to suggest that syntax provides a paradigm case for thinking about core knowledge.

\subsection{slide-131}
In addition, I want to use the case of syntax for thinking about the question, What is innate in humans?

I was astonished how many people considered this question in the unassessed essay, some people seem really fascinated by it.

But almost no one discussed the case of syntax in depth.  If you're going to talk about innateness, you really need to know a little bit about syntax.

So I'm also going to provide you with that understanding.

\subsection{slide-132}
Consider a phrase like 'the red ball'.

What is the syntactic structure of this noun phrase?

In principle there are two possibilities.

\subsection{slide-135}
How can we decide between these?

\begin{enumerate}



\item ‘red ball’ is a constituent on (b) but not on (a)


\item anaphoric pronouns can only refer to constituents



                \item In the sentence ‘I’ll play with this red ball and you can play with that one.’, the word ‘one’ is an anaphoric prononun that refers to ‘red ball’ (not just ball).
                \citep{lidz:2003_what,lidz:2004_reaffirming}.




\end{enumerate}

\subsection{slide-137}
What I've just shown you is, in effect, how we can decide whihc way an adult human understands a phrase like 'the red ball'.

We can discover this by finding out how they understand a sentence like 'I’ll play with this red ball and you can play with that one.'.

But how could we do this with infants who are incapable of discussing sentences with us?

\subsection{slide-138}
Here's how the experiment works (see \citealp{lidz:2003_what}) ...

The experiment starts with a background assumption:

‘The assumption in the preferential looking task is that infants prefer to look at an image that matches the linguistic stimulus, if one is available’ \citep{lidz:2003_what}.

\subsection{slide-139}
So the key question was whether infants would look more at the yellow bottle (which is familiar) or the blue bottle (which is novel).

If they think 'one' refers to 'bottle', we'd expect them to look longer at the blue bottle;

and conversely if they think one refers to 'yellow bottle', then they're being asked whether they see another yellow bottle.

\subsection{slide-140}
And, as always, we need a control condition to check that infants aren't looking in the ways predicted irrespective of the manipulation.

\subsection{slide-141}
And here's what they found ...

\subsection{slide-142}
What can we conclude so far?

So there is core knowledge of syntax ... or is there?

\subsection{slide-143}
Core knowledge is often characterised as innate.

I think this is a mistake (more about this later), but many of you do not.

How could we tell whether these representations are innate?

\subsection{slide-144}
What do we mean by innate here?

The easy answer is: not learned.

But I think there's a more interesting way to approach understanding what 'innate' means.

Quite a few people pointed out that there isn't agreement on what innateness is.

But this is not very interesting by itself because there's disagreement about most things and potential causes of disagreement include ignorance and stupidity.

It's also important that the mere fact that a single term is used with multiple meanings isn't an objection to anyone.

As philosophers, some of you are tempted to catalogue different possible notions of innateness.

I encourage you to resist this temptation; if you want to collect something, pick something useful like banknotes.

There's a much better way to approach things.

Let's see what kind of findings are, or would be, taken to show that something is innate.

We can use these to constrain our thinking about innateness.

We will say: assuming that this is a valid argument that X is innate, what could innateness be?

\subsection{slide-145}
Aside: we have too approach science as radical interpreters ...

How does radical interpretation work?

Interpretation is hard because there are two factors: truth and meaning.

The proposal Davidson makes is that we assume truth and infer meaning.

I'm recommending a similar strategy.

We take for granted that this argument establishes that X is innate; we then ask what innateness could be given that this is so.

‘All understanding of the speech of another involves radical interpretation’
\citep[p.\ 125]{Davidson:1973jx}

\subsection{slide-146}
The best argument for innateness is the poverty of stimulus argument.

We need to step back and understand how poverty of stimulus arguments work.

Here I'm following \citet{pullum:2002_empirical}, but I'm simplifying their presentation.

How do poverty of stimulus arguments work? See \citet{pullum:2002_empirical}.

First think of them in schematic terms ...

This is a good structure; you can use it in all sorts of cases, including the one about chicks' object permanence.

Now fill in the details ...

\subsection{slide-153}
In our case, X is knowledge of the syntactic structure of noun phrases.  (Caution: this is a simplification; see\citet[p,\ 158]{lidz:2004_reaffirming}).)

\subsection{slide-154}
This is what the Lidz et al experiment showed.

Note that no one takes this to be evidence for innateness by itself.

\subsection{slide-155}
What is the crucial evidence infants would need to learn the syntactic structure of noun phrases?

This is actually really hard to determine, and an on-going source of debate I think.

But roughly speaking it's utterances where the structure matters for the meaning, utterances like 'You play with this red ball and I'll play with that one'.

\subsection{slide-156}
\citet{lidz:2003_what} establish this by analysing a large corpus (collection) of conversation involving infants.

\subsection{slide-157}
What can we infer about innateness from this argument?

First, think about what is innate.  The fact that knowledge of X is acquired other than by data-driven learning doesn't mean that X is not innate; it just means that something which enables you to learn this is.

Second, think about the function assigned to innateness.  That which is innate is supposed to stand in for having the crucial evidence.

This, I think, is the key to thinking about what we *ought* to mean by innateness.

So attributes like being genetically specified are extraneous---they may be typical features of innate things, but they aren't central to the notion.

By contrast, that what is innate is not learned must be constitutive (otherwise that which is innate couldn't stand in for having the crucial evidence)

\subsection{slide-158}
Contrary to what many philosophers (including Stich and Fodor) will tell you ...

\subsection{slide-159}
But they wrote this before \citet{lidz:2003_what} came out.

\subsection{slide-160}
I asked you this question, but what do I think?

I'd approach it by distinguishing two sub-questions (the second of which has two sub-sub-questions)

**todo: Stress other conceptions and arguments good; start with a project from \citet{spelke:2012_core} or from \citet{haun:2010_origins} and you reach a different point!

\subsection{slide-163}
Arguments from the poverty of stimulus are the best way to establish innateness.

The argument concerning syntax we've just been discussing is quite convincing, although if you follow up on the references given in the handout you'll see it's not decisive (as always).

For things other than knowlegde of syntax, the evidence concerning humans is far less clear.

There are, however, quite good cases in nonhuman animals, as many of you know.

So it's not unreasonable to conjecture that learning in the several domains where infants appear to know things early in their first year is innately-primed rather than entirely data-driven.

But, one or two cases aside, there's enough evidence to rule out the converse conjecture.

\subsection{slide-164}
I don't think what is innate is knowledge, nor do I think it's concepts.

But I think there's a good chance that modules are innate (and therefore core knowledge if I'm right to suppose that 'core knowledge' is a term for the fundamental principles describing the operation of a module).

\subsection{slide-165}
On content: I think quite a lot is known about the modules thanks to detailed tests that have little to do directly with controversy about inateness.

\subsection{slide-166}
Why care about whether something is innate?  (This isn't suppose to be dismissive.)

\subsection{slide-167}
Here are two reasons why I think we shouldn't worry too much about innateness in trying to
understand the origins of mind.

(1) The question about innateness concerns the first transition, whereas I think the second
should be our focus (for pragmatic reasons: there's more research).

(2) Discoveries about innately-primed learning make only a relatively modest contribution to
understanding the emergence of core knowledge in development. So even when we consider the first
transition, it's not obvious that discoveries about innateness are very illuminating, for all
their pop-science appeal.

Metaphor: we find a cake in the ruins of Pompeii preserved for a couple of thousand years. We're
trying to reconstruct its manufacture.
Its good if someone obsesses about where the eggs came from. Did the baker have her own chickens
or did she get them from a friend? But knowing where the eggs came from is unlikely to be
critical to understanding how the cake was manufactured. We're not finished when we know where
the eggs came from, and we're not doomed to fail if we don't know.

\subsection{slide-168}
So let me put the innateness issue aside and get back to what I think matters most ...

\subsection{slide-169}
This paradigm allows me to highlight something about core knowledge.
I would be a mistake to suppose that there is some core knowledge which later becomes knowledge proper --- e.g. the fact that barriers stop solid objects is first core knowledge then later knowledge.
The content of the core knowledge is a theory of syntax (let's say).
Or, in another case, the content of core knowledge is some principles of object perception.
These are things that human adults do not typically know at all, at least not in the sense that they could state the principles.
So core knowledge enables us to do things, like anticipate where unseen objects will re-appear or communicate with words.
It doesn't seem to be linked directly to the acquisition of concepts.




 %--- end paste
%---------------






\bibliography{$HOME/endnote/phd_biblio}



\end{document}
