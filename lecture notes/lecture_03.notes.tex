%!TEX TS-program = xelatex
%!TEX encoding = UTF-8 Unicode

%\def \papersize {a5paper}
\def \papersize {a4paper}
%\def \papersize {letterpaper}

%\documentclass[14pt,\papersize]{extarticle}
\documentclass[12pt,\papersize]{extarticle}
% extarticle is like article but can handle 8pt, 9pt, 10pt, 11pt, 12pt, 14pt, 17pt, and 20pt text

\def \ititle {Origins of Mind: Lecture Notes}
\def \isubtitle {Lecture 01}
%comment some of the following out depending on whether anonymous
\def \iauthor {Stephen A.\ Butterfill}
\def \iemail{s.butterfill@warwick.ac.uk% \& corrado.sinigaglia@unimi.it
}
%\def \iauthor {}
%\def \iemail{}
%\date{}

%\input{$HOME/Documents/submissions/preamble_steve_paper4}
\input{$HOME/Documents/submissions/preamble_steve_lecture_notes}

%no indent, space between paragraphs
\usepackage{parskip}

%comment these out if not anonymous:
%\author{}
%\date{}

%for e reader version: small margins
% (remove all for paper!)
%\geometry{headsep=2em} %keep running header away from text
%\geometry{footskip=1.5cm} %keep page numbers away from text
%\geometry{top=1cm} %increase to 3.5 if use header
%\geometry{bottom=2cm} %increase to 3.5 if use header
%\geometry{left=1cm} %increase to 3.5 if use header
%\geometry{right=1cm} %increase to 3.5 if use header

% disables chapter, section and subsection numbering
\setcounter{secnumdepth}{-1}

%avoid overhang
\tolerance=5000

%\setromanfont[Mapping=tex-text]{Sabon LT Std}


%for putting citations into main text (for reading):
% use bibentry command
% nb this doesn’t work with mynewapa style; use apalike for \bibliographystyle
% nb2: use \nobibliography to introduce the readings
\usepackage{bibentry}

%screws up word count for some reason:
%\bibliographystyle{$HOME/Documents/submissions/mynewapa}
\bibliographystyle{apalike}


\begin{document}



\setlength\footnotesep{1em}






%---------------
%--- start paste



\title {Origins of Mind \\ Lecture 03}



\maketitle

\subsection{slide-3}
The question for this course is ...
Our current question is about physical objects.
How do humans first come to know simple facts about particular physical objects?

In attempting to answer this question, we are focussing on the abilities of
infants in the first six months of life.

What have we found so far? ...

\subsection{slide-4}
Here’s what we’ve found so far.

We examined how three requirements on having knowledge of physical objects are met.
Knowledge of objects depends on abilities to (i) segment objects, (ii) represent them as
persisting and (iii) track their interactions.
To know simple facts about particular physical objects you need, minimally,
to meet these three requirements.

First discovery: infants meet all three requirements in the first months of their lives:
from around four months of age, and maybe from much earlier.

\subsection{slide-5}
Lots of experiments on each case, e.g. ...

\subsection{slide-7}
Here’s what we’ve found so far.

We examined how three requirements on having knowledge of physical objects are met.
Knowledge of objects depends on abilities to (i) segment objects, (ii) represent them as
persisting and (iii) track their interactions.
To know simple facts about particular physical objects you need, minimally,
to meet these three requirements.

The second discovery concerned how infants meet these three requirements this.

\subsection{slide-8}
The second was that a single set of principles is formally adequate to
explain how someone could meet these requirements, and to describe
infants' abilities with segmentation, representing objects as persisting
and tracking objects' interactions.

This is exciting in several ways.
\begin{enumerate}
\item That infants have all of these abilities.
\item That their abilities are relatively sophisticated: it doesn’t seem
that we can characterise them as involving simple heuristics or relying
merely on featural information.
\item That a single set of principles underlies all three capacities.
\end{enumerate}

\subsection{slide-9}
three requirements, one set of principles: this suggests us that infants’
capacities are characterised by a model of the physical.

\subsection{slide-10}
[slide: model]
three requirements, one set of principles: this suggests us that infants’
capacities are characterised by a model of the physical (as opposed to
being a collection of unrelated capacities that only appear, but don’t
really, have anything to do with physical objects).

\subsection{slide-11}
1. How do four-month-old infants  model  physical objects?

In asking how infants
model physical objects, we are seeking to understand not how physical objects
in fact are but how they appear from the point of view of
an individual or system.

The model need not be thought of as something used by the system: it is
a tool the theorist uses in describing what the system is for and
broadly how it works.
This therefore leads us to a second question ...

\subsection{slide-12}
2. What is the relation between the model and the infants?

\subsection{slide-13}
3. What is the relation between the model and the things modelled (physical objects)?

\subsection{slide-15}
A natuaral answer is the Simple View: the principles of object perception are things that we know or believe, and we generate expectations from these principles by a process of inference.

The Simple View is worth taking seriously for several reasons.
First, it requires no theoretical or conceptual innovation.

Second, as we saw, the Simple View appears to be quite widely supported by
developmental psychologists including Baillargeon and, in writings from the
last millenium, Spelke too.

Third, it can be deduced from the Uncomplicated Account of Minds and Actions.
That is, the developmental evidence together with
a background theory about minds and actions which seems implicit in
much philosophy and perhaps ordinary thinking too
commits us to the Simple View.

\subsection{slide-17}
The evidence suggests that infants can act for the reasons which are
simple facts about particular physical objects.

\subsection{slide-18}
And according to the Uncomplicated Account, this entails that those
simple facts about particular physical objects
are things they believe.

\subsection{unit\_207}


\section{A Problem}

\subsection{slide-20}
As just mentioned, the Simple View is the view that the principles of object perception are things that we know or believe, and we generate expectations from these principles by a process of inference..

\subsection{slide-21}
Why must we reject the simple view?

\subsection{slide-22}
Some philosophers have offered intuitive arguments against the Simple View.
\citet{Bermudez:2003dj}, for instance, holds that those without the
ability to use a language cannot make inferences;
and \citet{Davidson:1975eq} holds that those without language cannot
think at all.
It may be hard to accept that four-month-old infants are in the business
of inferring truths about particular objects’ locations from abstract
principles.
(And perhaps it is no less hard to accept that adults typically do this
in segmenting objects.)
But scientific and mathematical discoveries sometimes require us to
reject intuitions, even intuitions about very fundamental things like
space and time.
For this reason there seems to be slim prospect of effectively
challenging the Simple View on the basis of intuitions about the nature
of knowledge, belief and inference.
Doing so is also unnecessary as there are scientific reasons for
rejecting the simple view.

\subsection{slide-23}
I think we shouldn't try to challenge the simple view on the basis of intution.

\subsection{slide-24}
And we don't need to because there are also scientific reasons for rejecting the simple view.

One set of reasons concerns the apparent discrepancy between looking times and manual search ...

\subsection{slide-25}
*(The basic idea is to say there's a discrepancy regarding BOTH (a) permanence and (b)
causal interactions)

\subsection{slide-26}
Recall this experiment which used habituation to demonstrate infants' abilities to represent
objects as persiting while unperceived (in this case, because occluded).
Infants can do this sort of task from 2.5 months or earlier \citep{Aguiar:1999jq}.

But what happens if instead of measuring how infants look, we measure how they reach?

\subsection{slide-27}
\citet{Shinskey:2001fk} did just this.
Here you can see their appratus.
They had a screen that infants could pull forwards to get to an object that was sometimes
hidden behind it.
They made two comparisons.
First, were infants more likely to pull the screen forwards when an object was placed behind it?
Second, were how did infants' performance compare when the barrier was not opaque but transparent?

\subsection{slide-28}
Here are their results with 7-month old infants.

\subsection{slide-29}
Now we have the beginnings of a problem.
The problem is that, if the Simple View is right, infants should succeed in tracking persisting
objects regardless of whether we measure their eye movements or their reaching actions.
But there is a gap of around five months between looking and reaching.

The attraction of the simple view is that it explains the looking.
The problem for the simple view is that it makes exactly the wrong prediction about the reaching.

Can we explain the discrepancy in terms of the additional difficulty of reaching?
A lot of experiments have attempts to pin the discrepancy on this, or on other extraneous factors like task demands.
But none of these attempts have succeeded.
After all, we know infants are capable of acting because they move the transparent screen.

\subsection{slide-30}
As Jeanne Shinskey, one of the researchers most dedicated to this issue says,

\subsection{slide-31}
If there were just one discrepancy, concerning performance, we might be able to hold on to the
Simple View.  But there are systematic discrepancies along these lines.

Related discrepancies concerning infants' understanding of physical objects occur in the case
of their abilities to track causal interactions, too.

\subsection{slide-32}
Recall this experiment about causal interactions, which used a habituation paradigm.
Now imagine a version which involved getting infants to reach for the object rather than simply looking.
What would the results be?
There is an experiment much like this which has been replicated several times, and which shows
a discrepancy between looking and searching.
Basically infants will look but not search.

\subsection{slide-33}
*todo

\subsection{slide-34}
*todo

\subsection{slide-35}
*todo

\subsection{slide-36}
Here are the looking time results.

\subsection{slide-37}
You can even do looking time and reaching experiments with the same subjects and apparatus \citep{Hood:2003yg}.

2.5-year-olds look longer when experimenter removes the ball from behind the wrong door, but don't reach to the correct door

\subsection{slide-38}
here are the search results (shocking).

\subsection{slide-39}
*todo: describe

**todo: Mention that \citep{mash:2006_what} show infants can also predict the location of the object (not just identify a violation, but look forward to where the object is)

\subsection{slide-40}
Amazingly, 2 year old children still do badly when only the doors are opaque, so that the
ball can be seen rolling between the doors, as in this diagram \citep{Butler:2002bv}.

\subsection{slide-43}
This is the end of the road for the Simple View.
If, like Baillargeon, you want to cling to the Simple View, then you
need something very convincing to say about the fact that it appears
to generate multiple incorrect predictions.

\subsection{slide-44}
Similar discrepancies between looking and reaching are also found in some
nonhuman primates, both apes and monkeys (chimpanzees, cotton-top tamarins
and marmosets). (Some of this is based on the gravity tube task and
concerns gravity bias.)

‘A similar permanent dissociation in understanding object support relations
      might exist in chimpanzees. They identify impossible support relations in looking tasks,
      but fail to do so in active problem solving.’
\citep{gomez:2005_species}

\subsection{slide-46}
Note that this research is evidence of dissociations between looking and
search in adult primates, not infants.
This is important because it indicates that the failures to search are a
feature of the core knowledge system rather than a deficit in human
infants.

‘to date, adult primates’ failures on search tasks appear to
      exactly mirror the cases in which human toddlers perform poorly.’
\citep[p.\ 17]{santos:2009_object}

\subsection{slide-47}
What about the chicks and dogs?
This isn't straightforward.
As I mentioned earlier, \citep{kundey:2010_domesticated} show that domestic dogs are good
at solidity on a search measure.  And as we covered in seminars,
\citep{chiandetti:2011_chicks_op} demonstrate object permanence with a search measure in
chicks that are just a few days old.
Indeed, for many of the other animals I mentioned, object permanence is measured in search
tasks, not with looking times.
To speculate, it may be that the looking/search dissociation is more likely to occur in
adult animals the more closely related they are to humans.
But let's focus on the fact that you get the looking/search in any adult animals at all.
This is evidence that the dissociation is a consequence of something fundamental about
cognition rather than just a side-effect of some capacity limit.

\subsection{slide-48}
This really is the end of the road for the Simple View.
But it is actually worse ...

\subsection{slide-49}
Because this point is controversial, I want to mention one further
piece of the puzzle.
Five-month-olds not only sometimes fail to search for hidden objects but
...

\subsection{slide-50}
... they also sometimes fail to look longer when a momentarily hidden object fails
to reappear as if by magic.
Infants will reach for an object hidden in darkness \citep[e.g.][]{jonsson:2003_infants}.
But what happens if instead of measuring reaching we measure looking times?
\citet{charles:2009_object} compared what happens when an object is
momentarily hidden behind a screen with what happens when an object is
momentarily hidden by darkness.
They used a trick with light and mirrors so that for some of the infants,
the object did not reappear when the screen came up or the light returned.
Surprisingly, five-month-old infants’ looking times indicated that an
expectation had been violated only when the object was hidden behind a
screen but not when hidden by darkness.

I think this pattern of findings is good evidence against the hypothesis
that four- or five-month-olds have beliefs about, or knowledege of,
the locations of
unperceived objects.
After all,
a belief is essentially the kind of state that can inform actions of any
kind, whether they involve looking, searching with the hands or anything
else.

NB: Charles \& Rivera did the v-of-e part; the manual search part has been
done by others.

\subsection{slide-54}
(There was also a fade condition which I’m not discussing.)

\subsection{slide-55}
The results are complicated.  They compare occlusion to empty
and endarkening to empty.

They comment that, for occlusion vs empty:
‘the condition-by- outcome interaction usually interpreted as ‘having
object permanence’, though definitely not present in the first trial pair,
became visible by the fourth trial pair (see Figure 5).’
This is why you see that figure there.

Occulsion: But actually their analysis depends on
‘The significant three-way interaction existed because the disparity
between infants’ looking at the two outcomes increased across trial pairs
in the Empty Condition, but decreased across pairs in the Occlusion
Condition, F(3, 108) = 2.83, p < .05, partial g2 = .07.’

Occulsion:  (Note: ‘The predicted two-way condition-by-outcome interaction was not
significant, F(1, 36) = 0.43, p > .51, partial g2 = .01, but one main
effect and the alternatively predicted three- way interaction were
significant.’)

Occlusion: ‘Infants did not show the two-way interaction between condition
and outcome; however, they did show the specific three-way interaction
interpretable in terms of infants’ exhibiting an understanding of object
permanence, but needing time to acclimate to the procedures. That is, the
pattern suggests that infants came to expect the occluded object to
reappear over the course of trials in the Occlusion Condition, but came to
expect it to be gone over the course of trials in the Empty Condition.’

[The complication here is the main effect: infants look longer when there’s
something than when there’s nothing, not surprisingly.
A better approach might be to do Wynn’s 1992 (‘two mouse’) experiment with
occlusion and endarkening---in her design there is always something to look at,
and she found that infants don’t prefer to look at one vs two mice.]

\subsection{slide-56}
When you compare the Empty and Endarkening conditions, you don’t get an interaction.
‘The three-way interaction found in the Occlusion vs. Empty experiment
was not present here, F(3, 108) = .176, p > .90. The lack of interaction
is clearly visible in ... Figure 6.’

‘Infants’ looking patterns in the Darkness Condition were almost
identical to those of the Empty Condition. ...
If the results of the Occlusion Condition are taken to indicate that
infants expect occluded objects to continue existing, then infants’
behavior in the Darkness Condition must be taken to indicate that infants
do not expect endarkened objects to continue existing.’

[NOTE: Imperfect to compare Endarkening and Empty since Empty involves the
screen coming down but no change in luminance (as far as I can tell --- it’s
not clear exactly how they did that from the procedure.)]

[NOTE: I’m not discussing the Fade condition but the results are interesting:
‘Infants in the Fade Condition behaved similarly to infants in the
Occlusion Condition, demonstrating expectations for the reappearance of
faded objects. This is the opposite of the expectation adults have under
similar conditions, and the opposite of what would be expected under the
ecological hypothesis. The pattern of results across all conditions –
expectation for the reappearance of occluded and faded objects, but not
for endarkened objects – cannot be reconciled with the traditional
ecological hypothesis.’]

\subsection{slide-57}
Why is this a major challenge for the Simple View?
Because it shows that to defend the Simple View, it’s not enough to
explain failures of manual search.

\subsection{slide-59}
To defend the Simple View, you also have to explain failure in
a violation-of-expectations task when manual search succeeds.

\subsection{slide-60}
This really is the end of the road for the Simple View.
We must reject this view because it makes systematically incorrect
predictions about actions like searching for occluded objects and about
looking behaviours involving endarkened objects.

This a problem? Why? Because, as we'll see, it is hard to identify an
alternative.

\subsection{unit\_211}


\section{Like Knowledge and Like Not Knowledge}

\subsection{slide-62}
I'm sorry to keep repeating this but I want everyone to understand where we
are. There are principles of object perception that explain abilities to
segment objects, to represent them while temporarily unperceived and to
track their interactions. These principles are not known. What is their
status?

\subsection{slide-64}
The Simple View answers this question.
But the Simple View is incorrect.
So we need an alternative answer.
And it is difficult to find one because ...

\subsection{slide-67}
The problem is quite general.
It doesn't arise only in the case of knowledge of objects but also in other domains
(like knowledge of number and knowledge of mind).
And it doesn't arise only from evidence about infants or nonhuman primates; it would also
arise if our focus were exclusively on human adults.
More on this later.
For now, our aim is to better understand the problem as it arises in the case of knowledge of objects.

\subsection{slide-74}
One hopeful alternative is to shift from talk about knowlegde to talk about representation.
Will this help?
Only as a way of describing the problem.

We need to say what we mean by representation.
The term is used in a wide variety of ways.
As I use it, representation is just a generic term covering knowledge,
belief and much else besides.

\subsection{slide-75}
If we are going to substitute representation for knowledge,
we need to characterise what kind of representation we have in mind.
The term is tricky.
As \citet{Haith:1998aq} says, ‘no concept causes more problems in discussions of infant cognition than that of representation.’

\subsection{slide-76}
Take a paradigm case of representation.

\subsection{slide-78}
The subject might not be the agent but some part of it.

That is, we can imagine that some component of an agent, like her
perceptual system or motor system, represents things that she herself does
not.

(Of course, to make sense of this idea we need to invoke some notion of system.)

\subsection{slide-80}
The content is what distinguishes one belief from all others, or one desire from all others.

The content is also what determines whether a belief is true or false, and
whether a desire is satisfied or unsatisfied.

There are two main tasks in constructing a theory of mental states.

The first task is to characterise the different attitudes.

This typically involves specifying their distinctive functional and normative roles.

The second task is to find a scheme for specifying the contents of mental states.

\subsection{slide-82}
The second task is to find a scheme for specifying the contents of mental states.

Usually this is done with propositions.

But what are propositions?

Propositions are abstract objects like numbers.

They have more mystique than numbers, but, like numbers, they are abstract objects that can be constructed using sets plus a few other basic ingredients such as objects, properties and possible worlds.

\subsection{slide-83}
So that was some quick background on representation.

Note that the issue of representation comes up twice for us.

There is a question about whether the principles of object perception are represented.

And there is a question about whether objects, their locations, properties, and interactions are represented.

The problem raised by the discrepancy between looking and acting is a
problem for two claims:
(i) the simple view (the principles of object perception are knowledge \&c); and also
(ii) the claim that the representations of objects which derive from the
principles of object perception are knowledge states.

\subsection{slide-84}
So to say that we don't know the principles of object perception but only represent them doesn't tell us much.
This is a step in the right direction.
But it tells us that we represent them without knowing them.

What we need if we're to have an explanatory answer to Q2a is to know positively how we do represent the principles of object perception --- subject, attitude and content.
We need to characterise a form of representation that is like knowledge but not like knowledge.

\subsection{slide-85}
Your handbag is bluging, and when you swing it at me something really hard hits me.

It must be full of rocks.

Except it can't be because you are not strong enough to lift such a big bag full of rocks.

In that case, it must be wrocks not rocks.

A wrock is just like a rock except that it lacks mass.

Compare: this representation is just like knowledge except that it doesn't guide action; this process is just like inference except that it lacks the normative aspects of inference.

\subsection{slide-86}
A different way to duck Davidson’s Challenge
has been proposed by \citeauthor{Munakata:2001ch} (\citeyear{Munakata:2001ch}; see also \citealp{munakata:1997_rethinking}).
She suggests that knowledge can be ‘graded’: some knowledge states are ‘stronger’ while others are ‘weaker’.
She also holds that weaker knowledge states can drive looking time behaviours but not control purposive action.
This allows her to hold that four-month-olds know principles about objects generally and facts about particular objects,
just as adults do.
So on this view infants’ representations of objects are not different in kind from adults’ knowledge of facts about particular physical objects.
% \citep{shinskey:2010_something} specify that the graded representations idea is supposed to be an alternative to postulating core knowledge.

The idea that knowledge can be graded is initially attractive.
It appears to avoid the incorrect predictions of the Simple View, yet it does not require meeting Davidson’s Challenge and identify things in between mindless behaviour and knowledge or belief.
But this idea also faces a challenge.
Talk about ‘strength’ in this context needs to be anchored in a theory of representation.
When talking about a radio signal, a notion of strength can be defined in
terms of physical properties of waves.
In this case it is possible to specify what signal strength amounts to, and it is clear that signals of varying strength can all carry the same message.
But what does strength amount to in the case of a mental representation or knowledge state?

Without an answer to this question, invoking graded representations will not explain anything.
It amounts merely to retrospectively postulating a novel aspect of representation to characterise findings about what four-month-olds can and can’t do.

To see the force of this challenge, consider that proponents of graded knowledge hold that weaker representations can guide many looking behaviours whereas manual search behaviours generally require stronger representations.
Why is it this way around?
Why is a stronger representation is needed for manually searching than for looking?
Until we can answer this question, postulating graded knowledge will not explain the developmental puzzles about knowledge of objects.

The idea might well make sense if we were talking about neural representations.

But here we aren't.  Let's not introduce radically new ideas about representation unless we really have to.

(By the way, \citet{Munakata:2001ch} is a nice review of dissociations, not only developmental dissociations.)

\subsection{slide-87}
Recall what Davidson said: we need a way of describing what is in between thought and
mindless nature.  This is the challenge presented to us by the failure of the Simple View.

\subsection{unit\_215}


\section{The Problem with the Simple View: Summary}

I'm sorry to keep repeating this but I want everyone to understand where we
are. There are principles of object perception that explain abilities to
segment objects, to represent them while temporarily unperceived and to
track their interactions. These principles are not known. What is their
status?

\subsection{slide-89}
The question is ...
How do humans first come to know simple facts about particular physical objects?

\subsection{slide-90}
Here’s what we’ve found so far.

We examined how three requirements on having knowledge of physical objects are met.
Knowledge of objects depends on abilities to (i) segment objects, (ii) represent them as
persisting and (iii) track their interactions.
To know simple facts about particular physical objects you need, minimally,
to meet these three requirements.

The second discovery concerned how infants meet these three requirements this.

\subsection{slide-91}
The second was that a single set of principles is formally adequate to
explain how someone could meet these requirements, and to describe
infants' abilities with segmentation, representing objects as persisting
and tracking objects' interactions.

This is exciting in several ways.
\begin{enumerate}
\item That infants have all of these abilities.
\item That their abilities are relatively sophisticated: it doesn’t seem
that we can characterise them as involving simple heuristics or relying
merely on featural information.
\item That a single set of principles underlies all three capacities.
\end{enumerate}

\subsection{slide-92}
2. What is the relation between the model and the infants?

\subsection{slide-93}
A natuaral answer is the Simple View: the principles of object perception are things that we know or believe, and we generate expectations from these principles by a process of inference.

\subsection{slide-94}
The Simple View generates incorrect predictions.

\subsection{slide-95}
However, as we saw, the Simple View must be wrong because it generates incorrect predictions.
This was the lesson of the discrepancy bewteen looking and search measures for both
infants' abilities to represent objects as persisting and their abilities to track causal
interactions.

\subsection{slide-96}
As I've just been arguing, the failure of the Simple View presents us with a problem.
The problem is to understand the nature of infants' apprehension of the principles given
that it doesn't involve knowledge.
This is a problem that will permeate our discussion of the origins of mind because it
problems of this form come up again and again in different domains.
It isn't the only problem we'll encounter, but none of the problems are more important or more
general than this one.

\subsection{unit\_601\_warwick}


\section{What Is Core Knowledge?}

\subsection{slide-98}
I talked about the notion of core knowledge in the very first lecture, but since then I
have not appealed to the notion.
This is deliberate because the notion is tricky; so I thought it would be good to postpone
our discussion of it for as long as possible.
Now I can put it off no longer.

\subsection{slide-100}
The first, very minor thing is to realise that there are two closely related notions, core
knowledge and core system.

\subsection{slide-101}
These are related this: roughly, core knowledge states are the states of core systems.  More
carefully:

For someone to have \textit{core knowledge of a particular principle or
fact} is for her to have a core system where
either the core system includes a representation of that principle or
else the principle plays a special role in describing the core system.

So we can define core knowlegde in terms of core system.

\subsection{slide-102}
What do people say core knowledge is?

There are two parts to a good definition. The first is an analogy that
helps us get a fix on what we is meant by 'system' generally. (The second
part tells us which systems are core systems by listing their
characteristic features.)

\subsection{slide-103}
So talk of core knowledge is somehow supposed to latch onto the idea of a system.

What do these authors mean by talking about 'specialized perceptual systems'?

They talk about things like perceiving colour, depth or melodies.

Now, as we saw when talking about categorical perception of colour, we can think of the 'system' underlying categorical perception as largely separate from other cognitive systems--- we saw that they could be knocked out by verbal interference, for example.

So the idea is that core knowledge somehow involves a system that is separable from other cognitive mechanisms.

As Carey rather grandly puts it, understanding core knowledge will involve understanding something about 'the architecture of the mind'.

\subsection{slide-104}
Illustration: edge detection.

\subsection{slide-105}
This, them is the two part definition.  An analogy and a list of features.

\subsection{slide-106}
There is one more feature that I want to mention; this is important
although I won't disucss it here.
To say that a represenation is iconic means, roughly, that parts of the
representation represent parts of the thing represented.
Pictures are paradigm examples of representations with iconic formats.
For example, you might have a picture of a flower where some parts of the
picture represent the petals and others the stem.
\subsection{slide-107}
Why postulate core knowledge?

\subsection{slide-108}
The first problem we encountered was that the Simple View is false.
But maybe we can appeal to the Core Knowledge View.

According to the Core Knowledge View, the principles of object perception, and maybe also the
expectations they give rise to, are not knowledge.
But they are core knowledge.

This raises some issues.  Is the Core Knowledge View consistent with the claims that
we have ended up with, e.g. about categorical perception and the Principles of Object
Perception characterising the way that object indexes work?
I think the answer is, basically, yes.  Categorical perception involves a system that has
many of the features associated with core knowledge.

[*looking ahead (don’t say):]
Consider this hypothesis.
The principles of object perception, and maybe also the expectations they give rise to, are not knowledge.
But they are core knowledge.
The \emph{core knowledge view}: the principles of object perception are
not knowledge, but they are core knowledge.
But look at those features again --- innate, encapsulated, unchanging and the rest.
None of these straightforwardly enable us to predict that core knowledge
of objects will guide looking but not reaching.
So the \emph{first problem} is that (at this stage) it's unclear what we
gain by shifting from knowledge to core knowledge.

\subsection{slide-109}
The Core Knowledge view may also help us to resolve
Discrepant Findings in other domains too ...
\subsection{slide-110}
Why postulate core knowledge?

\subsection{unit\_604\_short}


\section{Objections to Core Knowledge}

\subsection{slide-112}
Recall that we defined core systems by listing properties.
[(Actually it was a two-part definition so there’s hope.)]

\subsection{slide-113}
One objection is that there are multiple definitions, each slightly different
from the others, and no obvious way to choose between them.

But although this indicates that we need to impose some theoretical discipline,
it doesn’t seem like an objection that could show there is a deep problem
with the notion of Core Knowledge.

\subsection{slide-114}
Here is a second objection ...

One reason for doubting that the notion of a core system is explanatory arises from the
way we have introduced it.
We have introduced it by providing a list of features.
But why suppose that this particular list of features constitutes a natural kind?
This worry has been brought into sharp focus by criticisms of 'two systems' approaches.
(These criticisms are not directed specifically at claims about core
knowledge, but the criticisms apply.)

‘there is a paucity of … data to suggest that they are the only or the best way of carving up the processing,

‘and it seems doubtful that the often long lists of correlated attributes should come as a package’
\citep[p.\ 759]{adolphs_conceptual_2010}

\subsection{slide-115}
‘we wonder whether the dichotomous characteristics used to define the two-system
models are … perfectly correlated …
[and] whether a hybrid system that combines characteristics from both
systems could not be … viable’
\citep[p.\ 537]{keren_two_2009}

\subsection{slide-116}
This is weak.

Remember that criticism is easy, especially if you don't have to prove someone is wrong.

Construction is hard, and worth more.

\subsection{slide-118}
Even so, there is a problem here.

‘the process architecture of social cognition is still very much in need of a detailed theory’
\citep[p.\ 759]{adolphs_conceptual_2010}

\subsection{slide-119}
Is definition by listing features (a) justified, and is it (b) compatible with the claim that core knowledge is explanatory?






So far I've been explaining objection (a).  Now let me say a bit more about (b) ...




\subsection{slide-120}
So far I've been explaining objection (a).  Now let me say a bit more about (b) ...

\subsection{slide-121}
We can get the strongest objection by asking ...

Why do we need a notion like core knowledge?

\subsection{slide-122}
So why do we need a notion like core knowledge?
Think about these domains.
In each case, we're pushed towards postulating that infants know things, but
also pushed against this.
Resolving the apparent contradiction is what core knowledge is for.

Key question: What features do we have to assign to core knowledge if it's to
describe these discrepancies?

\subsection{slide-123}
In the case of Physical Objects, we want to  expalin this
puzzling pattern of findings ...

\subsection{slide-124}
If this is what core knowledge is for (if it exists to explain these discrepancies), what
features must core knowledge have?

\subsection{slide-125}
Which of these features explain the discrepancy between
measures on which infants do, and measures on which they do not,
manifest their abilities to track physical objects?

Why do they fail on some search tasks and but pass some v-of-e tasks
when the mode of disappearance is occlusion?

And, equally pressingly, why do they do the converse (pass search, fail v-of-e)
when the mode is endarkening?

\subsection{slide-127}
Encapsulated : there are limits on what information can get into the system.
But if we are to explain any successes, it must be possible for information
about the locations of physical objects to get into the system.
So there’s no way we can use encapsulation to explain the puzzling developmental
findings.

\subsection{slide-132}
So, to return to my question,

If this is what core knowledge is for, what features must core knowledge have?

The answer seems to be: none of the features that are stipluated in introducing it.
This gives us a \textbf{first objection}: there seems to be a mismatch between
the definition and application.

[The feature we most need is actually missing from their list: limited accessibility.
But this thought comes later.]

\subsection{slide-133}
summary

\subsection{unit\_231}


\section{Core System vs Module}

\subsection{slide-136}
The problems with core knowledge look like they might be
the sort of problem a philospoher might be able to help with.

Jerry Fodor has written a book called 'Modularity of Mind' about what he calls modules.
And modules look a bit like core systems, as I'll explain.
Further, Spelke herself has at one point made a connection.
So let's have a look at the notion of modularity and see if that will help us.

\subsection{slide-137}
‘In Fodor’s (1983) terms, visual tracking and preferential looking each may depend on modular mechanisms.’
\citep[p.\ 137]{spelke:1995_spatiotemporal}

\subsection{slide-138}
So what is a modular mechanism?

\subsection{slide-139}
Fodor’s three claims about modules:

\begin{enumerate}

\item they are ‘the psychological systems whose operations present the world to thought’;

\item they ‘constitute a natural kind’; and

\item there is ‘a cluster of properties that they have in common’ \citep[p.\ 101]{Fodor:1983dg}.

\end{enumerate}

Modules are widely held to play a central role in explaining mental development and in accounts of the mind generally.

Jerry Fodor makes three claims about modules:

\subsection{slide-140}
What are these properties?

Properties of modules:

\begin{itemize}

\item domain specificity (modules deal with ‘eccentric’ bodies of knowledge)

\item limited accessibility (representations in modules are not usually inferentially integrated with knowledge)

\item information encapsulation (modules are unaffected by general knowledge or representations in other modules)

\item innateness (roughly, the information and operations of a module not straightforwardly consequences of learning; but see \citet{Samuels:2004ho}).

\end{itemize}

For something to be informationally encapsulated is for its operation to be unaffected by the mere existence of general knowledge or representations stored in other modules (Fodor 1998b: 127)

Domain specificity

\subsection{slide-141}
limited accessibility

\subsection{slide-142}
Let me illustrate limited accessibility ...

Limited accessbility is a familar feature of many cognitive systems.
When you grasp an object with a precision grip, it turns out that there is a very
reliable pattern.
At a certain point in moving towards it your fingers will reach a maximum grip aperture
which is normally a certain amount wider than the object to be grasped, and then start to close.
Now there's no physiological reason why grasping should work like this, rather than grip hand
closing only once you contact the object.
Maximum grip aperture shows anticipation of the object: the mechanism responsible for
guiding your action does so by representing various things including some features of
the object.
But we ordinarily have no idea about this.
The discovery of how grasping is controlled depended on high speed photography.
This is an illustration of limited accessibility.
(This can also illustrate information encapsulation and domain specificity.)

\subsection{slide-143}
Illusion sometimes affects perceptual judgements but not actions:
information is in the system;
information is not available to knowledge \citep{glover:2002_visual}.

See further \citet{bruno:2009_when}:
They argue that Glover \& Dixon's model \citep{glover:2002_dynamic} is
incorrect, at least for grasping (pointing is a different story), because
it predicts that the presence or absence of visual information during
grasping shouldn't matter. But it does.

\subsection{slide-144}
You also get evidence for information encapsulation in four month olds.
To illustrate, consider \citet{vanwermeskerken:2013_getting} ...

A Ponzo-like background can make one frog further away than the other.

\subsection{slide-145}
This affects which object four-month olds reach for,
but does not affect the kinematics of their reaching actions.

\subsection{slide-146}
What are these properties?

For something to be informationally encapsulated is for its operation to be unaffected by the mere existence of general knowledge or representations stored in other modules (Fodor 1998b: 127)

\subsection{slide-147}
Information encapsulation

\subsection{slide-148}
Innateness

\subsection{slide-149}
So these are the key properties associated with modularity.

\subsection{slide-150}
We've seen something like this list of properties before ...
Compare the notion of a core system with the notion of a module

The two definitions are different, but the differences are subtle enough that we don't want both.
My recommendation: if you want a better definition of core system, adopt
core system = module as a working assumption and then look to research on modularity
because there's more of it.

\subsection{slide-155}
I think it is reasonable to identify core systems with modules and to
largely ignore what different people say in introducing these ideas.
The theory is not strong enough to support lots of distinctions.

\subsection{slide-156}
Will the notion of modularity help us in meeting the objections to the Core Knowledge View?

\subsection{slide-158}
Recall that the challenges were these:



multiple definitions


justification for definition by list-of-features


definition by list-of-features rules out explanation


mismatch of definition to application



Let’s go back and see what Fodor says about modules again ...

\subsection{slide-159}
Consider the first objection, that there are multiple defintions ...

\subsection{slide-160}
Not all researchers agree about the properties of modules.  That they are
informationally encapsulated is denied by Dan Sperber and Deirdre Wilson (2002: 9),
Simon Baron-Cohen (1995) and some evolutionary psychologists (Buller and Hardcastle 2000: 309),
whereas Scholl and Leslie claim that information encapsulation is the essence of modularity
and that any other properties modules have follow from this one (1999b: 133; this also seems
to fit what David Marr had in mind, e.g. Marr 1982: 100-1).  According to Max Coltheart,
the key to modularity is not information encapsulation but domain specificity; he suggests
Fodor should have defined a module simply as 'a cognitive system whose application is domain
specific' (1999: 118).  Peter Carruthers, on the other hand, denies that domain specificity
is a feature of all modules (2006: 6).  Fodor stipulated that modules are
'innately specified' (1983: 37, 119), and some theorists assume that modules,
if they exist, must be innate in the sense of being implemented by neural regions
whose structures are genetically specified (e.g. de Haan, Humphreys and Johnson 2002: 207;
Tanaka and Gauthier 1997: 85); others hold that innateness is 'orthogonal' to modularity
(Karmiloff-Smith 2006: 568).  There is also debate over how to understand individual
properties modules might have (e.g. Hirschfeld and Gelman 1994 on the meanings of domain
specificity; Samuels 2004 on innateness).

In short, then, theorists invoke many different notions of modularity, each barely different
from others.  You might think this is just a terminological issue.  I want to argue that
there is a substantial problem: we currently lack any theoretically viable account of what
modules are.  The problem is not that 'module' is used to mean different things-after all,
there might be different kinds of module.  The problem is that none of its various meanings
have been characterised rigorously enough.  All of the theorists mentioned above except Fodor
characterise notions of modularity by stipulating one or more properties their kind of module
is supposed to have.  This way of explicating notions of modularity fails to support principled
ways of resolving controversy.

No key explanatory notion can be adequately characterised by listing properties because the
explanatory power of any notion depends in part on there being something which unifies its
properties and merely listing properties says nothing about why they cluster together.

So much the same objections which applied to the very notion of core knowledge
appear to recur for module.  But note one interesting detail ...

\subsection{slide-165}
Will the notion of modularity help us in meeting the objections to the Core Knowledge View?

We’ve been considering the first objection, that there are multiple defintions ...

\subsection{slide-166}
What about the objection that picking out a set of features is unjustified ...

\subsection{slide-168}
Interestingly, Fodor doesn't define modules by specifying a cluster of properties
(pace Sperber 2001: 51); he mentions the properties only as a way of gesturing towards
the phenomenon (Fodor 1983: 37) and he also says that modules constitute a natural kind
(see Fodor 1983: 101 quoted above).

\subsection{slide-169}
Will the notion of modularity help us in meeting the objections to the Core Knowledge View?

We’ve been considering the first objection, that there are multiple defintions ...

\subsection{slide-170}
Same point applies to the claim that defining module by listing features is
unexplanatory: if we are not listing features but identifying a natural kind,
then the objection doesn’t quite get strarted.

As far as the
‘justification for definition by list-of-features’ and
‘definition by list-of-features rules out explanation’ problems go,
everything rests on the idea that modules are a natural kind.
I think this idea deserves careful scruitiny but as far as I know there's
only one paper on this topic, which is by me.
I'm not going to talk about the paper here; let me leave it like this:
if you want to invoke a notion of core knowledge or modularity,
you have to reply to these problems.  And one way to reply to them---
the only way I know---is to develop the idea that modules are a natural
kind.  If you want to know more ask me for my paper and I'll send it to you.

\subsection{slide-172}
Recall the discrepancy in looking vs search measures.
What property of modules could help us to
explain it?

\subsection{slide-174}
For something to be informationally encapsulated is for its operation to be unaffected by the mere existence of general knowledge or representations stored in other modules (Fodor 1998b: 127)

\subsection{slide-175}
We already considered innateness and inforamtion encapsulation

\subsection{slide-177}
To say that a system or module exhibits limited accessibility is to say
that the representations in the system are not usually inferentially
integrated with knowledge.

This is a key feature we need to assign to modular
representations (=core knowledge) in order to explain the apparent
discrepancies in the findings about when knowledge emerges in development.

Limited accessibility explains why the representations might drive
some actions (e.g. certain looking behaviours) but not others (e.g.
certain searching actions).

But't the bare appeal to limited accessibility leaves open why the
looking and not the searching (rather than conversely).
Further, we have to explain why not searching with occlusion
whereas not looking with endarkening.
Clearly we can’t explain this pattern just by invoking information
encapsulation.

And, of course, to say that we can explain something by invoking
information encapsulation is too much.
After all, limited accessibility is more or less what we're trying to explain.
But this is the first problem --- the problem with the standard way of
characterising modularity and core systems merely by listing features.

\subsection{slide-179}
The main objection is unresolved

\subsection{unit\_665}


\section{A Hypothesis: Object Indexes Underpin Infants’ Abilities}

\subsection{slide-182}
How do four- and five-month-olds track briefly occluded objects?

\subsection{slide-184}
The leading, best defended hypothesis is that their abilities to do so
depend on a system of
object indexes like that which underpins multiple object tracking or
object-specific preview benefits
\citep{Leslie:1998zk,Scholl:1999mi,Carey:2001ue,scholl:2007_objecta}.

\subsection{slide-185}
But what is an object index?
Formally, an object index is ‘a mental token that functions as a
pointer to an object’ \citep[p.\ 11]{Leslie:1998zk}.
If you imagine using your fingers to track moving objects,
an object index is the mental counterpart of a finger \citep[p.~68]{pylyshyn:1989_role}.

The interesting thing about object indexes is that a system of object
indexes (at least one, maybe more)
appears to underpin cognitive processes which are not
strictly perceptual but also do not involve beliefs or knowledge states.
While I can’t fully explain the evidence for this claim here,
I do want to mention the two basic experimental tools that are used to
investigate the existence of, and the principles underpinning,
a system of object indexes which operates
between perception and thought ...

\subsection{slide-186}
Suppose you are shown a display involving eight stationary circles, like
this one.

\subsection{slide-187}
Four of these circles flash, indicating that you should track these circles.

\subsection{slide-188}
All eight circles now begin to move around rapidly, and keep moving unpredictably for some time.

\subsection{slide-189}
Then they stop and one of the circles flashes.
Your task is to say whether the flashing circle is one you were supposed to track.
Adults are good at this task \citep{pylyshyn:1988_tracking}, indicating that they can use at least four object indexes simultaneously.

(\emph{Aside.} That this experiment provides evidence for the existence of
a system of object indexes has been challenged.
See \citet[p.\ 59]{scholl:2009_what}:
\begin{quote}
`I suggest that what Pylyshyn’s (2004) experiments show is exactly what they intuitively
seem to show: We can keep track of the targets in MOT, but not which one is which.
[...]
all of this seems easily explained [...] by the view
that MOT is simply realized by split object-based attention to the MOT targets as a set.'
\end{quote}
It is surely right that the existence of MOT does not, all by itself,
provide support for the existence of a system of object indexes.
However, contra what Scholl seems to be suggesting here, the MOT paradigm
can be adapated to provide such evidence.
Thus, for instance, \citet{horowitz:2010_direction} show that, in a MOT paradigm, observers
can report the direction of one or two targets without advance knowledge of which
targets' directions they will be asked to report.)

\subsection{slide-190}
There is a behavioural marker of object-indexes called the object-specific preview benefit.
Suppose that you are shown an array of two objects, as depicted here.
At the start a letter appears briefly on each object.
(It is not important that letters are used; in theory, any readily
distinguishable features should work.)

\subsection{slide-191}
The objects now start moving.

\subsection{slide-192}
At the end of the task, a letter appears on one of the objects.
Your task is to say whether this letter is one of the letters that appeared at the start or whether it is a new letter.
Consider just those cases in which the answer is yes: the letter at the end is one of those which you saw at the start.
Of interest is how long this takes you to respond in two cases: when the letter appears on the same object at the start and end, and, in contrast, when the letter appears on one object at the start and a different object at the end.
It turns out that most people can answer the question more quickly in the first case.
That is, they are faster when a letter appears on the same object twice than when it appears on two different objects
\citep{Kahneman:1992xt}.
This difference in response times is the
% $glossary: object-specific preview benefit
\emph{object-specific preview benefit}.
Its existence shows that, in this task, you are keeping track of which object is which as they move.
This is why the existence of an object-specific preview benefit is taken to be evidence that object indexes exist.

The \emph{object-specific preview benefit}: ‘observers can identify target
letters that matched the preview letter from the same object faster than
they can identify target letters that matched the preview letter from the
other object’ \citep[p.\ 2]{Krushke:1996ge}.

\subsection{slide-193}
In what follows I will take it for granted that, in adult humans,
there is a system of object indexes which enables them to track
potentially moving objects in ongoing actions such as visually tracking or
reaching for objects, and which influences how their attention is allocated
\citep{flombaum:2008_attentional}.

\subsection{slide-194}
This system of object indexes
does not involve belief or knowledge
and may assign indexes to objects in ways that are inconsistent with
a subject’s beliefs about the identities of objects
\citep[e.g.][]{Mitroff:2004pc, mitroff:2007_space}

\subsection{slide-195}
We have observed one behavioural marker of object
indexes, namely the object-specific preview benefit.

\subsection{slide-196}
There are also neural markers of object indexes.
That is, in adults there is a pattern of brain activity which appears to be
characteristic of processes involved in maintaining an object index
for an object that is briefly hidden from view.

\subsection{slide-197}
The system of object indexes is also subject to signature limits.
In general, a \emph{signature limit of a system} is a pattern of behaviour the system exhibits which is both defective given what the system is for and peculiar to that system.

One signature limit of a system of object indexes is that featural information sometimes fails to influence how objects are assigned in ways that seem quite dramatic.
Let me illustrate ...

\subsection{slide-198}
In this scenario,
a patterned square disappears behind the barrier; later a plain black ring emerges.
If you consider speed and direction only, these movements are consistent with there being just one object.
But given the distinct shapes and textures of these things, it seems all but certain that there must be two objects.
Yet in many cases these two objects will be assigned the same object index \citep{flombaum:2006_temporal,mitroff:2007_space}.
So one signature limit of systems of object indexes is that information about speed and distance can override information about shape and texture.

\subsection{slide-200}
As the findings I just describes imply,
object indexes can survive brief occlusion.
That is, an object index
can remain attached to an object even if that
object is briefly occluded by a screen.
(Sameness of object index may be detected by the presence of an
object-specific preview benefit).

\subsection{slide-201}
To clarify terminology,
I should say that whereas I’m talking about object indexes,
researchers more typically interpret this research in terms of object
files.
I’m sticking to object indexes rather than object files for
reasons of simplicity and caution.
If you believe in object files then you can interpret what I’m saying
as referring to object files.
And if you have doubts about object files, you might still have reason
to accept that a system of object indexes exists.

\subsection{slide-202}
So far I have been talking about object indexes in adult humans.

\subsection{slide-203}
But our interest in object indexes stems from a Hypothesis about
four-month-old infants’
abilities to track briefly occluded objects.

\subsection{slide-204}
According to this hypothesis, these abilities depend on a system of
object indexes like that which underpins multiple object tracking or
object-specific preview benefits
\citep{Leslie:1998zk,Scholl:1999mi,Carey:2001ue,scholl:2007_objecta}.
What makes this hypothesis attractive?

Hypothesis:



Tracking occluded objects depends on object indexes.



(And reaching for endarkened objects depends on motor representations of objects.)


Several considerations favour the hypothesis about object indexes ...

\subsection{slide-205}
One reason the hypothesis seems like a good bet is that object
indexes are the kind of thing which could in principle explain
infants’ abilities to track unperceived objects because object indexes
can, within limits, survive occlusion.

\subsection{slide-206}
If we consider six-month-olds, we can also find behavioural markers
of object indexes in infants \citep{richardson:2004_multimodal} ...

\subsection{slide-207}
... and there are is also a report of neural markers too \citep{kaufman:2005_oscillatory}.

(\citet{kaufman:2005_oscillatory} measured brain activity in
six-month-olds infants as they observed a display typical of an object
disappearing behind a barrier.
They found the pattern of brain activity characteristic of maintaining
an object index.
This suggests that in infants, as in adults, object indexes can attach
to objects that are briefly unperceived.)

\subsection{slide-208}
The evidence we have so far gets us as far as saying, in effect, that someone capable of committing a murder was in the right place at the right time.
Can we go beyond such circumstantial evidence?

\subsection{slide-209}
The key to doing this is to exploit signature limits.
\citet{carey:2009_origin} argues that what I am calling the signature
limits of object indexes in adults are related to signature limits on
infants’ abilities to track briefly occluded objects.

\subsection{slide-210}
To illustrate, a moment ago I mentioned that one signature limit of
object indexes is that featural information sometimes fails to influence how objects are assigned in ways that seem quite dramatic.

\subsection{slide-211}
There is evidence that, similarly, even 10-month-olds will sometimes
ignore featural information in tracking occluded objects
\citep{xu:1996_infants}.%
\footnote{
This argument is complicated by evidence that infants around 10 months of age do not always fail to use featural information appropriately in representing objects as persisting \citep{wilcox:2002_infants}.
In fact \citet{mccurry:2009_beyond} report evidence that even five-month-olds can make use of featural information in representing objects as persisting \citep[see also][]{wilcox:1999_object}.
%they use a fringe and a reaching paradigm.  NB the reaching is a problem for the simple interpretation of looking vs reaching!
Likewise, object indexes are not always updated in ways that amount to ignoring featural information \citep{hollingworth:2009_object,moore:2010_features}.
It remains to be seen whether there is really an exact match between the signature limit on object indexes and the signature limit on four-month-olds’ abilities to represent objects as persisting.
The hypothesis under consideration---that infants’ abilities
to track briefly occluded objects depend on a system of
object indexes like that which underpins multiple object tracking or
object-specific preview benefits---is a bet on the match being exact.
}

\subsection{slide-212}
Here are the results.
The central column shows that infants looked longer when they saw
two objects at test rather than when they saw a single object.
This is not different from how they performed in a base line condition
when the information about number was not present.
And it is different from how they performed in the ‘spatiotemporal
condition’ in which the two objects were at simultaneously
visible at one point before the test phase.

\subsection{slide-213}
While I wouldn’t want to suggest that the evidence on
siganture limits is
decisive, I think it does motivate considering the hypothesis and its
consequences.
In what follows I will assume the hypothesis is true:
infants’ abilities
to track briefly occluded objects depend on a system of
object indexes.

\subsection{slide-214}
The hypothesis has an advantage which I don’t think is widely
recognised.
This is that object indexes are independent of beliefs and knowledge
states.
Having an object index pointing to a location is not the same thing
as believing that an object is there.
And nor is having an object index pointing to a series of locations over time
is the same thing as believing or knowing that these locations
are points on the path of a single object.
Further, the assignments of object indexes do not invariably give rise
to beliefs and need not match your beliefs.

\subsection{slide-215}
To emphasise this point, consider once more this scenario
in which a patterned square disappears behind the barrier; later a
plain black ring emerges.  You probably don't believe that they are
the same object, but they probably do get assigned the same object index.
Your beliefs and assignments of object indexes are inconsistent in this
sense: the world cannot be such that both are correct.

\subsection{slide-216}
So assignments of object indexes can conflict with beliefs.
Why is this an advantage?

At the start of this talk I emphasised the variety of evidence
which shows that infants, from four months of age or earlier,
can track briefly occluded objects.
However there is also a substantial body of evidence which suggests that
infants of this age, and even infants who are several months older,
systematically fail to search for briefly occluded objects.

\subsection{slide-217}
To illustrate, consider an ingenious experiment by \citet{Shinskey:2001fk}.
There was an opaque screen that could rotate between lying flat on the ground and being raised to conceal a toy behind it.
\citeauthor{Shinskey:2001fk} also used a second piece of apparatus just like the first except that the screen was transparent rather than opaque.
They reasoned that infants would quite often pull the screen forwards just for fun, regardless of what is behind it.
However, they also guessed that when infants know there is an interesting toy behind the screen, then they will pull it forwards more often than when they know that there is nothing behind the screen.
This is just what happened when infants were presented with the apparatus involving a transparent screen:
they sometimes pulled the screen forwards when there was no toy behind it, but they pulled it forwards significantly more often when the toy was behind it.
What happened  when infants were presented with the opaque screen?
Here infants pulled the screen forwards no more often when they had observed a toy being placed behind it then when they had observed that there was nothing behind it.
This is evidence that  seven-month-old infants do not know that a toy they have very recently seen hidden behind a screen is behind the screen.
After all, since knowledge guides action we would expect infants who know that a toy is behind an opaque screen to pull the screen forward more often than infants who know there is nothing behind the screen, just as they do when the screen is transparent.

More than two decades of research strongly supports the view that
infants fail to search for objects hidden behind barriers or screens
until around eight months of age \citep[p.\ 202]{Meltzoff:1998wp} or
maybe even later \citep{moore:2008_factors}.
Researchers have carefully controlled for the possibility that infants’
failures to search are due to extraneous demands on memory or the
control of action.
We must therefore conclude, I think, that four- and five-month-old
infants do not have beliefs about the locations of briefly occluded
objects.
It is the absence of belief that explains their failures to search.

\subsection{slide-219}
Let me summarise ...

\subsection{slide-221}
Why do 5 month olds fail to manifest their ability to track briefly
occluded objects by initiating searches for them after they have been
fully occluded?  Because object indexes are independent of beliefs
and do not by themselves support the initiation of action.
Further, I guess that occlusion interferes
with motor representations  of objects in infants because occlusion
involves two objects, one in front of the other.

\subsection{slide-222}
Why do 5 month olds fail to manifest their ability to track endarkened
objects on v-of-e experiments?  Because endarkening interferes with object
indexes; and although endarkening does not eliminate motor representations,
I guess that these representations do not generally influence looking times.

\subsection{slide-223}
Why do infants succeed in searching for momentarily endarkend objects?
Because they can represent objects motorically, and endarkening does
not immediately interfere with such representations.

\subsection{slide-224}
Why do infants manifest an ability to track briefly occluded objects on
violoation-of-expectations tasks?
Ah ... just here we face a significant challenge ...

As I said at the start of this talk, infants’ abilities to track briefly
occluded objects are manifested in several different ways.
They are manifested in (iii) anticipatory looking, (ii) reactions
indicating the violation of an expectation, and (i) dishabituation
indicating interest in certain stimuli.

Can all of these behaviours be explained merely by invoking object
indexes?

\subsection{slide-225}
This is an important question for me so I want to pause to emphasise it.
This question is, What can the operations of a system of
object indexes explain?

\subsection{slide-226}
The primary functions of object indexes include influencing the allocation
of attention and perhaps guiding ongoing action.
If this is right, it may be possible to explain anticipatory looking
directly by appeal to the operations of  object indexes.
But the operations of object indexes cannot directly explain differences
in how novel things are to an infant.
And nor can the operations of object indexes directly explain why infants
look longer at stimuli involving discrepancies in the physical behaviour
of objects.

\subsection{slide-227}
To illustrate this point, recall this famous violation-of-expectations
experiment by \citet{wynn:1992_addition}.
Her subjects were five-month-olds.

\subsection{slide-230}
We know that infants are likely to maintain object indexes for the two
mice while they are occluded.
Accordingly, when the screen drops in the condition labelled
‘impossible outcome’, there is an interruption to the normal
operation of object indexes: infants have assigned two object indexes
but there is only one object.
But why does this cause infants to look longer at in the
‘impossible outcome’ condition than in the ‘possible outcome’ condition?
How does a difference in operations involving object indexes result
in a difference in looking times?

\subsection{unit\_667}


\section{Phenomenal Expectations Connect Object Indexes to Looking Behaviours}

\subsection{slide-232}
So those who, like me, are impressed by the evidence for the hypothesis
that four- and five-month-olds’ abilities to track occluded objects
are underpinned by the operations of a system of object indexes
are left with a question.
The question is, What links the operations of object indexes to
patterns in looking duration?

I’ve just argued that it can’t be beliefs or knowledge states.

\subsection{slide-233}
Now I want to suggest that it is something called a phenomenal expectation.

Before I explain what phenomenal expectations are, let me illustrate the
idea informally.

\subsection{slide-234}
Recall this situation.
Suppose you have seen it a hundred times before, so you know just what
to expect.
Still, the tendancy to expect two objects is on some level barely
diminished, and event in which a single object is revealled
is liable to feel magical in some small way.
This feeling of magic is a phenomenal expectation.

Let me give you two more illustrations [the wire and the face]. ...

\subsection{slide-235}
What is a phenomenal expectation?  Consider a second illustration.

Here is a wire.
Contrast two sensory encounters with this wire. In the first you visually
experience the wire as having a certain shape. In the second you receive an
electric shock from the wire without seeing or touching it.%
\footnote{This illustration is borrowed from Campbell (2002: 133–4); I use it to support a claim weaker than his.}
The first sensory encounter involves perceptual experience as of a property of
the wire whereas, intuitively, the second does not.
I take this intuition to be correct.%
\footnote{
Notice that the intuition is not that the shock involves no
perceptual experience at all, only that the shock does not involve
perceptual experience as of any property of the wire. Notice also that the
intuition concerns what a perceptual experience is as of, and not directly
what is represented in perception. The relation between these two is
arguably not straightforward (compare, e.g., \citet[p.~28]{Shoemaker:1994el} or
\citet[pp.~50--2]{Chalmers:2006xq} on distinguishing representational from
phenomenal content).
}

The intuition is potentially revealing because the electric shock involves
rich phenomenology, and its particular phenomenal character depends in part
on properties of its cause (changes in the strength of the electric current
would have resulted in an encounter with different phenomenal character).
So there are sensory encounters which, despite having phenomenal characters
that depend in part on which properties are encountered, are not perceptual
experiences as of those properties.

\subsection{slide-236}
What is a phenomenal expectation?  Consider a third (and final) illustration.

Here is a face that I hope will seems familiar to most people.
When you see this face, you have a feeling of familiarity.
This feeling of familiarity is not just a matter of belief:
even if you know for sure that you have never encountered the person
depicted here (and trust me, you haven’t), the feeling of familiarity
will persist.
Nor is the feeling a matter of perceptual experience: you can’t
perceptually experience familiarity
any more than you can perceptually experience electricity.

(The face is a composite of Bush and Obama.  It is chosen to illustrate that
the feeling of familiarity is not a consequence of how familiar things
actually are; instead it may be a consequnece of
the degree of fluency with which unconscious processes can identify
perceived items \citep{Whittlesea:1993xk,Whittlesea:1998qj}.
Learning a grammar can also generate feelings of familiarity.
Subjects who have implicitly learned an artificial grammar report feelings
of familiarity when they encounter novel stimuli that are part of
the learnt grammar \citep{scott:2008_familiarity}.
They are also not doomed to treat feelings of familiarity as being
about actual familiarity:
instead subjects can use feeling of
familiarity in deciding whether a stimulus is from that grammar
\citep{Wan:2008_familiarity}.)

I could go on to mention the feeling you have when someone’s eyes are
boring into your back, or the feeling that a name is on the tip of your
tongue.
But let me focus just on the feelings associated with electricity and
with familiarity.
These feelings are paradigm cases of phenomenal expectation.

\subsection{slide-237}
All three examples (the feelings of magic, of electricity and of
familiarity) show that:

There are aspects of the overall phenomenal character of experiences which their subjects take to be informative about things that are only distantly related (if at all) to the things that those experiences intentionally relate the subject to.

To illustrate, having a feeling of familiarity is not a matter of standing
in any
intentional relation to the property of familiarity, but it is something
that we can interpret as informative about famility.

\subsection{slide-238}
Phenomenal expectations are these aspects of experience.

\subsection{slide-239}
Phenomenal expectations can be thought of as sensations in approximately
Reid’s sense.%
\footnote{
\citet{Reid:1785cj,Reid:1785nz}.
Even if you don’t believe that there are sensations in Reid’s sense,
thinking of phenomenal expectations as if they were sensations will
serve to illustrate their characteristic features.
The main points that follow are consistent with several different ways of
thinking about phenomenal expectations.
For instance, you might take the view that
what I am calling phenomenal expectations are
perceptual experiences of the body or of bodily reactions,
or that they involve some kind of cognitive phenomenology.
The essential claim is just that the phenomenal expectations associated with
the operations of object indexes are not constituted by states which involve
intentional relations to any of the things which are assigned an object index.
}

\subsection{slide-240}
Sensations are:
\begin{enumerate}
\item monadic properties of events, specifically perceptual experiences,
\item individuated by their normal causes% %{Tye, 1984 #1744@204}
---in the case of feelings of familiarity, its normal cause is ease of processing
\item which alter the overall phenomenal character of those experiences
\item in ways not determined by the experiences’ contents
(so two perceptual experiences can have the same content while one has a sensational property which the other lacks).
\end{enumerate}

\subsection{slide-241}
An important consequence is that phenomenal expectations can lead to beliefs
only via associations or further beliefs.
They are signs which need to be interpreted by their subjects
(\citealp[Essay~II, Chap.~16, p.~228]{Reid:1785cj}
\citealp[Chap.~VI sect.~III, pp.~164–5]{Reid:1785nz}).
Let me explain.

As a scientist, you can pick out the feeling of familiarity as that
phenomenal expectation which is normally caused by the degree to which
certain processes are fluent.
But as the subject of who has that phenomenal expectation, you do not
necessarily know what its typical causes are.
This is something you have to work out in whatever ways you work out
the causes of any other type of event.

(Contrast phenomenal expectations with perceptual experiences.
Having
a perceptual experience of, say, a wire’s shape, involves standing
in an intentional relation to the wire’s shape; and the phenomenal
character of this perceptual experience is specified by this
intentional relation.%
\footnote{
Compare \citet[p.~380]{Martin:2002yx}:
‘I attend to what it is like for me to inspect the lavender bush through
perceptually attending to the bush itself.’
And \citet[p.~211]{byrne:2001_intentionalism}
‘subject can only discover the phenomenal character of her experience by
attending to the world ... as her experience represents it.’
}
Such perceptual experiences are often held to reveal the wire’s shape to the
subject and so lead directly to beliefs.%
\footnote{
Compare \citet[p.~222]{Johnston:1992zb}:
‘[j]ustified belief … is available simply on the basis of visual perception’;
\citet[p.~143–4]{Tye:1995oa}:
‘Phenomenal character “stands ready … to make a direct impact on beliefs’;
and
\citet[p.~291]{Smith:2001iz}:
‘[p]erceptual experiences are … intrinsically … belief-inducing.’
})

(By contrast, having a phenomenal expectation concerning familiarity or an
physical object’s path does not involve standing in any intentional relation
to these things.
The phenomenal expectation is individuated by its normal causes, rather
than by any intentional relation.
And a phenomenal expectation leads to belief, if at all, only indirectly.
For learning is required in order for the subject to come to a view on
what tends to cause the phenomenal expectation.)

Phenomenal expectations have been quite widely neglected in philosophy and
developmental psychology.
They are a means by which cognitive processes enable perceivers to
acquire dispositions to form beliefs about objects’ properties which are
reliably true.
Phenomenal expectations provide a low-cost but efficient bridge between
non-conscious cognitive processes and conscious reasoning.

\subsection{slide-242}
So my question was how the operations of object indexes might
explain patterns of looking duration in habituation and
violation-of-expectation experiments.
My guess is that some operations of object indexes give rise to
phenomenal expectations, which in turn influence looking durations.

\subsection{slide-243}
\section{Development Is Rediscovery}

This guess gives rise to a further question (which I want to articulate
but won’t attempt to answer).
In asking how the operations of object indexes might give rise to
patterns in looking duration, we have been concerned with what happens a
short interval of time.
But the guess about phenomenal expectations raises a question
about the course of development in the first months or years of life.
Let me explain.

\subsection{slide-244}
In the beginning Spelke and others conjectured that infants’
abilities to track briefly occluded objects were a consequence of their
having core knowledge for objects.
This conjecture is related to the later hypothesis about object indexes.
The idea is that we can further specify the mechanisms that realise
infants’ core knowledge of physical objects by identifying it with
two things: a system of object indexes and a system capable of representing
physical objects motorically.

\subsection{slide-245}
There was always a question about how infants’ core knowledge about
objects might explain the emergence of knowledge knowledge
(that is, knowledge proper) about objects.
Now this question becomes, What is the role of a system of object
indexes in the emergence in development of knowledge of physical
objects?
In short, How do you get from object indexes to knowledge?

Answers to these questions typically assume that core knowledge
provides a conceptual identification of objects and some of their
properties such as location or size,
or else that it involves standing in some kind of intentional
relation to these things.
This is true of
Spelke’s suggestion that mature understanding of objects, number,
and mind derives from core knowledge by virtue of core knowledge
representations being assembled \citep{Spelke:2000nf};
claims by Leslie and others
that modules provide conceptual identifications of their inputs
\citep{Leslie:1988ct};
Karmiloff-Smith’s representational re-description
\citep{Karmiloff-Smith:1992lv};
and Mandler’s claim
that ‘the earliest conceptual functioning consists of a redescription
of perceptual structure’ \citep{Mandler:1992vn}.

\subsection{slide-246}
But recall the guess about phenomenal expectations linking object
indexes to patterns of looking duration.
If this guess is right, then it is not true that
core knowledge provides a conceptual identification of objects.
And it is not true that having core knowledge involves standing in any
kind of intentional relation to objects and their properties.
This makes the question about development particularly
difficult to answer.

It means that rather than assembing or redescribing representations,
development must be a process of rediscovery.

The step from phenomenal expectations to knowledge is like the step
from feeling electric shocks to understanding electricity.
So coming to know simple facts about particular physical
objects may begin with object indexes and the
phenomenal expectations these give rise to, but it does not end there.
Interpreting the phenomenal expectations may involve interacting with
objects, learning to use tools, and perhaps interacting with others
and objects simultaneously.

Coming to know facts about physical objects is a matter of
rediscovering things already implicit in a system of object indexes.
Some might object that development can’t require such rediscovery
because it would be hopelessly inefficient to require things already
encoded to be learnt anew.
But rediscovery is an elegant solution to a practical problem.
If you are building a survival system you want quick and dirty
heuristics that are good enough to keep it alive: you don’t
necessarily care about the truth.
If, by contrast, you are building a thinker, you want her to be
able to think things that are true irrespective of their survival value.
This cuts two ways.
On the one hand, you want the thinker’s thoughts not to be
constrained by heuristics that ensure her survival.
On the other hand, in allowing the thinker freedom to pursue the
truth there is an excellent chance she will end up profoundly
mistaken %(Malebranche?)
or deeply confused %(Hegel?)
about the nature of physical objects.
So you don’t want thought contaminated by survival heuristics and you
don’t want survival heuristics contaminated by thought.  Or, even if
some contamination is inevitable, you want to limit it.
%So you want inferential isolation.
This combination is beautifully achieved by giving your thinker a
system or some systems for tracking objects and their interactions
which appear early in development, and also a mind which allows her
to acquire knowledge of physical objects gradually over months or
years, taking advantage of interactions with objects as well as
social interactions about objects—providing, of course, that the
two are not directly connected but rather linked only very loosely,
via phenomenal expectations.

\subsection{slide-248}
\section{Conclusion}

To conclude,
I started by mentioning the wide variety of evidence that
four- and five-month-olds can track briefly occluded objects.
This evidence raises the question, How do infants do that?
On the leading, best supported hypothesis,
four- and five-month-olds’ abilities to track briefly occluded objects
depend on a system of object indexes like that which underpins multiple
object tracking or object-specific preview benefits.
This hypothesis
also has the virtue of being consistent with the most straightforward
explanation of why infants of this age (four- to five-months) and even older
systematically fail to manually search for occluded objects.
(The explanation is that they lack beliefs about the locations of objects.)

Accepting this hypothesis forces us to confront a question.
How could the operations of object indexes explain patterns in looking
duration?
This question arises because
facts about the operations of object indexes do not themselves
straightforwardly imply anything about how things seem to infants, nor about
what they believe.

The answer, I suggested, is phenomenal expectations.
Much as there are phenomenal expectations associated with the ease or
difficulty of processing a complex stimulus like a face or letter sequence,
so also phenomenal expectations are associated with operations involving
object indexes.
These phenomenal expectations are not intentional relations to
the phyiscal objects whose behaviours normally cause them.
Instead they can be thought of as sensations in roughly Reid’s sense.
So they are monadic properties of perceptual experiences which carry
information about physical objects.

Importantly, phenomenal expectations (like sensations) require
interpretation.
In order to get from a phenomenal expectation to a belief you need to
form a view about what the phenomenal expectation is a sign of.
This requires learning, and your view can change as you learn more.

This has consequences for understanding the emergence in
development of knowledge of physical objects.
Such knowledge is probably a consequence of the (core) system of
object indexes, but on the view I have been defending the two can be only
indirectly related.
Having core knowledge of objects is a matter of having a system of object
indexes.
The system can affect what you believe or know about objects only by way
of phenomenal expectations.
Gaining knowledge proper requires interpreting the phenomenal expectations,
and so is in part a matter of rediscovering information already processed
by your core systems.

\subsection{slide-249}
The question is ...
How do humans first come to know simple facts about particular physical objects?

\subsection{slide-250}
Here’s what we’ve found so far.

We examined how three requirements on having knowledge of physical objects are met.
Knowledge of objects depends on abilities to (i) segment objects, (ii) represent them as
persisting and (iii) track their interactions.
To know simple facts about particular physical objects you need, minimally,
to meet these three requirements.

The second discovery concerned how infants meet these three requirements this.

\subsection{slide-251}
The second was that a single set of principles is formally adequate to
explain how someone could meet these requirements, and to describe
infants' abilities with segmentation, representing objects as persisting
and tracking objects' interactions.

This is exciting in several ways.
\begin{enumerate}
\item That infants have all of these abilities.
\item That their abilities are relatively sophisticated: it doesn’t seem
that we can characterise them as involving simple heuristics or relying
merely on featural information.
\item That a single set of principles underlies all three capacities.
\end{enumerate}

\subsection{slide-252}
2. What is the relation between the model and the infants?

\subsection{slide-253}
The principles of object perception



are not items of knowledge



instead



they characterise the operation of



object indexes (FINSTs, components of mid-level object files)

Their upshot is not knowledge about particular objects and their movements but rather a
perceptual representation involving an object index.
\citep{Leslie:1998zk,Scholl:1999mi,Carey:2001ue,scholl:2007_objecta}.

This amazing discovery is going to take us a while to fully digest.  As a first step, note its
significance for Davidson's challenge about characterising what is going on in the head of the
child who has a few words, or even no words.

[*TODO*] \citep{keane:2006_motion} suggests that `If early vision gathers trajectory
information, that information cannot reliably be utilized to extrapolate in MOT.'  If this is
right, either object files aren't FINSTs and the link between object files and multiple-object
tracking (MOT) is wrong; or else the claim on this slide identifying principles of object
perception and object indexes is wrong!

\subsection{slide-254}
We saw this quote in the first lecture ...

The discovery that the principles of object perception characterise the operation of
object-indexes doesn't mean we have met the challenge exactly.
We haven't found a way of describing the processes and representations that underpin infants'
abilities to deal with objects and causes.
However, we have reduced the problem of doing this to the problem of characterising how
some perceptual mechanisms work.
And this shows, importantly, that understanding infants' minds is not something different from
understanding adults' minds, contrary to what Davidson assumes.
The problem is not that their cognition is half-formed or in an intermediate state.
The problem is just that understanding perception requires science and not just intuition.

\subsection{slide-255}
Return to this amazing discovery.

Let me make some more points about it.

First, it doesn't fully answer our question about the relation between the Principles of Object
Perception and mechanisms in infants.
It tells us that the Principles characterise a certain kind of perceptual process.
This is progress; but we can still ask about the nature of the procesess and representations
involved.  This will become important when we consider knowledge in other domains.

Second, we haven't fully explained the discrepancy between looking and action-based measures for
representing objects as persisting and tracking their causal interactions.
After all, why do these perceptual representations of objects--the object indexes--not guide
purposive actions like reaching and pulling?
This is an issue we shall return to.

Third, it leaves us with a question we didn't have before.
What is the relation between these abilities to segment objects, represent them as persisting
and track their causal interactions and knowledge about objects?
Clearly having an object-index stuck to an object is not the same thing as having knowledge
about the object's location and movements.  (If it were, we'd face just the problems that are
fatal for the Simple View.)
What then is the relation between these things?

This third point is related to an issue about the relation between infant and adult capacities,
one that  I raised at the start of this lecture ...

\subsection{slide-256}
What is the relation between infants' competencies with objects and adults'?
Is it that infants' competencies grow into more sophisticated adult competencies?
Or is it that they remain constant throught development, and are supplemented by quite
separate abilities?

\subsection{slide-260}
The identification of the Principles of Object Perception with object-indexes suggests that
infants' abilities are constant throughout development.
They do not become adult conceptual abilities; rather they remain as perceptual systems
that somehow underlie later-developing abilities to acquire knowledge.

Confirmation for this view comes from considering that there are discrepancies in adults'
performances which resemble the discrepancies in infants between looking and action-based
measures of competence ...
[This links to unit 271 on perceptual expectations ...]

\subsection{slide-261}
Only phenomenal expectations



connect ‘core knowledge’ of objects



to thought.


Phenomenal expectations have been quite widely neglected in philosophy and
developmental psychology.
They are a means by which cognitive processes enable perceivers to
acquire dispositions to form beliefs about objects’ properties which are
reliably true.
Phenomenal expectations provide a low-cost but efficient bridge between
non-conscious cognitive processes and conscious reasoning.

Development is rediscovery.



If you accept my story about phenomenal expectations then
you also face a problem understanding the how the existence of core knowledge
systems can explain the emergence of knowledge in development.


If you accept my story about phenomenal expectations then
you also face a problem understanding the how the existence of core knowledge
systems can explain the emergence of knowledge in development.





%--- end paste
%---------------






\bibliography{$HOME/endnote/phd_biblio}



\end{document}
