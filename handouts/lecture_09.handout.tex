%!TEX TS-program = xelatex
%!TEX encoding = UTF-8 Unicode

\documentclass[12pt]{extarticle}
% extarticle is like article but can handle 8pt, 9pt, 10pt, 11pt, 12pt, 14pt, 17pt, and 20pt text

\def \ititle {Origins of Mind}
 
\def \isubtitle {Lecture 08}
 
\def \iauthor {Stephen A. Butterfill}
\def \iemail{s.butterfill@warwick.ac.uk}
\date{}

%for strikethrough
\usepackage[normalem]{ulem}

\input{$HOME/Documents/submissions/preamble_steve_handout}

%\bibpunct{}{}{,}{s}{}{,}  %use superscript TICS style bib
%remove hanging indent for TICS style bib
%TODO doesnt work
\setlength{\bibhang}{0em}
%\setlength{\bibsep}{0.5em}


%itemize bullet should be dash
\renewcommand{\labelitemi}{$-$}

\begin{document}

\begin{multicols}{3}

\setlength\footnotesep{1em}


\bibliographystyle{newapa} %apalike

%\maketitle
%\tableofcontents




%--------------- 
%--- start paste
\def \ititle {Origins of Mind}
 
\def \isubtitle {Lecture 09}
 
 
 
\
 
 
 
\begin{center}
 
{\Large
 
\textbf{\ititle}: \isubtitle
 
}
 
 
 
\iemail %
 
\end{center}
 
 
 
\section{Action: The Basics}
 
‘by the end of the first year infants are indeed capable of taking the intentional stance (Dennett, 1987) in interpreting the goal- directed behavior of rational agents.’
\citep[p.\ 184]{Gergely:1995sq}
 
‘12-month-old babies could identify the agent’s goal and analyze its actions causally in relation to it’
\citep[p.\ 190]{Gergely:1995sq}
 
'Six-month-olds and 9-month-olds showed a stronger novelty response (i.e., looked longer) on new-goal trials than on new-path trials (Woodward 1998). That is, like toddlers, young infants selectively attended to and remembered the features of the event that were relevant to the actor’s goal.'
\citep[p.\ 153]{woodward:2001_making}
 
‘just as the visual system works to recover the physical structure of the world by inferring properties such as 3-D shape, so too does it work to recover the causal and social structure of the world by inferring properties such as causality’
\citep[p.\ 299]{Scholl:2000eq}
 
 
 
\section{How Do Infants Model Actions?}
 
‘in perceiving one object as having the intention of affecting another, the infant attributes to the object [...] intentions’
\citep[p.\ 14]{Premack:1990jl}
 
‘by taking the intentional stance the infant can come to represent the agent’s action as intentional without actually attributing a mental representation of the future goal state’
\citep[p.\ 188]{Gergely:1995sq}
 
‘to the extent that young infants are limited [...], their understanding of intentions would be quite different from the mature concept of intentions’
\citep[p.\ 168]{woodward:2001_making}
 
 
 
\section{Does Infants’ Model of Action Involve Intentions?}
 
`The expression `the intention with which James went to church' has the outward form of a description, but in fact it
 
...\ % is syncategorematic and
 
cannot be taken to refer to an entity, state, disposition, or event. Its function in context is to generate new descriptions of actions in terms of their reasons; thus `James went to church with the intention of pleasing his mother' yields a new, and fuller, description of the action described in `James went to church'.'
\citep[p.\ 690]{davidson:1963_orig}
 
 
 
\section{Pure Goal Ascription: the Teleological Stance}
 
Csibra \& Gergely's principle of rational action: `an action can be explained by a goal state if, and only if, it is seen as the most justifiable action towards that goal state that is available within the constraints of reality.'\citep{Csibra:1998cx,Csibra:2003jv}
 
(Contrast a principle of efficiency: `goal attribution requires that agents expend the least possible amount of energy within their motor constraints to achieve a certain end' \citep[p.\ 1061]{Southgate:2008el}).
 
`Such calculations require detailed knowledge of biomechanical factors that determine the motion capabilities and energy expenditure of agents. However, in the absence of such knowledge, one can appeal to heuristics that approximate the results of these calculations on the basis of knowledge in other domains that is certainly available to young infants. For example, the length of pathways can be assessed by geometrical calculations, taking also into account some physical factors (like the impenetrability of solid objects). Similarly, the fewer steps an action sequence takes, the less effort it might require, and so infants’ numerical competence can also contribute to efficiency evaluation.’
 
‘when taking the teleological stance one-year-olds apply the same inferential principle of rational action that drives everyday mentalistic reasoning about intentional actions in adults’
(György Gergely and Csibra 2003; cf. Csibra, Bíró, et al. 2003; Csibra and Gergely 1998: 259)
 
‘What it is to be a true believer is to be … a system whose behavior is reliably and voluminously predictable via the intentional strategy.’
\citep[p.\ 15]{Dennett:1987sf}
 
 
 
 
 
%--- end paste
%--------------- 
 
\footnotesize 
\bibliography{$HOME/endnote/phd_biblio}

\end{multicols}

\end{document}