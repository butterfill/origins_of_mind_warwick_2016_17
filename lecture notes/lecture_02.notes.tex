 %!TEX TS-program = xelatex
%!TEX encoding = UTF-8 Unicode

%\def \papersize {a5paper}
\def \papersize {a4paper}
%\def \papersize {letterpaper}

%\documentclass[14pt,\papersize]{extarticle}
\documentclass[12pt,\papersize]{extarticle}
% extarticle is like article but can handle 8pt, 9pt, 10pt, 11pt, 12pt, 14pt, 17pt, and 20pt text

\def \ititle {Origins of Mind: Lecture Notes}
\def \isubtitle {Lecture 01}
%comment some of the following out depending on whether anonymous
\def \iauthor {Stephen A.\ Butterfill}
\def \iemail{s.butterfill@warwick.ac.uk% \& corrado.sinigaglia@unimi.it
}
%\def \iauthor {}
%\def \iemail{}
%\date{}

%\input{$HOME/Documents/submissions/preamble_steve_paper4}
\input{$HOME/Documents/submissions/preamble_steve_lecture_notes}

%no indent, space between paragraphs
\usepackage{parskip}

%comment these out if not anonymous:
%\author{}
%\date{}

%for e reader version: small margins
% (remove all for paper!)
%\geometry{headsep=2em} %keep running header away from text
%\geometry{footskip=1.5cm} %keep page numbers away from text
%\geometry{top=1cm} %increase to 3.5 if use header
%\geometry{bottom=2cm} %increase to 3.5 if use header
%\geometry{left=1cm} %increase to 3.5 if use header
%\geometry{right=1cm} %increase to 3.5 if use header

% disables chapter, section and subsection numbering
\setcounter{secnumdepth}{-1} 

%avoid overhang
\tolerance=5000

%\setromanfont[Mapping=tex-text]{Sabon LT Std} 


%for putting citations into main text (for reading):
% use bibentry command
% nb this doesn’t work with mynewapa style; use apalike for \bibliographystyle
% nb2: use \nobibliography to introduce the readings 
\usepackage{bibentry}

%screws up word count for some reason:
%\bibliographystyle{$HOME/Documents/submissions/mynewapa} 
\bibliographystyle{apalike} 


\begin{document}



\setlength\footnotesep{1em}






%--------------- 
%--- start paste

\title {Origins of Mind \\ Lecture 02}
 
 
 
\maketitle
 
\subsection{slide-3}
The question for this course is ...
We are going to approach this question by examining the evidence from developmental science, 
and identifying philosophical problems created by the evidence.
 
\subsection{slide-4}
A key challenge concerns the nature of mental states and actions in 
children who developing capacities to know simple facts about of the world.
Last week I illustrated this by discussing knowledge of physical objects ...
 
\subsection{slide-5}
When do humans first come to know facts about the locations of objects they are not perceiving?
 
Suppose we ask,
When do humans first come to know facts about the locations of objects they
are not perceiving?
Some evidence points to an early age, perhaps 2.5 months or earlier.
But other evidence points to a much later age, 7 months at the earliest.
 
You might think, this is just an issue about measuring age.
But I want to argue that it points to a deeper problem.
The problem is how to characterise the mental states and actions of 
typically developing infants in their first months of life,
when they can perform some actions for the reason that there is an 
object behind a screen but when this ability is strictly limited.
 
\subsection{slide-6}
If I’m right, we need to rethink fundamental claims about mental states.
These are coded in the Uncomplicated Account of Minds and Actions.
 
But faced with this, we should try to hold on to the Uncomplicated Account
for as long as possible.
There are two ways to do this.
 
\subsection{slide-7}
One possibility is to insist that infants, despite failures to search,
really can have beliefs about, and knowlegde of, the locations of unseen objects.
For all we have seen so far, it might be that this is right.
Perhaps, for example, there is something especially tricky about searching.
Or perhaps there are other studies which show, contra Shinskey and Munakata,
that four months olds really can search for unperceived objects.
This deserves careful investigation.
 
\subsection{slide-8}
Another possibility is to insist that infants, despite apparently demonstrating
intelligent responses to unperceived objects in their looking behaviours,
just can’t act for the reason that there is an object behind a screen.
Their responses are not really intelligent but driven by some more basic process.
 
I want to consider this possibility first.
 
\subsection{slide-9}
We’re going to consider lots of evidence.
As you’ll see, different researchers have quite different theories.
Why?
I could just tell you what I think the balance of scientific evidence
allows us to conclude.
But I want you to learn to evaluate the scientific evidence for yourself.
More importantly, there is uncertainty about what the 
balance of scientific evidence allows us to conclude.
If you look at the research carefully, I thin you’ll find that
no one yet has a fully adequate answer to the question,
When do humans first come to know facts about the locations of objects they
are not perceiving?
So this question presents a significant challenge for us.
 
\subsection{unit\_161}
 
 
\section{Objects vs Features}
 
\subsection{slide-11}
The question for this lecture concerns knowledge of physical objects.
When do humans first come to know simple facts about particular physical objects?
To illustrate, consider the fact that this telephone is located here,
or the fact that this telephone is square.
I take it that no one is born knowing any such facts.
So there was a time when you knew no facts about particular physical objects at all,
and then, sometime later, you came to know some such facts.
How did you make this transition?
How do humans first come to know facts about particular objects?
 
(For the rest of this lecture 
I'll drop the qualifier `physical' since this is all about physical objects as opposed to, 
say, abstract objects like numbers or forms.)
 
\subsection{slide-12}
[features picture]
First, what does knowledge of physical objects involve?
One way to approach this question is by contrasting objects with features.
Physical objects contrast with features in three ways
%
\begin{enumerate}
%
\item physical objects have boundaries whereas mere features do not. (This needs qualifying 
because there is a sense in which we can regard features as having boundaries; but when it 
comes to 
features, the boundaries are merely projections.  To see the contrast, consider that we could 
all be permanently mistaken about the boundaries of a physical object but not about the 
boundaries of a feature---another species could not discover a million years from now that 
humans are wrong about the boundaries of this feature, that they thought it was one feature
whereas really it is two.)
%
\item physical objects persist in a way that features do not.  You cannot ask, concerning a 
feature now, whether this is the same feature as some time ago.  At least, you cannot ask this
in the same sense that you can ask it about a physical objects.
%
\item physical objects can interact with each other in way that features cannot: they
can causally interact.
%
\end{enumerate}
%
Imagine not knowing anything at all about particular physical objects and living in a world
consisting entirely of features.  Nothing interacts, there are only patterns.
And things outside your perceptual field do not exist.
From your point of view, the world is limited your perceptual field now.
 
\subsection{segment\_persist\_interact}
Contrasting features with physical objects suggests three requirements on having any 
knowledge about particular physical objects.
 
\subsection{slide-14}
Knowledge of objects depends on abilities to (i) segment objects, (ii) represent them as 
persisting and (iii) track their interactions.
 
Let's look at each of these in turn.
 
\subsection{slide-15}
How do infants and adults discern where one object begins and another ends?
 
\subsection{slide-16}
[ducks picture]
The way objects are ordinarily arranged in space, so that one occludes parts of another, 
prevents us from doing this in any simple way.
 
\subsection{slide-17}
[features picture]
Recall my imaginary world of features.  In this world there is no principled way of saying 
where one object ends and another begins.
As I said, features differ from genuine objects in not allowing us to make sense of the question of 
whether we are carving them at their joints.
So an ability to segment physical objects is not necessary for knowing anything about
mere features but it is probably necessary for having any knowledge 
concerning particular physical objects.
 
\subsection{segment\_persist\_interact2}
So much for the first requirement (segmentation) ...
 
\subsection{slide-19}
... what about the second requirement, representing objects as persisting?
 
\subsection{slide-20}
When Hannah hides behind the logs and a girl later pops up, we can ask whether it is Hannah 
again or another girl.
That is, we know that objects can persist despite disappering from view---and despite becoming 
entirely imperceptible.
 
\subsection{slide-21}
[features picture]
Contrast features again.  You might see this red feature moving across the scene. 
But suppose it disappears and then, later a similar looking feature appears. 
There is no fact of the matter about whether this is the same feature or a different one.
As I mentioned before, in the case of features we can't make sense of them as persisting over 
time, or as there being interruptions in their presence.
I suppose, then, that to have knowledge concerning physical objects rather than merely 
concerning features, it is necessary to be able to represent objects as persisting
even while unperceived.
 
\subsection{segment\_persist\_interact3}
That was the second requirement, now there's just one more ...
 
\subsection{slide-23}
This is the requirement that you can track objects' interactions.
 
\subsection{slide-24}
Objects causally interact with each other; one pan supports another, two people collide and 
bounce off each other.  Relatedly, objects have counterfactual lives: sometimes you can say,
truly, that if that barrier had not been there, the car would be at the bottom of the valley 
now.
 
\subsection{slide-25}
[features picture]
As I mentioned, this is another respect in which objects are distinct from features.
Features do not causally interact with each other and they do not have counterfactual lives 
either.
This point of contrast suggests that knowledge concerning physical objects as opposed to mere
features requires at least a limited ability to track causal interactions.
 
\subsection{slide-27}
So reflection on how physical objects differ from mere features suggests three minimal 
requirements on having any knowledge at all of facts about particular physical objects.
Knowing things about particular physical objects, unlike knowing things about mere features, 
requires abilities to segment objects, to represent them as persisting, and to track at least 
some of their cauasl interactions.
 
\subsection{slide-28}
As mentioned, the question we'd like to answer is how humans first come to know any facts about
particular physical objects.
Before you know any such facts you live in something like a world of mere features.
In this feature world, nothing persists and there are no causal interactions only patterns.
And nothing exists except in your perceptual fields.
 
Now the question of how humans make this transition to knowing some facts about particular 
physical objects is too hard to face head on.  But we can approach it by asking,
How do humans come to meet the three requirements on knowledge of objects?
 
\subsection{unit\_171}
 
 
\section{Segmentation and the Principles of Object Perception}
 
\subsection{slide-30}
How do humans segment objects?
 
\subsection{slide-31}
Recall that the way objects are ordinarily arranged in space, so that one occludes parts of another, prevents us from doing this in any simple way.
 
\subsection{slide-32}
Infants from 4.5 months of age can use featural information to segment objects.
 
\subsection{slide-33}
In Amy Needham's 1998 study, 4.5 months old infants were shown a display like this.
Featural information---the difference in textures of the objects---suggests that these are two 
separate objects.  But can infants use this information to detect that there are two objects?
 
\subsection{slide-34}
Some infants were then shown the object being moved like this, so that it is clearly two 
separate objects.
 
\subsection{slide-35}
Other infants where shown the object being moved like this.
If infants think there is one object, they should expect the second kind of movement.
But if infants think there are two objects---if, that is, they can use the featural 
information to segment objects---then they should expect the former kind of movement.
What were the results?  ...
 
\subsection{slide-36}
Needham's results are evidence that infants from 4.5 months of age can use featural information 
to segment objects.
 
\subsection{slide-37}
[I need to explain the method used in violation-of-expectations, and to compare it with
the method of habituation.]
A violation-of-expectations experiment involves a pair of events.
Infants are divided into two groups; one group sees one event, the other sees the other event.
(This is the between-subject version; it might also be done within subjects.)
The experimenter measures how long the infants look at each event.
Of interest is whether infants reliably look longer at one of the two events.
If they do, this is interpreted as evidence that this event---the one infants reliably look 
longer at---is in some way interesting to them. 
And, if the events are well chosen, their interest indicates that the event violates an
expectation they have.
In the experiment we are considering, the expectation violated is the expectation that 
the two objects should move separately.
 
\subsection{slide-38}
At this point you might well ask, What is an expectation?
This is an important question but let me postpone it for now.
 
\subsection{slide-39}
To return to Needham's experiment, interestingly, 4.5 month old infants were able to succeed 
even when the point of contact between the two objects was occluded, as in this diagram.
 
\subsection{slide-40}
These are the results for 4.5 month old infants.
 
One further thing: infants can also use shape information in segmenting objects, and shape information appears to trump featural information \citep{needham:1999_role}.
 
\subsection{slide-41}
Can we fully explain how infants segment objects just by appeal to features?
To see why it couldn't be just features that we use to segment objects, consider 
some more cases ...
 
\subsection{slide-42}
Here is an occluded object---a stick behind a box.
 
The movement is enough to convince 4-month-old infants that there is just one stick even 
though they never see its middle \citep{kellman:1983_perception}.
We can discover this by measuring how different displayes cause them to dishabituate.
 
\subsection{slide-43}
After being habituated to this this, 3-month-old infants were shown one of two displays.
 
\subsection{slide-44}
And here are the results (subjects were 3-month-old infants).
 
\subsection{slide-45}
The fact that infants can correctly segment partially occluded objects based on their movements 
already indicates that they can't be thinking about features only.
 
For more evidence, consider this display.
The two parts of the moving object are featurally different.
Despite this, infants expect to see a single connected object behind the block 
(\citealp{kellman:1983_perception}, Experiment 6; \citealp{Spelke:1990jn}).
 
\subsection{slide-46}
Here are the test stimuli (each groups is shown one or the other).
 
\subsection{slide-47}
And here are the results.
 
Subjects in this experiment were 4-month-old infants.
 
So we saw that infants can use featural information to segment objects, 
but the principle of cohension can trump featural indicators of difference.
 
So infants' abilities to segment objects are not based entirely on recognising features.
 
\subsection{slide-48}
If infants do not rely only on features to do this, then
how do infants segment the objects in the displays we've just been seeing?
 
\subsection{slide-49}
\citet{Spelke:1990jn} suggests that infants rely on a set of principles to segment objects.
But what are the principles?
 
\subsection{slide-50}
Recall this diplay with on object moving behind a stationary block.
What kind of principle could be used to identify that the occluded thing is a single object?
 
\citet{Spelke:1990jn} suggests the principle of rigidity.
This principle says that ‘objects are interpreted as moving rigidly if such an interpretation 
exists’
The hypothesis that this principle describes in part how infants segment objects correctly
predicts that they will treat the moving occluded stick as a single object.
 
But rigidity is not the only principle we need to explain how infants segment objects ...
What justifies us in supposing that a rigidly moving object needs to be joined up?
 
What justifies us in supposing that a rigidly moving object needs to be joined up?
 
\subsection{slide-51}
... to answer this question, consider the Principle of Cohension
 
\subsection{slide-52}
Another principle which seems to be involved in segmenting objects is the principle of 
cohension.
According to this principle, ‘two surface points lie on the same object only if the points are 
linked by a path of connected surface points’ \citep{Spelke:1990jn}.
 
\subsection{slide-53}
For example, objects arranged as on your left were percevied by 3-month-olds as two objects, 
whereas infants treated the displays like that on your right as if they were one object.
(This was measured using a habituation paradigm \citep{kestenbaum:1987_perception}.  Infants 
were habituated to the display.  Then either one object's position changed, or both objects' 
positions changed but in such a way as to preserve the overall configuration of the two 
objects.  Infants could show that they perceived the configuration as a single object by 
looking longer when just one object's position changed.)
 
\subsection{slide-54}
Here's a second example using moving rather than static stimuli and a different method: 
reaching rather than looking.
Let me explain the stimuli first.
 
How does the principle of cohension apply to this moving display? 
As we just formulated it, it doesn't seem to.  After all, in both cases all points on the 
stimuli are lnked by a path of connected surface points.
However, the principle should be read as saying more implying that:
‘When two surfaces are separated by a spatial gap (as in Figure 4a) or undergo relative motions 
that alter the adjacency relations among points at their border (as in Figure 4i), the 
surfaces lie on distinct objects’ \citep[p.\ 49]{Spelke:1990jn}.
 
The question is, Do infants segment these objects in accordance with the Principle of Cohesion?
\citet[Experiment 2]{spelke:1989_reaching} used a reaching experiment with 5-month-old infants.
The smaller of the two objects was always closer to the infants.  
Infants should reach more often for the smaller, nearer object when they represent the simuli 
as two separate objects than when they represent it as a single object.
(This is not obvious, but the researchers do justify this claim carefully 
\citep[p.\ 186]{spelke:1989_reaching}.)
So the idea is that by comparing how often 5-month-olds reach for the smaller object, we can
see whether they treat it as a separate object in one case but not the other.
To make this vivid, let me show you their apparatus ...
 
\subsection{slide-55}
Here you can see the infant sitting in front of the two objects which could be made to move
together or separately.
 
\subsection{slide-56}
And here are the results.  You don't need to read the table, I put it here just to mention
that this is a within-subject design.
 
[*explain within- vs between-subject].
 
Overall, infants reached to the smaller, top object more often when they moved in opposite 
directions than when they moved together.
Given the background assumption, this is evidence that infants segmented the objects 
differently depending on their motions, and did so in just the way that adults would
\citep[Experiment 2]{spelke:1989_reaching}.
 
\subsection{slide-57}
\citet{Spelke:1990jn} proposes that our ability to segment objects depends on four principles.
We've already seen two of these in action (rigidy and cohesion), and we will shortly see 
that a further principle is needed, too.
 
\subsection{slide-58}
We've already seen this principle in action.
 
\subsection{slide-59}
Boundedness is just the converse of cohesion.
Strictly speaking, cohension allows us to infer that we have two 
distinct objects, but not to infer that we have a single object---for that, we need boundedness.
So when I was talking a moment ago about the Principle of Cohesion, strictly speaking I was 
also appealing to the Principle of Boundedness.
 
\subsection{slide-60}
We saw an example of the principle of rigidity in action earlier, with the moving stick 
experiment.
 
\subsection{slide-61}
The final Principle, no action at a distance, is a converse to rigidity.
 
\subsection{slide-62}
I don't want to obsess too much about the details of these principles.  
It isn't important that there are exactly four, nor are their precise formulations.
(Surely the principles as stated here are not exactly the principles we need to characterise
how infants segment objects.)
What I want us to focus on is just the fact that we can use a small number of principles to 
characterise how infants segment objects in a way that generates testable predictions,
and these principles have been confirmed.
This motivates us to ask ...
 
What is the status of these principles?
 
Spelke’s position might be put like this:
 
\begin{enumerate}

\item We (as perceivers) start with a cross-modal representation of three-dimensional 
perceptual features which includes their locations and trajectories.

\item Our task is to get from these representations of features to representations of objects.

\item \emph{Descriptive component} We do this as if in accordance with certain principles 
(cohesion, boundedness, rigidity, and no action at a distance).

\item \emph{Explanatory component}  We acquire representations of objects because we apply the 
principles to representations of features and draw appropriate inferences.

\end{enumerate}
 
The key point for our purposes is the explanatory component.
The principles are not supposed to be merely heuristics for describing and predicting infants’ 
performance on preferential looking tasks.
Rather, these principles are supposed to explain why infants look longer at some things than at 
others.
This what motivates the hypothesis that infants know these principles and use them in 
reasoning about objects: unless this hypothesis is true, it’s hard to understand how the 
principles could have explanatory relevance.
 
\subsection{slide-63}
The conjecture that someone can segment and represent physical objects
does not by itself generate readily testable predictions.
Everything depends on which model of physical objects characterises her
phyiscal cognition.
 
So we should ask ...
 
\subsection{slide-64}
1. How do four-month-old infants  model  physical objects?
 
In asking how infants
model physical objects, we are seeking to understand not how physical objects
in fact are but how they appear from the point of view of
an individual or system.
 
The model need not be thought of as something used by the system: it is
a tool the theorist uses in describing what the system is for and
broadly how it works.
This therefore leads us to a second question ...
 
\subsection{slide-65}
2. What is the relation between the model and the infants?
 
\subsection{slide-66}
3. What is the relation between the model and the things modelled (physical objects)?
 
\subsection{slide-68}
The Simple View is inspired by two famous cognitive scientists, Marr and Chomsky.
Marr showed that many visual processes can be described as inferences.
And Chomsky pioneered the idea that humans’ knowledge of language depends on their knowing of a small number of 
principles.
Similarly, Spelke’s suggestion is that human infants (and adults) come to
know facts about particular physical objects by virtue of making inferences
from a small number of principles which they know or believe.
 
What unites these three cases, Spelke on object segmentation, Marr on vision and Chomsky on 
syntax?
It’s that they are straightforwardly cognitivist in appeal to knowledge and inference.
Principles are known, and they are used via a process of inference.
(There’s a nice quote from Fodor underlining this point.)
 
\subsection{slide-69}
‘Chomsky’s nativism is primarily a thesis about knowledge and belief; it aligns problems 
in the theory of language with those in the theory of knowledge.  Indeed, as often as not, 
the vocabulary in which Chomsky frames linguistic issues is explicitly epistemological.  
Thus, the grammar of a language specifies what its speaker/hearers have to know qua speakers 
and hearers; and the goal of the child’s language acquisition process is to construct a 
theory of the language that correctly expresses this grammatical knowledge.’
\citep[p.\ 11]{Fodor:2000cj}
 
\subsection{slide-70}
So what is the status of Spelke’s principles of object perception?
Consider what I shall call the Simple View ...
 
\textbf{The simple view}
The principles of object perception are things that we know or believe, 
and we generate expectations from these principles by a process of inference.
 
The simple view is that the Spelke principles are just known in whatever sense anything is 
known or believed.
(We can't say the principles are known because strictly speaking they are not truths but only
heuristics.)
The simple view isn’t exactly Spelke’s, but it’s a useful starting point for discussion.
 
\subsection{slide-71}
The Simple View is worth considering in its own right because it is so, well, simple.
But our interest in it may be piqued by the fact that   
Spelke herself appears to have accepted the Simple View at one point in her thinking:
 
‘objects are conceived: Humans come to know about an object’s unity, boundaries, and 
persistence in ways like those by which we come to know about its material composition or its 
market value’
\citep[p.\ 198]{Spelke:1988xc}.
 
\subsection{slide-72}
Now you might think that the case for these principles is not yet very strong.
In that case, asking hard questions about their status would hardly be necessary.
So let’s consider further evidence for these principles.
We can do this by turning from segmentation (which was our first requirement on knowledge of 
objects) 
to representing objects as permanent.
 
\subsection{slide-73}
So I’m arguing that infants act for reasons which involve particular physical 
objects well before they are 7 months old.
This is what is pushing us in the direction of ascribing beliefs about
those objects to the infants.
 
\subsection{unit\_181}
 
 
\section{Permanence}
 
\subsection{slide-76}
[*TODO*] Integrate converging findings on anticipatory (predictive) looking 
\citep{rosander:2004_infants}: `The obtained results are in general agreement with the numerous 
habituation studies that have investigated infants' emerging ability to represent temporarily 
occluded moving objects. The individual data show that 9–12-week-old infants begin to predict 
the reappearance of the object towards the end of the centrally occluded trials.'
 
[*TODO*] integrate this on reaching: \citep{vanwermeskerken:2011_anticipatory}
(Interpretation is a bit out there, but it nicely illustrates how occlusion duration
can affect reaching at around 7 months of age.)
 
[*TODO*] integrate this ERP measure of permanence: \citep{kaufman:2005_oscillatory}
 
\subsection{slide-77}
Permanence is a matter of living in a world where things don't go out of existence when unperceived.
 
You may not be perceiving your keys now, but there is a fact of the matter about where they are and you know this.  (If not where they are, then at least you know that there is a fact about where they are.)
 
\subsection{slide-79}
Although segmentation and permanence are conceptually distinct, they are closely related 
because movement is a clue to segmentation and movement sometimes invovles occlusion.
 
This becomes evident if we think about one more principle of object perception, the principle of continuity.
 
\subsection{slide-80}
We easily understand this principle by considering cases that accord with, and violate, it.
 
Here is motion in accord with it.
 
\subsection{slide-81}
Here is one violation of continuity.
 
\subsection{slide-82}
And here is another violation of continuity.
 
\subsection{slide-83}
\citet{spelke:1995_spatiotemporal} tested sensitivity to the principle of continuity in 4-month-old infants.
 
The infants were habituated to one of two displays.
 
\subsection{slide-84}
Now in the continuous event we should perceive one object whereas in the
discontinous event we should perceive two objects. But is this about
segmentation or persistence? Segmentation since it's about distinguishing
one object from another; and persistence since it's about representing
temporarily unperceived objects.
 
\subsection{slide-85}
They were then shown one of two test stimuli.
 
\subsection{slide-86}
The measure was the degree of dishabituation as measured by looking time.
 
\subsection{slide-88}
What's beautiful about these results is that the two groups show opposite patterns of dishabituation.
 
\subsection{slide-89}
Recall that the continuity principle could be violated in two ways.
 
We've just seen a `continuity violation'.  Next I want to show you a solidity violation.
 
\subsection{slide-90}
Further evidence that infants represent unperceived objects from around four months 
includes Baillargeon's famous drawbridge study.
 
These are the test events from Experiment 1 of Baillargeon et al's 1987 study.
 
'The habituation event was exactly the same as the impossible event, except that the yellow box was absent.' (Baillargeon et al 1985, 200)
 
\subsection{slide-93}
These are the results from Experiment 1 of Baillargeon et al's 1987 study.
 
\subsection{slide-94}
I'm presenting this experiment as showing that infants represent objects as persisting, and do so
in accordance with the Principle of Continuity.  However, the experiment is also about 
causal interactions between objects.  After all, infants are demonstrating sensitivity to
the fact that a solid object must stop the drawbridge from rotating all the way back.
 
\subsection{slide-95}
Some have been critical of the methods used in this experiment.
 
\subsection{slide-96}
‘The lack of interaction between rotation angle and presence of a box in
the looking time data is inconsistent with the suggestion of object
permanence in our sample.’ 
\citep[p.~73]{sirois:2012_pupil}
 
‘our use of a factorial design as opposed to collapsing rotation angle and
box in a single pair of test events clarifies the picture.’
\citep[p.~74]{sirois:2012_pupil}
 
\subsection{slide-97}
So Baillargeon’s drawbridge study doesn’t demonstrate object permanence?
 
\subsection{slide-98}
Things are rarely so straightforward.
\citeauthor{sirois:2012_pupil} used computer generated stimuli whereas Baillargeon had a physical set-up,  they studied 10-month-olds rather than 4-month-olds, and they used a different method (‘ children were ... not habituated by the time testing began’).
So what can we conclude from the fact that \citeauthor{sirois:2012_pupil}
did not find evidence for an ability to represent objects as persisting?
This certainly justifies caution in relying on any single experiment.
Taken alone, \citeauthor{baillargeon:1987_object}’s
(\citeyear{baillargeon:1987_object}) studies are inspiring but not fully
convincing.
However many further experiments involving different groups of researchers,
different scenarios and different methods provide converging evidence for
the same conclusion: even four-month-olds can represent objects as
persisting (for reviews see \citealp{Spelke:2001pg} or
\citealp{Baillargeon:2002hb}).
The initial, groundbreaking studies are probably methodologically
imperfect, but the balance of evidence from subsequent experiments suggests
that the discovery they illuminate is probably real.%
\footnote{
For an opposing view see \citet{schoner:2006_using}; for critical
discussion of measures involving looking times generally, see
\citet{aslin:2007_whats}.
}
 
Whatever your views on this experiment,
not everything hangs on it.
Fortunately there are at least a hundred further experiments which
provide evidence pointing in the same direction.
Here we'll look at just one more experiment.
 
\subsection{slide-99}
Here is another way of demonstrating object permanence.
 
This experiment will suggest, incidentally, that the principles we have seen---continuity,
rigidity and the rest---don't fully explain how infants succeed in representing objects as persisting.
 
The subjects were 4 month old infants.
 
They were shown a large object disappearing inside a small conatiner, or behind a narrow screen.
 
\subsection{slide-101}
The experiment was very simple.
 
All the experimenters did was measure how long infants looked in at the two events.
 
Infants looked longer at the narrow-occulder event.
 
\subsection{slide-102}
There was also a control condition.
 
In the control condition, infants saw a small rather than a large object.
 
\subsection{slide-103}
Here’s the experimental condition again for comparison.
 
\subsection{slide-104}
And here's the control condition again.
 
\subsection{slide-105}
As you can see, there was a difference in looking times only in the experimental condition.
 
By the way, this experiment is interesting partly because it doesn't use habituation, as 
Baillargeon's earlier drawbridge experiment did.
It is also hard to explain the result by appeal only to the Principle of Object Perception
that we have so far listed.
 
\subsection{slide-106}
We're considering abilities to represent objects as persisting even when not perceived. 
Where are we?
We've seen that characteristing these abilities in terms of Principles of Object Perception
enables us to make testable predictions, many of which have been confirmed.
Importantly, we made the same claim about these Principles for abilities to segment objects.
The abilities to segment objects and to represent them as persisting are conceptually 
distinct.
However it may be that beliefs about a single set of principles underlies both abilities.
This is one of Spelke's brilliant insights.
 
Where does this leave us?
We still want to know about the status of the principles of object perception.
As I said before, it is one thing to say they are descriptively adequate and another thing
to understand how the Princples relate to cognitive mechanisms (processes and 
representations).
But now the question about the status of these Principles is more pressing because 
the claim that these principles of object perception explain infants' (and adults', 
and other primates') performance is now harder to reject.
It's harder to reject because we have converging evidence for the psychological reality of 
the principles from both segmentation and permanence.
 
\subsection{slide-109}
On the status of the Principles, consider this claim about the interpretation of the 
results of a violation-of-expectation experiment:
 
‘evidence that infants look reliably longer at the unexpected than at the expected event is 
taken to indicate that they (1) possess the expectation under investigation; (2) detect the 
violation in the unexpected event; and (3) are surprised by this violation. The term 
surprise is used here simply as a short-hand descriptor, to denote a state of heightened 
attention or interest caused by an expectation violation.’ \citep[p.\ 168]{wang:2004_young}
 
\subsection{slide-110}
What does ‘surprise’ mean here?
 
\subsection{slide-111}
So this is not surprise in a sense that requires awareness of a change in one's own beliefs.
It is rather that there is a particular way in which the detection of the violation is 
manifested.
 
\subsection{slide-112}
Note that we are talking about expectations.
This raises two questions: How do we arrive at these expectations? and What is an 
expectation?
Spelke's claim is that we arrive at these expectations by inference from the Principles of 
Object Perception, including the principle of contintuity.
So what is an expectation?  
On the simple view we are adopting for now, an expectation is just a belief.
The attraction of this simple view is it allows us to take literally the claim that we know 
the principles of object perception and arrive at expectations by a process of inference.
 
\subsection{slide-113}
Here is an illustration of the Simple View ...
 
‘To make sense of such results [i.e. the results from violation-of-expectation tasks], we … must assume that infants, like older learners, formulate … hypotheses about physical events and revise and elaborate these hypotheses in light of additional input.’
 
\subsection{slide-114}
So infants formulate hypotheses
 
\subsection{slide-115}
And infants revise and elaborate these hypotheses in light of additional input.
Now you might suggest that these researchers in talking about formulating and revising
hypotheses do not mean to suggest that infants are doing this in the sense that you or I 
might, and so do not mean to imply that they have beliefs or knowledge.  But ...
 
\subsection{slide-116}
... they explicitly specify that infants do this ‘like older learners’.
 
\subsection{slide-117}
So our current working hypothesis about the Principles is the Simple View.
But before we go any further, let me say a little more about the third thing on our list, 
causal interactions ...
 
\subsection{slide-118}
We are far from fully understanding how humans are first able to represent
objects as persisting.
However, the fact that the ability appears so early in development entails
that it does not demand language, nor much conceptual sophistication.
This view is supported by the fact that the ability to represent objects as
persisting is found in a wide variety of nonhuman animal including
monkeys \citep{santos:2006_cotton-top},
lemurs \citep{deppe:2009_object},
dogs \citep{kundey:2010_domesticated}, % replication of Baillargeon’s drawbridge with dogs
wolves \citep{fiset:2013_object}, %this is actually dogs and wolves
cats \citep{triana:1981_object},
crows \citep{hoffmann:2011_ontogeny},
chicks \citep{chiandetti:2011_chicks_op},
and 
dolphins \citep{jaakkola:2010_what}.%
\footnote{
If you read these studies you will find that some of the authors talk about
Piaget's stages of object permanence, and about visible and invisible
displacements.
For our purposes few of these details matter;
% $glossary:object permanence
the main thing you need to know is just that having \emph{\index{$object permanence$}object permanence} is being able to represent objects as persisting even when they are briefly hidden from your view.
}
It is possible that humans’ abilities to represent objects as persisting
are unrelated to some or all of these animals’, of course.
Nevertheless, the fact that chicks can represent objects as persisting does
show that doing this is not necessary something that requires much
cognitive effort or conceptual sophistication.
 
\begin{enumerate}
 
\item monkeys \citep{santos:2006_cotton-top}
 
\item lemurs \citep{deppe:2009_object}
 
\item crows \citep{hoffmann:2011_ontogeny}
 
\item dogs and wolves \citep{fiset:2013_object}
 
\item cats \citep{triana:1981_object}
 
\item chicks \citep{chiandetti:2011_chicks_op}
 
\item dolphins \citep{jaakkola:2010_what}
 
\item ...
 
\end{enumerate}
 
(Wolves matter because their performing similarly to dogs that show dogs' performance probably isn't a consequence of domestication, as \citet{fiset:2013_object} argue.)
 
Most of these animals have been tested using search as the measure, rather than looking times.
 (This will be important later.)
 
 
 
Note also that many of these studies contrast visible with invisible displacements, or talk about Piaget's stages of object permanence.  For simplicity, that's not something I'm covering.
 
\subsection{slide-119}
[Aside] Comparative research is hard.
 
\subsection{unit\_201}
 
 
\section{Causal Interactions}
 
\subsection{slide-123}
The third requirement on knowledge of objects is an ability to track
objects through causal interactions.
 
Here we're interested in very simple causal interactions, such as the
collision of two balls or the interaction of a ball with a barrier.
 
\subsection{slide-124}
Consider this case where a ball falls and lands on a bench.
 
Suppose that there was a barrier in front of the bench, like the dotted line.
 
Because the bench protrudes from the barrier, you could easily see where the ball will land.
 
But of course you can only see this if you know that barriers stop solid balls.
 
Spelke used this observation to provide evidence that 4-month-old infants can track objects' causal interactions.
 
Infants were habituated to a display in which a ball fell behind a screen, the screen came forwards and the ball was revealed to be on the ground, just where you'd expect it to be.
 
After habituation infants were shown one of two displays.
 
Infants in the 'consistent group' were shown this.
 
Whereas infants in the 'inconsistent group' were shown this.
 
What should we predict?
 
If infants were only paying attention to the shapes and ignoring properties like solid, they should have dishabituated more to the consistent than to the inconsistent stimlus.
 
After all, that stimlus is more different from the habituation stimulus in terms of the surfaces.
 
But if infants were are to track some simple causal interactions, then they might dishabituate to the 'inconsistent' stimulus more than to the 'consistent' stimulus because that one involves an apparent violation of a physical laws.
 
\subsection{slide-125}
Here are the results.
 
(Recall that the subjects are 4-month-old infants.)
 
This is evidence that infants can track causal interactions among objects, even when those causal interactions are occluded.
 
\subsection{slide-126}
Chimpanzees also understand something of phyiscal interactions insofar as their looking times show sensitivity to support relations \citep{cacchione:2004_recognizing}.
 
\subsection{slide-127}
Here are the results.
 
Lots of studies like this have been done with infants in their first six months of life.
 
\subsection{slide-128}
Dogs can do this too.
 
This experiment used a search measure rather than a looking time measure.
 
Dogs had to retrieve a treat.  The right location to search depended on whether the barrier was present or absent.
 
\subsection{slide-129}
The results show brilliant performance.
 
'Dogs correctly searched the near location when the barrier was present and the far location when the barrier was absent. They displayed this behavior from the first trial' \citep{kundey:2010_domesticated} (from the abstract).
 
\subsection{slide-130}
How do infants, adult humans and nonhumans track causal interactions among objects (including causal relations like support)?
 
Spelke suggests that the principles of object perception can explain this.
 
\subsection{slide-131}
For example, the position of an object falling onto a bench is predicted by the principle of continuity mentioned earlier.
 
\emph{Principle of continuity} An object traces exactly one connected path over space and time \citep[p.\ 113]{spelke:1995_spatiotemporal}.
 
(The other principles of object perception are on your handout.)
 
\subsection{slide-134}
This is Spelke's brilliant insight.
 
I think there's something here that should be uncontroversial, and something that's more controversial.
 
But first let me recap ...
 
\subsection{unit\_206}
 
 
\section{Recap and Questions}
 
\subsection{slide-136}
As I said at the start, knowledge of objects depends on abilities to (i) segment objects, (ii) represent them as persisting and (iii) track their interactions.
 
\emph{Question 1}  How do humans come to meet the three requirements on knowledge of objects?
 
Until quite recently it was held, following Piaget and others, that these three abilities appeared relatively late in development.
 
However, as we saw last week, more recent investigations provide strong evidence that all three abilities are present in humans from around four months of age or earlier.
 
Infants' looking behaviours indicate that they have expectations concerning segmentation, persistence and causal interactions.
 
\emph{Discovery 1} Infants manfiest all three abilities from around four months of age or earlier.
 
\subsection{slide-137}
We've seen that infants' abilities to segement objects, represent them as persisting and track their causal 
interactions can be described by appeal to a single set of principles, 
the principles of cohension, boundedness, rigidity and no action at a distance.
 
This suggests that 
\emph{Discovery 2} Although abilities to segment objects, to represent them as persisting through occlusion and  to track their causal interactions are conceptually distinct, they may all be consequences of a single mechanism (in humans and perhaps in other animals).
 
Spelke suggests, further, that these principles of object perception explain infants' looking behaviours.
 
This means we must ask
\emph{Question 2} What is the relation between the principles of object perception and infants’ looking behaviours?
 
Let me explain this question.
 
\subsection{slide-140}
formal adequacy:
 
Let's suppose that Spelke is right that the principles are \emph{formally adequate}.
That is, someone who knew the principles and had unlimited cognitive
resources could use the principles to infer the track physical objects
through simple causal interactions like those we've been considering.
(So formal adequacy is a question of what is possible in principle.)
I don't think we should question this.
 
\subsection{slide-141}
descriptive adequacy
 
I also want to allow that Spelke's principles are \emph{descriptively adeqaute}.
That is, they successfully describe how infants, adults and nonhumans
deal with various situations.
We can think of this in terms of \emph{as if}: it is as if these
individuals are using the principles.
But we have yet to come to what really matters to Spelke and to us.
For accepting formal and descriptive adequacy is consistent with denying
that Carey and Spelke's claim that ‘A single system of knowledge …
appears to [does] underlie object perception and physical reasoning’
\citep[p.\ 175]{Carey:1994bh}.
 
That's because formal and descriptive adequacy leave open the question of
what mechanisms are involved in tracking physical objects' causal
interactions.
 
\subsection{slide-142}
mechansim:
Finally, we might claim that these principles are realised in the
cognitive mechansisms invovled in object tracking.
It just here that we have to face the second question, What is the relation between the principles of object perception and infants’ looking behaviours?.
(e.g. the simple view)
In answer to Q2, I suggested that we start with the simple view.
The \emph{simple view} is the view that the principles of object perception are things that we know or believe, and we generate expectations from these principles by a process of inference.
The attraction of the Simple View is that it promises to explain infants'
sensitivity to objects' boundaries, their persistence and their causal
interactions as manifested in a variety of looking behaviours.
But, as we're about to see, the Simple View is completely wrong.
 
\subsection{unit\_207}
 
 
\section{A Problem}
 
\subsection{slide-144}
As just mentioned, the Simple View is the view that the principles of object perception are things that we know or believe, and we generate expectations from these principles by a process of inference..
 
\subsection{slide-145}
Why must we reject the simple view?
 
\subsection{slide-146}
Some philosophers have offered intuitive arguments against the Simple View.
\citet{Bermudez:2003dj}, for instance, holds that those without the
ability to use a language cannot make inferences;
and \citet{Davidson:1975eq} holds that those without language cannot
think at all.
It may be hard to accept that four-month-old infants are in the business
of inferring truths about particular objects’ locations from abstract
principles.
(And perhaps it is no less hard to accept that adults typically do this
in segmenting objects.)
But scientific and mathematical discoveries sometimes require us to
reject intuitions, even intuitions about very fundamental things like
space and time.
For this reason there seems to be slim prospect of effectively
challenging the Simple View on the basis of intuitions about the nature
of knowledge, belief and inference.
Doing so is also unnecessary as there are scientific reasons for
rejecting the simple view.
 
\subsection{slide-147}
I think we shouldn't try to challenge the simple view on the basis of intution.
 
\subsection{slide-148}
And we don't need to because there are also scientific reasons for rejecting the simple view.
 
One set of reasons concerns the apparent discrepancy between looking times and manual search ...
 
\subsection{slide-149}
*(The basic idea is to say there's a discrepancy regarding BOTH (a) permanence and (b) 
causal interactions)
 
\subsection{slide-150}
Recall this experiment which used habituation to demonstrate infants' abilities to represent
objects as persiting while unperceived (in this case, because occluded).
Infants can do this sort of task from 2.5 months or earlier \citep{Aguiar:1999jq}.
 
But what happens if instead of measuring how infants look, we measure how they reach?
 
\subsection{slide-151}
\citet{Shinskey:2001fk} did just this.
Here you can see their appratus.
They had a screen that infants could pull forwards to get to an object that was sometimes
hidden behind it.
They made two comparisons.
First, were infants more likely to pull the screen forwards when an object was placed behind it?
Second, were how did infants' performance compare when the barrier was not opaque but transparent?
 
\subsection{slide-152}
Here are their results with 7-month old infants.
 
\subsection{slide-153}
Now we have the beginnings of a problem.
The problem is that, if the Simple View is right, infants should succeed in tracking persisting
objects regardless of whether we measure their eye movements or their reaching actions.
But there is a gap of around five months between looking and reaching.
 
The attraction of the simple view is that it explains the looking.
The problem for the simple view is that it makes exactly the wrong prediction about the reaching.
 
Can we explain the discrepancy in terms of the additional difficulty of reaching?
A lot of experiments have attempts to pin the discrepancy on this, or on other extraneous factors like task demands.
But none of these attempts have succeeded.
After all, we know infants are capable of acting because they move the transparent screen.
 
\subsection{slide-154}
As Jeanne Shinskey, one of the researchers most dedicated to this issue says,
 
\subsection{slide-155}
If there were just one discrepancy, concerning performance, we might be able to hold on to the
Simple View.  But there are systematic discrepancies along these lines.
 
Related discrepancies concerning infants' understanding of physical objects occur in the case
of their abilities to track causal interactions, too.
 
\subsection{slide-156}
Recall this experiment about causal interactions, which used a habituation paradigm.
Now imagine a version which involved getting infants to reach for the object rather than simply looking.
What would the results be?
There is an experiment much like this which has been replicated several times, and which shows 
a discrepancy between looking and searching.
Basically infants will look but not search.
 
\subsection{slide-157}
*todo
 
\subsection{slide-158}
*todo
 
\subsection{slide-159}
*todo
 
\subsection{slide-160}
Here are the looking time results.
 
\subsection{slide-161}
You can even do looking time and reaching experiments with the same subjects and apparatus \citep{Hood:2003yg}.
 
2.5-year-olds look longer when experimenter removes the ball from behind the wrong door, but don't reach to the correct door
 
\subsection{slide-162}
here are the search results (shocking).
 
\subsection{slide-163}
*todo: describe
 
**todo: Mention that \citep{mash:2006_what} show infants can also predict the location of the object (not just identify a violation, but look forward to where the object is)
 
\subsection{slide-164}
Amazingly, 2 year old children still do badly when only the doors are opaque, so that the
ball can be seen rolling between the doors, as in this diagram \citep{Butler:2002bv}.
 
\subsection{slide-166}
Similar discrepancies between looking and reaching are also found in some nonhuman primates,
 
both apes and monkeys (chimpanzees, cotton-top tamarins and marmosets).
 
(Some of this is based on the gravity tube task and concerns gravity bias.)
 
‘A similar permanent dissociation in understanding object support relations  
might exist in chimpanzees. They identify impossible support relations in looking tasks, 
but fail to do so in active problem solving.’
\citep{gomez:2005_species}
 
\subsection{slide-168}
Note that this research is evidence of dissociations between looking and search in adult primates, not infants.
 
This is important because it indicates that the failures to search are a feature of the core knowledge system rather than a deficit in human infants.
 
‘to date, adult primates’ failures on search tasks appear to 
exactly mirror the cases in which human toddlers perform poorly.’
\citep[p.\ 17]{santos:2009_object}
 
\subsection{slide-169}
What about the chicks and dogs?
This isn't straightforward.
As I mentioned earlier, \citep{kundey:2010_domesticated} show that domestic dogs are good 
at solidity on a search measure.  And as we covered in seminars, 
\citep{chiandetti:2011_chicks_op} demonstrate object permanence with a search measure in 
chicks that are just a few days old.
Indeed, for many of the other animals I mentioned, object permanence is measured in search 
tasks, not with looking times.
To speculate, it may be that the looking/search dissociation is more likely to occur in
adult animals the more closely related they are to humans.
But let's focus on the fact that you get the looking/search in any adult animals at all.
This is evidence that the dissociation is a consequence of something fundamental about 
cognition rather than just a side-effect of some capacity limit.
 
\subsection{slide-173}
So far we can draw two conclusions about infants' and adults abilities to track 
interactions.  My \textbf{first conclusion} from this section is that infants from around 
4 months of 
age or younger and nonhuman animals are able to track simple causal interactions.
 
\subsection{slide-174}
I started by identifying three requirements for knowledge of physical objects: 
abilities to segment objects, to represent them as persisting, and to track their causal 
interactions.
My \textbf{second conclusion} is that a single set of principles likely underlies these 
abilities.  The ability to segment objects is bound up with the ability 
to represent them as persisting and with the ability to track their interactions.
 
\subsection{slide-175}
My \textbf{third conclusion} is that we have a problem.
The problem is that we have to reject the simple view.
Recall that the simple says that the principles of object perception are things that we know 
or believe.
We must reject this view because it makes systematically incorrect predictions about actions 
like searching for objects.
 
But why is this a problem? Because, as we'll see, it is hard to identify an alternative.
 
\subsection{unit\_211}
 
 
\section{Like Knowledge and Like Not Knowledge}
 
\subsection{slide-177}
I'm sorry to keep repeating this but I want everyone to understand where we are.
 
There are principles of object perception that explain abilities to segment objects, to represent them while temporarily unperceived and to track their interactions.
 
These principles are not known.  What is their status?
 
\subsection{slide-178}
The problem is quite general.
It doesn't arise only in the case of knowledge of objects but also in other domains 
(like knowledge of number and knowledge of mind).
And it doesn't arise only from evidence about infants or nonhuman primates; it would also 
arise if our focus were exclusively on human adults.
More on this later.
For now, our aim is to better understand the problem as it arises in the case of knowledge of objects.
 
\subsection{slide-179}
One hopeful alternative is to shift from talk about knowlegde to talk about representation.
 
Will this help?
 
Only as a way of describing the problem.
 
We need to say what we mean by representation.
The term is used in a wide variety of ways.
As I use it, representation is just a generic term covering knowledge, belief and much else besides.
 
\subsection{slide-180}
If we are going to substitute representation for knowledge, 
we need to characterise what kind of representation we have in mind.
The term is tricky.
As \citet{Haith:1998aq} says, ‘no concept causes more problems in discussions of infant cognition than that of representation.’
 
\subsection{slide-181}
Take a paradigm case of representation.
 
\subsection{slide-183}
The subject might not be the agent but some part of it.
 
That is, we can imagine that some component of an agent, like her perceptual system or motor system, represents things that she herself does not.
 
(Of course, to make sense of this idea we need to invoke some notion of system.)
 
\subsection{slide-185}
The content is what distinguishes one belief from all others, or one desire from all others.
 
The content is also what determines whether a belief is true or false, and whether a desire is satisfied or unsatisfied.
 
There are two main tasks in constructing a theory of mental states.
 
The first task is to characterise the different attitudes.
 
This typically involves specifying their distinctive functional and normative roles.
 
The second task is to find a scheme for specifying the contents of mental states.
 
\subsection{slide-187}
The second task is to find a scheme for specifying the contents of mental states.
 
Usually this is done with propositions.
 
But what are propositions?
 
Propositions are abstract objects like numbers.
 
They have more mystique than numbers, but, like numbers, they are abstract objects that can be constructed using sets plus a few other basic ingredients such as objects, properties and possible worlds.
 
\subsection{slide-188}
So that was some quick background on representation.
 
Note that the issue of representation comes up twice for us.
 
There is a question about whether the principles of object perception are represented.
 
And there is a question about whether objects, their locations, properties, and interactions are represented.
 
The problem raised by the discrepancy between looking and acting is a problem for two claims: (i) the simple view (the principles of object perception are knowledge \&c); and also (ii) the claim that the representations of objects which derive from the principles of object perception are knowledge states.
 
\subsection{slide-189}
So to say that we don't know the principles of object perception but only represent them doesn't tell us much.
This is a step in the right direction.
But it tells us that we represent them without knowing them.
 
What we need if we're to have an explanatory answer to Q2a is to know positively how we do represent the principles of object perception --- subject, attitude and content.
We need to characterise a form of representation that is like knowledge but not like knowledge.
 
\subsection{slide-190}
Your handbag is bluging, and when you swing it at me something really hard hits me.
 
It must be full of rocks.
 
Except it can't be because you are not strong enough to lift such a big bag full of rocks.
 
In that case, it must be wrocks not rocks.
 
A wrock is just like a rock except that it lacks mass.
 
Compare: this representation is just like knowledge except that it doesn't guide action; this process is just like inference except that it lacks the normative aspects of inference.
 
\subsection{slide-191}
\citet{Munakata:2001ch} suggests that there are 'graded representations', that is knowlegde can be stronger or weaker.
 
Presupposes we have an account of subject, attitude and content.  Let's grant that.
 
What is strength?  Some additional component, over and above subject, attitude and content.
 
The idea is quite intuitive but difficult to make systematic sense of.
 
The idea might well make sense if we were talking about neural representations.
 
But here we aren't.  Let's not introduce radically new ideas about representation unless we really have to.
 
(By the way, \citet{Munakata:2001ch} is a nice review of dissociations, not only developmental dissociations.)
 
\subsection{slide-192}
Recall what Davidson said: we need a way of describing what is in between thought and 
mindless nature.  This is the challenge presented to us by the failure of the Simple View.
 
 
\subsection{slide-194}
To conclude, the question for this lecture concerned knowledge of physical objects.
 
We examined how three requirements on having knowledge of physical objects are met.
 
Knowledge of objects depends on abilities to (i) segment objects, (ii) represent them as persisting and (iii) track their interactions.
 
We looked at each of these in turn and found evidence for two discoveries.
First infants meet these requirements in the first months of their lives.
 
\subsection{slide-195}
Second, a single set of principles is formally adequate to explain how someone could 
meet these requirements, and to describe infants' abilities with segmentation, 
representing objects as persisting and tracking objects' interactions.
 
This left us with the a question about mechanism: What is the relation between these principles
and infants' competence?
 
\subsection{slide-196}
A natuaral answer is the Simple View: the principles of object perception are things that we know or believe, and we generate expectations from these principles by a process of inference.
 
However, as we saw, the Simple View must be wrong because it generates incorrect predictions.
This was the lesson of the discrepancy bewteen looking and search measures for both 
infants' abilities to represent objects as persisting and their abilities to track causal 
interactions.
 
As I've just been arguing, the failure of the Simple View presents us with a problem.
The problem is to understand the nature of infants' apprehension of the principles given
that it doesn't involve knowledge.
This is a problem that will permeate our discussion of the origins of mind because it 
problems of this form come up again and again in different domains. 
It isn't the only problem we'll encounter, but none of the problems are more important or more
general than this one.
 




%--- end paste
%--------------- 
 





\bibliography{$HOME/endnote/phd_biblio}



\end{document}