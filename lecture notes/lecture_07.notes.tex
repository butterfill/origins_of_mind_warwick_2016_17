% !TEX TS-program = xelatex
%!TEX encoding = UTF-8 Unicode

%\def \papersize {a5paper}
\def \papersize {a4paper}
%\def \papersize {letterpaper}

%\documentclass[14pt,\papersize]{extarticle}
\documentclass[12pt,\papersize]{extarticle}
% extarticle is like article but can handle 8pt, 9pt, 10pt, 11pt, 12pt, 14pt, 17pt, and 20pt text

\def \ititle {Origins of Mind: Lecture Notes}
\def \isubtitle {Lecture 01}
%comment some of the following out depending on whether anonymous
\def \iauthor {Stephen A.\ Butterfill}
\def \iemail{s.butterfill@warwick.ac.uk% \& corrado.sinigaglia@unimi.it
}
%\def \iauthor {}
%\def \iemail{}
%\date{}

%\input{$HOME/Documents/submissions/preamble_steve_paper4}
\input{$HOME/Documents/submissions/preamble_steve_lecture_notes}

%no indent, space between paragraphs
\usepackage{parskip}

%comment these out if not anonymous:
%\author{}
%\date{}

%for e reader version: small margins
% (remove all for paper!)
%\geometry{headsep=2em} %keep running header away from text
%\geometry{footskip=1.5cm} %keep page numbers away from text
%\geometry{top=1cm} %increase to 3.5 if use header
%\geometry{bottom=2cm} %increase to 3.5 if use header
%\geometry{left=1cm} %increase to 3.5 if use header
%\geometry{right=1cm} %increase to 3.5 if use header

% disables chapter, section and subsection numbering
\setcounter{secnumdepth}{-1}

%avoid overhang
\tolerance=5000

%\setromanfont[Mapping=tex-text]{Sabon LT Std}


%for putting citations into main text (for reading):
% use bibentry command
% nb this doesn’t work with mynewapa style; use apalike for \bibliographystyle
% nb2: use \nobibliography to introduce the readings
\usepackage{bibentry}

%screws up word count for some reason:
%\bibliographystyle{$HOME/Documents/submissions/mynewapa}
\bibliographystyle{apalike}


\begin{document}



\setlength\footnotesep{1em}






%---------------
%--- start paste




\title {Origins of Mind \\ Lecture 07}



\maketitle

\subsection{title-slide}


\section{Knowledge of Mind}

\subsection{slide-3}
The challenge is to explain the developmental emergence of mindreading.

Let me explain.

\subsection{slide-4}
\textit{Mindreading} is
the process of
identifying mental states and purposive actions
as the mental states and purposive actions of a particular subject.

\subsection{slide-5}
Researchers sometimes use the term ‘theory of mind’.

‘In saying that an individual has a theory of mind, we mean that the individual imputes mental states to himself and to others’
\citep[p.\ 515]{premack_does_1978}

\subsection{slide-6}
So, to be clear about the terminology, to have a theory of mind is just to be able to
to mindread,
that is, to identify mental states and purposive actions
as the mental states and purposive actions of a particular subject.

\subsection{slide-7}
So the challenge is to explain the emergence of mindreading.
You know (let's say) that Ayesha belives Beatrice is in the library.
Humans are not born knowing individuating facts about others' beliefs.
How do they come to be in a position to know such facts?
Meeting this challenge initially seems simple.
But, as you'll see, we quickly end up with a puzzle.
I think this puzzle requires us to rethink what is involved in having a conception of
the mental.

\subsection{slide-8}
I shall focus on awareness of others' beliefs to the exclusion of other mental states.

There's no theoretical reason for this; it's just a practical thing.

And what we learn about belief will generalise to other mental states.

\subsection{slide-9}
How can we test whether someone is able to ascribe beliefs to others?
Here is one quite famous way to test this, perhaps some of you are even aware of it
already.
Let's suppose I am the experimenter and you are the subjects.
First I tell you a story ...

In a standard \textit{false belief task}, `[t]he subject is aware that he/she and
another person [Maxi] witness a certain state of affairs x. Then, in the absence of
the other person the subject witnesses an unexpected change in the state of affairs
from x to y' \citep[p.\ 106]{Wimmer:1983dz}. The task is designed to measure the
subject's sensitivity to the probability that Maxi will falsely believe x to obtain.

\subsection{slide-13}
Here's the really surprising thing.

Children do really badly on this until they are around four years of age.

And they seem to develop the ability to pass this task only gradually, over months or years.

(There's something else that isn't surprising to most people but should be: adult humans not only nearly always provide the answer we're calling 'correct': they also believe that there is an obviously correct answer and that it would be a mistake to give any other answer.  I'll return to this point later.)

(NB: The figure is not Wimmer \& Perner's but drawn from their data.)

\subsection{slide-14}
There's been some stuff in the press recently about bad science, mainly some dodgy methods and failures to replicate.

\subsection{slide-15}
So you'll be pleased to know that a meta-study of 178 papers confirmed Wimmer \& Perner's findings.

Now there is clearly some variation here.

That's because different researchers implemented different versions of the original task.

We can use the meta-analysis of these experiments as a shortcut to finding out what sorts of factors affect children's performance.

\subsection{slide-16}
One factor that seems to make hardly any difference is whether you ask children about others' beliefs or their own beliefs.

To repeat, you get essentially the same results whether you ask children about others' beliefs or their own beliefs.

Children literally do not know their own minds.

\subsection{slide-17}
What happens if we involve the child by having her interact with the protagonist?

The task becomes easier for children of all ages, but the transition is essentially the same
          (participation does not interact with age \citealp[pp.\ 665-7]{Wellman:2001lz}).

\subsection{slide-18}
Finally, although there are some cultural differences, you get the same transition in seven diferent countries.

\subsection{slide-19}
So our challenge was to explain the emergence of mindreading.
At this point, up until around, it seemed quite straightforward to most researchers.
We seemed to know that children are unaware of mental states until around four years.
And a lot of studies looked at which factors affect their acquiring this awareness.
These studies showed that executive function, language and rich forms of social
interaction are all important.
All of this supported something like the story that Sellars tells in his famous Myth of
Jones.

\subsection{slide-20}
*todo*: describe Sellars' myth; link to Gopnik theory theory idea.

\subsection{slide-21}
But there was a big surprise in store for us.

\subsection{unit\_411}


\section{Infants Track False Beliefs}

\subsection{slide-23}
How can we test whether someone is able to ascribe beliefs to others?
Here is one quite famous way to test this, perhaps some of you are even aware of it
already.
Let's suppose I am the experimenter and you are the subjects.
First I tell you a story ...

In a standard \textit{false belief task}, `[t]he subject is aware that he/she and
another person [Maxi] witness a certain state of affairs x. Then, in the absence of
the other person the subject witnesses an unexpected change in the state of affairs
from x to y' \citep[p.\ 106]{Wimmer:1983dz}. The task is designed to measure the
subject's sensitivity to the probability that Maxi will falsely believe x to obtain.

Recall the experiment that got us started.

These experimenters added an anticipation prompt and measured to which box subjects looked first \citep{Clements:1994cw}.

(Actually they didn't use this story; theirs was about a mouse called Sam and some cheese, but the differences needn't concern us.)

\subsection{slide-26}
What got me hooked philosophical psychology,
and on philosophical issues in the development of mindreading in particular
was a brilliant finding by Wendy Clements who was Josef Perner's phd student.

These findings were carefully confirmed \citep{Clements:2000nc,Garnham:2001ql,Ruffman:2001ng}.

Around 2000 there were a variety of findings pointing in the direction of a confict between different measures.

These included studies on word learning \citep{Carpenter:2002gc,Happe:2002sr} and false denials \citep{Polak:1999xr}.

But relatively few people were interested until ...

\subsection{unit\_416}


\section{Mindreading: a Developmental Puzzle}

\subsection{slide-56}
The challenge is to explain the emergence, in evolution or development, of mindreading.
Initially it looked like this was going to be relatively straightforward and involve just language, social interaction and executive function.
So a Myth of Jones style story seemed viable.
But the findings of competence in infants of around one year of age changes this.
These findings tell us that not all abilities to represent others' mental states can depend on things like language.
And, as I've been stressing, these findings also create a puzzle.
The puzzle is, roughly, how to reconcile infants' competence with three-year-olds' failure.

*todo*: There are at least two possible puzzles you might focus on:
1. How can we avoid the apparent contradiction in the evidence?
2. How can we explain the apparent discrepancy between infants and 3-year-olds' performances?
Minimal theory of mind might resolve puzzle 1 but it won't by itself resolve puzzle 2.
This wasn't clear enough in the lecture (sorry!).

\subsection{slide-60}
Let me start by taking you back to the early eighties.
(Has anyone else been enjoying Deutschland drei und achtzig?)

3-year-olds fail false belief tasks

\subsection{slide-61}
3-year-olds fail a wide variety of tasks where they are asked about a
false belief, or  asked to predict how someone with a false belief will
act, ...

\subsection{slide-62}
or asked to predict what someone with a false belief will desire ...

\subsection{slide-63}
... or to retrodict or explain a false belief after being shown how someone acts.

\subsection{slide-64}
Further,
lots of factors make no difference to 3-year-olds’ performance: they fail
tasks about other’s beliefs and they fail tasks about their own beliefs;

\subsection{slide-65}
they fail when they are merely observers as well as when they are actively
involved;

\subsection{slide-66}
they fail when a verbal response is required and also when an
nonverbal communicative response or even a noncommunicative response (such
as hiding an object) is required.

\subsection{slide-67}
And they fail test questions which are word-for-word identical to desire and pretence tasks

\subsection{slide-68}
An A-Task is any false belief task that children tend to fail until around
three to five years of age.

\subsection{slide-69}
Why do children systematically fail A-tasks? There is a simple explanation ...

Children fail



because they rely on a model of minds and actions that does not incorporate beliefs


[Stress that, on this view, children do have a model of minds and actions.
It’s just that it doesn’t incorporate belief.]
Perner and others have championed the view that children who failed A-tasks
lack a metarepresentational understanding of propositional attitudes
altogether.  But this view has recently (well, not that recently, it’s nearly
a decade old now) been challenged by Hannes’ discovery
that children can solve tasks which are like A-tasks but involve incompatible
desires \citep{rakoczy:2007_desire}.  I think this make plausible the thought
that there is an age at which
children fail A-tasks not because they have a problem with mental states
in general, but because they have a problem with beliefs in particular.

\subsection{slide-70}
It turns out that, in the first and second years of life, infants show abities
to track false beliefs on a variety of measures.

\subsection{slide-75}
A non-A-Task is a task that is not an A-task.  I was tempted to call these
B-Tasks, but that would imply that they have a unity.  And whereas we know
from a meta-analysis that A-tasks do seem to measure a single underlying
competence, we don’t yet know whether all non-A-tasks measure a single thing
or whether there might be several different things.

\subsection{slide-76}
Why do infants systematically pass non-A-tasks in the first year
or two of life? There is a simple explanation ...

And this is *almost* everything we need to generate the puzzle about development.

\subsection{slide-77}
Children fail A-tasks
because they rely on a model of minds and actions that does not incorporate beliefs.

Children pass non-A-tasks
by relying on a model of minds and actions that does incorporate beliefs.

The dogma of mindreading: any individual has at most one model of minds and actions
at any one point in time.

We’ve seen that ...
To get a contradiction we need one further ingredient.

\subsection{slide-78}
These three claims are jointly inconsistent so one of them must be false.
Researchers disagree about which claim to reject.
But I suppose you can tell from how I’ve labelled them which one I
propose to reject.

\subsection{slide-80}
The puzzle is a little bit like the puzzle we had in the case of knowledge
of physical objects.
But it's also different.
In the case of physical objects, the conflict was between measures
involving looking and measures involving searching.
In this case it's different, because on the infant side there is not just
looking but also acting (e.g. helping) and even communicating.

\subsection{slide-87}
Children fail A-tasks
because they rely on a model of minds and actions that does not incorporate beliefs.

Children pass non-A-tasks
by relying on a model of minds and actions that does incorporate beliefs.

The dogma of mindreading: any individual has at most one model of minds and actions
at any one point in time.

How does this bear on the case of mindreading?

\subsection{slide-88}
Conjecture





Infants have core knowledge of minds and actions.



Core knowledge is sufficient for success on non-A-tasks.



Infants lack knowledge of minds and actions.



Knowledge is necessary for success on A-tasks.




Why accept this conjecture?
So far no reason has been given at all.
And it barely makes sense. There are just so many assumptions.
All I’m really saying is that I hope this case, knowledege of minds,
will turn out to be like the other cases.

\subsection{slide-89}
Why accept this conjecture?
And what form does the core knowledge take.
In every case so far, we have had to identify infant with adult competencies.
(Core knowledege is for life, not just for infancy.)

\subsection{unit\_421}


\section{Mindreading in Adults: Dual Processes}

\subsection{slide-91}
A process involves \emph{belief-tracking} if how processes of this type unfold
typically and nonaccidentally depends on facts about beliefs.
So belief tracking can, but need not, involve representing beliefs.

belief-tracking is sometimes  but not always   automatic

\subsection{slide-92}
A process is \emph{automatic} to the degree that whether it occurs is independent of its
relevance to the particulars of the subject's task, motives and aims.

\subsection{slide-93}
Or, more carefully, does belief tracking in human adults depend only
on processes which are automatic?

There is now a variety of evidence that belief-tracking is sometimes and
not always automatic in adults.  Let me give you just one experiment here
to illustrate.

\subsection{slide-105}
[skip this slide]

One way to show that mindreading is automatic is to give subjects a task which does not require tracking beliefs and then to compare their performance in two scenarios:
a scenario where someone else has a false belief, and a scenario in which someone else has a true belief.
If mindreading occurs automatically, performance should not vary between the two scenarios because others’ beliefs are always irrelevant to the subjects’ task and motivations.

\subsection{slide-106}
[skip this slide]

\citet{Schneider:2011fk} did just this.
They showed their participants a series of videos and instructed them to detect when a figure waved or, in a second experiment, to discriminate between high and low tones as quickly as possible.
Performing these tasks did not require tracking anyone’s beliefs, and the participants did not report mindreading when asked afterwards.

on experiment 1: ‘Participants never reported belief tracking when questioned in an open format after the experiment (“What do you think this experiment was about?”). Furthermore, this verbal debriefing about the experiment’s purpose never triggered participants to indicate that they followed the actor’s belief state’ \citep[p.~2]{Schneider:2011fk}

Nevertheless, participants’ eye movements indicated that they were tracking the beliefs of a person who happened to be in the videos.

In a further study, \citet{schneider:2014_task} raised the stakes by giving participants a task that would be harder to perform if they were tracking another’s beliefs.
So now tracking another’s beliefs is not only irrelevant to performing the tasks: it may actually hinder performance.
Despite this, they found evidence in adults’ looking times that they were tracking another’s false beliefs.
This indicates that ‘subjects … track the mental states of others even when they have instructions to complete a task that is incongruent with this operation’ \citep[p.~46]{schneider:2014_task} and so provides evidence for automaticity.%
\footnote{%
% quote is necessary to qualify in the light of their interpretation; difference between looking at end (task-dependent) and at an earlier phase (task-independent)?
%\citet[p.~46]{schneider:2014_task}: ‘we have demonstrated here that subjects implicitly track the mental states of others even when they have instructions to complete a task that is incongruent with this operation. These results provide support for the hypothesis that there exists a ToM mechanism that can operate implicitly to extract belief like states of others (Apperly \& Butterfill, 2009) that is immune to top-down task settings.’
It is hard to completely rule out the possibility that belief tracking is merely spontaneous rather than automatic.
I take the fact that belief tracking occurs despite plausibly making subjects’ tasks harder to perform to indicate automaticity over spontaneity.
If non-automatic belief tracking typically involves awareness of belief tracking, then the fact that subjects did not mention belief tracking when asked after the experiment about its purpose and what they were doing in it further supports the claim that belief tracking was automatic.
}

Further evidence that mindreading can occur in adults even when counterproductive has been provided by \citet{kovacs_social_2010}, who showed that another’s irrelevant beliefs about the location of an object can affect how quickly people can detect the object’s presence,
and by \citet{Wel:2013uq}, who showed that the same can influence the paths people take to reach an object.
Taken together, this is compelling evidence that mindreading in adult humans sometimes involves automatic processes only.

\subsection{slide-107}
belief-tracking is sometimes  but not always automatic

\subsection{slide-112}
Children fail A-tasks
because they rely on a model of minds and actions that does not incorporate beliefs.

Children pass non-A-tasks
by relying on a model of minds and actions that does incorporate beliefs.

The dogma of mindreading: any individual has at most one model of minds and actions
at any one point in time.

How might this help us with the puzzle about development?

\subsection{slide-113}
Recall this conjecture from earlier ...

Conjecture





 Infants have core knowledge of minds and actions.



Core knowledge is sufficient for success on non-A-tasks.



 Infants lack knowledge of minds and actions.



 Knowledge is necessary for success on A-tasks.











The first challenge was to say what core knowledge of minds might be ...









core knowledge of minds = the representations underpining automatic belief-tracking?




\subsection{slide-114}
The first challenge was to say what core knowledge of minds might be ...

\subsection{slide-117}
So far we have no evidence for the conjecture!

\subsection{unit\_441}


\section{Minimal Theory of Mind}

\subsection{unit\_451}


\section{Signature Limits}

\subsection{slide-149}
There is  some evidence that this prediction is correct.
Jason Low and his collegaues set out to test it.
They have now published three
different papers showing such limits; and Hannes Rakoczy and others
have more work in progress on this.
Collapsing several experiements using different approaches,
the basic pattern of their findings is this ...

Take non-automatic responses first; in this case, communicative responses.
When you do a false-belief-identity task, you see the pattern you also
find for false-belief-locations tasks.
But things look different when you measure non-automatic responses ...

The non-automatic responses all show the signature limit of minimal
models of the mental.
This is evidence for the hypothesis that
Some automatic belief-tracking systems rely on minimal models of the mental.

I also hear that quite a few scientists have pilot data that speaks
against this signature limit.

One particular task for future research will be to examine whether other
automatic responses to scenarios involving false beliefs about identity,
such as response times and movement trajectories, are also subject
to this signature limit.

\subsection{slide-150}
Just say that you can do this with other stimuli and paradigms, and we
have done this with infants and would like to do it with adults.

These findings complicate the picture: is helping driven by automatic processes only?
If not, why do we predict that the signature limit of minimal theory of mind
is found in this case too?

\subsection{slide-152}
Look at the three year olds.
What might make us think that three year old’s responses are a consequence
of the same system that underpin’s adults’ automatic responses?
One compelling consideration is that three year old’s responses
manifest to the same signature limit as adults’.

same signature limit -> same system

\subsection{slide-153}
Scott and colleagues \citep{scott:2015_infants} provided other evidence
suggesting that infants’ mindreading may be relatively sophisticated.
Specifically, 17-month-olds watched a thief attempt to steal a preferred
object (a rattling toy) when its owner was momentarily absent by substituting
it with a less-preferred object (a non-rattling toy). Infants looked longer
when the thief substituted the preferred object with a non-visually-matching
silent toy compared to when the thief substituted it with a visually-matching
silent toy. The authors postulated that infants can ascribe to the thief an
intention to implant in the owner a false-belief about the identity of the
substituted toy. The authors further suggested that infants make such ascriptions
only when the substitution involves a visually-matching toy and the owner will
not test whether the toy rattles on her return.

However, Scott et al.’s \citep{scott:2015_infants} explanations also
require postulating that infants take the thief to be strikingly inept;
despite having opportunity simply to pilfer from a closed box known to
contain at least three rattling toys, the thief engages in elaborate
deception which will be uncovered whenever the substituted toy is next
shaken and the thief, as sole suspect, easily identified. A further
difficulty is that factors unrelated to the thief’s mental states vary
between conditions, such as the frequencies with which toys visually matching
one present during the final phase of the test trial have rattled. These
considerations jointly indicate that further evidence would be needed to
support the claim that humans’ early mindreading capacity enables them to
ascribe intentions concerning false beliefs involving numerical identity.

\subsection{slide-154}
It has to be said that not everyone is convinced ..

‘the theoretical arguments offered [...] are [...] unconvincing, and [...]
the data can be explained in other terms’
(\citealp{carruthers:2015_two}; see also \citealp{carruthers:2015_mindreading}).

What is my response?
Yes, the data can be explained in other terms, at least post hoc; and certainly
there is as yet insufficient data for certainty.
What about the theoretical arguments?
Partners in crime defence ... theoretical arguments
for multiple systems for belief are the same as the theoretical arguments
for physical cognition or number cognition (but that’s a different talk).

\subsection{slide-157}
So let me conclude.
The challenge we have been addressing was to understand the emergence of
mindreading.
Initially this seemed straightforward: you learn this from social
interaction using language as a tool (compare Gopnik's theory theory).
However, the discovery that abilities to track beilefs exist in infants
from around 7 months or earlier initially suggested a different picture:
one on which mindreading was likely to involve core knowledge. But, as
always, things are not so straightforward.

\subsection{slide-158}
Children fail A-tasks
because they rely on a model of minds and actions that does not incorporate beliefs.

Children pass non-A-tasks
by relying on a model of minds and actions that does incorporate beliefs.

The dogma of mindreading: any individual has at most one model of minds and actions
at any one point in time.

The puzzle.

\subsection{slide-159}
Recall this conjecture from earlier ...

Conjecture





 Infants have core knowledge of minds and actions.



Core knowledge is sufficient for success on non-A-tasks.



 Infants lack knowledge of minds and actions.



 Knowledge is necessary for success on A-tasks.











The first challenge was to say what core knowledge of minds might be ...









core knowledge of minds = the representations underpining automatic belief-tracking?




\subsection{slide-160}
The first challenge was to say what core knowledge of minds might be ...

\subsection{slide-163}
So far we have no evidence for the conjecture!

\subsection{slide-164}
theme A: explain the origins of knowledge of others minds : development as
rediscovery. There is a modular capacity ( = core knowledge). But this
doesn't lead to adult-like understanding for years, and the acquisition of
adult-like understanding hinges on language; may involve completely
different model of mental states.






%--- end paste
%---------------






\bibliography{$HOME/endnote/phd_biblio}



\end{document}
