% !TEX encoding = UTF-8 Unicode

%\def \paprsize {a5paper}
\def \papersize {a4paper}
%\def \papersize {letterpaper}

%\documentclass[14pt,\papersize]{extarticle}
\documentclass[12pt,\papersize]{extarticle}
% extarticle is like article but can handle 8pt, 9pt, 10pt, 11pt, 12pt, 14pt, 17pt, and 20pt text

\def \ititle {Origins of Mind}
\def \isubtitle {Pre-approved Essay Questions}
%comment some of the following out depending on whether anonymous
\def \iauthor {Stephen A.\ Butterfill}
\def \iemail{s.butterfill@warwick.ac.uk% \& corrado.sinigaglia@unimi.it
}
%\def \iauthor {}
%\def \iemail{}
%\date{}

% \usepackage{bibentry}
% \nobibliography{$HOME/endnote/phd_biblio}

%\input{$HOME/Documents/submissions/preamble_steve_paper3}
\input{$HOME/latex_imports/preamble_steve_report2_pdftex}

% disables chapter, section and subsection numbering
\setcounter{secnumdepth}{-1}

%avoid overhang
\tolerance=5000

% no indent, space between paragraphs:
\usepackage[parfill]{parskip}

\usepackage[toc]{multitoc}
\renewcommand*{\multicolumntoc}{2}
\setlength{\columnseprule}{0.5pt}
\begin{document}



\setlength\footnotesep{1em}


\maketitle

%\renewcommand{\cftsecfont}{\normalsize}
\setcounter{tocdepth}{1}

\noindent


\textit{\textbf{Not approved}
These questions need to be confirmed by the external examiner.  This is yet to happen.}


\textit{For any of these questions, your answer may focus on a particular domain, such as core knowledge of objects or of number.  You are not required to provide a comprehensive survey.}

\emph{The readings suggested here are to get you started.  Further reading can be found on the lecture handouts.  You may discuss readings with your tutor in relation to your essay plan.}

\minitoc

\tableofcontents
\clearpage

\section{Physical Objects}
What do four-month-old infants know of objects they are not perceiving?

\subsubsection{---Reading}
\fullcite{baillargeon:1987_object}

\fullcite{Shinskey:2001fk}

\fullcite{Spelke:1998im}

\fullcite{Aguiar:2002ob}

\fullcite{charles:2009_object}

\fullcite{mccurry:2009_beyond}

\fullcite{moore:2010_numerical}

\fullcite{Spelke:1990jn}

\fullcite{Spelke:2001pg}

\clearpage

\section{Causation}

What do 6-month-olds know about physical objects’ causal interactions?

\subsection{---Reading}

\fullcite{Spelke:1993no}

\fullcite{Hood:2000bf}

\fullcite{spelke:1992_origins}

\fullcite{Leslie:1987nr}

\fullcite{Hood:2003yg}

\fullcite{santos:2006_cotton-top}

\fullcite{Haith:1998aq}

\fullcite{Keen:2003xz}

\clearpage

\section{Core knowledge}
What is core knowledge and what role, if any, could it play in explaining the transition from being unable to know things to being able to know things?

[Your answer may focus on a  single domain,  such as knowledge of objects.]

\subsubsection{---Reading}

\fullcite{Carey:1996hl}

\fullcite{spelke:1992_origins}

\fullcite{Spelke:2007hb}

\fullcite{carey:2009_origin}



\clearpage

\section{Mindreading}
What is the puzzle about when humans can first represent others’ beliefs?
How might the puzzle be resolved?

\subsubsection{---Reading}
\fullcite{Onishi:2005hm}

\fullcite{kovacs_social_2010}

\fullcite{Baillargeon:gx}

\fullcite{butterfill_minimal}

\fullcite{carruthers:2013_mindreading}

\fullcite{low:2016_cognitive}



\clearpage



\section{Knowledge of colour}
At birth humans do not know this lime fruit is green whereas that tomato is red.
How could some humans come to be in a position to know this?

Hint: you should discuss categorical perception of colour and its relation to knowledge.  There was a lecture on this topic; the handout includes many references.

\subsubsection{---Reading}

\fullcite{Kowalski:2006hk}

\fullcite{Franklin:2005hp}

\fullcite{Franklin:2005xk}

%\fullcite{franklin:2010_hemispheric}

\fullcite{Wiggett:2008xt}


%
%
% \section{Social interaction}
% If humans were incapable of social interaction and could only observe each other from behind one-way mirrors (if such a thing existed), how if at all would this affect the processes by which they get to know things?
%
% \subsubsection{---Reading}
%
%
% \fullcite{tomasello:2008origins}
%
% \fullcite{Tomasello:2005wx} [Read the commentaries.]
%
% \fullcite{butterfill:2012_interacting}
%

\clearpage

\section{The Teleological Stance}
Is it true that ‘when taking the teleological stance one-year-olds apply the same inferential principle of rational action that drives everyday mentalistic reasoning about intentional actions in adults’?

\subsubsection{---Reading}
\fullcite{Gergely:2003gb}

\fullcite{Gergely:1995sq}

\fullcite{Woodward:1998dm}

\fullcite{green:2016_culture}

\fullcite{daum:2012_actions}

\fullcite{sinigaglia:2015_goal_ascription}

\fullcite{Csibra:2007hm}


\clearpage


\section{Action}
How and why are infants’ abilities to perform actions linked to their abilities to track the goals of others’ actions?

\subsubsection{---Reading}
\fullcite{woodward:2009_infants}

\fullcite{sommerville:2005_action}

\fullcite{sommerville:2008_experience}

\fullcite{ambrosini:2013_looking}

\fullcite{melzer:2012_production}

\fullcite{sinigaglia:2015_goal_ascription}

\fullcite{Csibra:2007hm}
\fullcite{Gergely:2003gb}


\clearpage

\section{Joint action}

What is joint action? Could there be a role for joint action in explaining the developmental origins of knowledge?

\subsubsection{---Reading}

\fullcite{Bratman:2009lv}

\fullcite{Moll:2007gu}

\fullcite{carpenter:2009_howjoint}

\fullcite{Tomasello:2007gl}

\fullcite{Tollefsen:2005vh}

\fullcite{Butterfill:2011fk}

\clearpage



\section{Referential Communication}

What underpins one-year-olds’ abilities to produce and comprehend pointing actions?



\subsubsection{---Hint}

You may consider this view as a target for discussion:

\begin{quote}
‘infant pointing is best understood---on many levels and in many ways---as depending on uniquely human skills and motivations for cooperation and shared intentionality, which enable such things as joint intentions and joint attention in truly collaborative interactions with others (Bratman, 1992; Searle, 1995).’  \citep[p.~706]{Tomasello:2007fi}
\end{quote}


\begin{quote}
‘to understand pointing, the subject needs to understand more than the individual goal-directed behaviour. She needs to understand that by pointing towards a location, the other attempts to communicate to her where a desired object is located; that the other tries to inform her about something that is relevant for her’
\citep[p.\ 6]{Moll:2007gu}.
\end{quote}



\subsubsection{---Reading}

\fullcite{Tomasello:2007fi}

\fullcite {liszkowski:2004_twelve}

\fullcite{Liszkowski:2007mm}

\fullcite{Liszkowski:2008al}

\fullcite{Moll:2007gu}

\fullcite{Liebal:2010lr}

\fullcite[compare][chapter 14]{Grice:1989ha}

\fullcite{Baldwin:1995xl}

\fullcite{Csibra:2003kp}

\clearpage

\section{Language}
Do ‘children learn words through the exercise of reason’?

\textit{The reading for this is one-sided, which makes this question difficult.}


\subsubsection{---Reading}

\fullcite{Bloom:2000qz}

\fullcite{Baldwin:2000qq}

\fullcite{Sabbagh:2001sp}

\fullcite{Matthews:2008yi}

\fullcite{Dummett:1993xn}

\fullcite{Goldin-Meadow:2003pj}


%nativism topic readings
%on animals:
% \fullcite{chiandetti:2011_chicks_op}
% \fullcite{haun:2010_origins}
%philosophy:
% \fullcite{Samuels:2004ho}


%
% \section{What is ‘shared intentionality’ and what might it explain?}
%
% Compare \citet[p. 688]{Tomasello:2005wx}: ‘Our ... hypothesis is that it is precisely these two developing capacities [to read intentions and to share psychological states]  that interact during the first year of life to create the normal human developmental pathway leading to participation in collaborative cultural practices.’
%
% %How (if at all) could  shared intentionality be involved in creating ‘the normal human developmental pathway leading to participation in collaborative cultural practices’?
%
% %(Tomasello, Carpenter, et al. 2005: 675): “it is only if a young child understands other persons as intentional agents that she can acquire and use linguistic symbols—because the learning and use of symbols requires an understanding that the partner can voluntarily direct actions and attention to outside entities.”
%
%
% \subsubsection{---Reading: theoretical}
% Pick one of these (first is probably the best; last is shortest; the middle one comes with critical commentaries).
%
% *\fullcite{Moll:2007gu}
%
% \fullcite{Tomasello:2005wx}
%
% \fullcite{Tomasello:2007gl}
%
%
% \subsubsection{---Reading: experimental (pick one or more)}
%
% \fullcite{Moll:2008fh}
%
% \fullcite{Warneken:2006qe}
%
% \fullcite{Tomasello:2007fi}
%
% \subsubsection{---Reading: background}
%
% \fullcite{Bratman:1992mi}
%
% \fullcite{Searle:1990em}
%
% \fullcite{Butterfill:2011fk}
%
%
%

\clearpage

\section{Innateness}
What if anything is innate in humans?

Hint: You should be careful to examine the notion of innateness  (see \citealp{Samuels:2004ho}).  Otherwise the reading is divided into topics; you should not try to cover all topics.  I also suggest \emph{not} structuring your essay by topic.

\subsubsection{---Reading}

\fullcite{Samuels:2004ho}

\subsubsection{---Reading: comparative (cross-species)}

\fullcite{chiandetti:2011_chicks_op}

\fullcite{haun:2010_origins}


\subsubsection{---Reading: syntax}

Note: this is one-sided.

\fullcite{lidz:2003_what}

\fullcite{lidz:2004_reaffirming}




\subsubsection{---Reading: replying to Fodor’s argument}

\fullcite{Fodor:1981ep}

\fullcite{carey:2009_origin} chapters 4, 8

(There is also an exchange between Carey and Rey forthcoming in the journal Mind and Language---their papers may be available by the time you read this.)



%\bibliography{$HOME/endnote/phd_biblio}



\end{document}
